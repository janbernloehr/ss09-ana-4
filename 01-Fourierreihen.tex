\section{Fourierreihen}

Fourierreihen sind eine Verallgemeinerung von orthonormalen Basen auf unendlich
dimensionale Vektorräume. Im Folgenden wollen wir untersuchen, wie sich
bekannte Aussagen aus der Linearen Algebra auf unendlich dimensionale
Vektorräume übertragen lassen.

\subsection{Orthonormalsysteme}

\begin{defn}
\label{defn:1.1}
Sei $L$ ein linearer Raum über $\C$. Eine Abbildung
\begin{align*}
\lin{\cdot,\cdot} : L\times L \to \C,
\end{align*}
heißt \emph{Skalarprodukt}, falls die folgenden Eigenschaften erfüllt sind.
\begin{enumerate}[label=(S\arabic{*})]
  \item\label{defn:1.1:1} $\lin{\alpha x_1+\beta x_2,y} =
  \alpha\lin{x_1,y}+\beta\lin{x_2,y}, \forall x_1,x_2,y\in L,\;
  \alpha,\beta\in\C$.
  \item\label{defn:1.1:2} $\lin{x,y} = \overline{\lin{y,x}}, \forall x,y\in L$.
  \item\label{defn:1.1:3} $\lin{x,x} > 0$, $\forall x\in L\setminus\{0\}$.
\end{enumerate}
$(L,\lin{\cdot,\cdot})$ heißt \emph{Prähilbertraum}.\fishhere
\end{defn}

\begin{bem}
\label{bem:1.2}
$\ref{defn:1.1:2}\Rightarrow \lin{x,x} = \overline{\lin{x,x}} \in\R$.
$\ref{defn:1.1:1}\Rightarrow \lin{0,0} = 0$.\\
$\ref{defn:1.1:1}\land\ref{defn:1.1:2}\Rightarrow \lin{x,\alpha y_1+\beta y_2}
= \overline{\alpha}\lin{x,y_1} + \overline{\beta}\lin{x,y_2}$.\maphere
\end{bem}

\begin{prop}
\label{prop:1.3}
Durch $\norm{x} := \sqrt{\lin{x,x}}$ wird eine Norm auf $L$ definiert. Es gilt
die Cauchy-Schwartz-Bounjakowski Ungleichung,
\begin{align*}
\abs{\lin{x,y}} \le \norm{x}\norm{y}.\fishhere
\end{align*}
\end{prop}
\begin{proof}
Der Nachweis der Normeigenschaften ist eine leichte Übung. Die
Cauchy-Schwartz-Bounjakowski Ungleichung ist ein Spezialfall der
Hölder-Ungleichung.\qedhere
\end{proof}

\begin{cor}
\label{prop:1.4}
Das Skalarprodukt ist stetig bezüglich der induzierten Norm, d.h. für
\begin{align*}
\norm{x-x_n}, \norm{y-y_n}\to0\quad\Rightarrow\quad
\abs{\lin{x,y}-\lin{x_n,y_n}}\to0.\fishhere
\end{align*}
\end{cor}
\begin{proof}
Seien $(x_n), (y_n)$ Folgen und $x_n\to x, y_n\to y$, dann gilt
\begin{align*}
\abs{\lin{x_n,y_n}-\lin{x,y}} &= \abs{\lin{x_n-x,y_n}-\lin{x,y-y_n}}\\
&\le \abs{\lin{x_n-x,y_n}} + \abs{\lin{x,y-y_n}}\\
&\le \norm{x_n-x}\norm{y_n} + \norm{x}\norm{y-y_n} < \ep,
\end{align*}
für $n$ hinreichend groß, da $y_n$ konvergent und daher beschränkt.\qedhere
\end{proof}

\begin{defn}
\label{defn:1.5}
$(L,\lin{\cdot,\cdot})$ heißt \emph{Hilbertraum}, falls $L$
vollständig bezüglich der durch $\lin{\cdot,\cdot}$ induzierten Norm
ist.\fishhere
\end{defn}

\begin{bsp}
\label{bsp:1.6}
\begin{enumerate}[label=\arabic{*}.)]
  \item Das Standardbeispiel für den Hilbertraum ist der Raum der
  quadratsummierbaren Folgen,
\begin{align*}
L = \setdef{(x_j)}{\sum\limits_{j=1}^\infty \abs{x_j}^2 < \infty},
\end{align*}
mit dem Skalarprodukt
\begin{align*}
\lin{(x_j),(y_j)} = \sum\limits_{j=1}^\infty x_j\overline{y_j}.
\end{align*}
\item $L:= C([a,b]\to\C)$ mit
\begin{align*}
\lin{f,g} := \int_a^b f(x)\overline{g(x)}\dx,
\end{align*} 
ist lediglich ein Prähilbertraum, man kann $L$ jedoch zu $\LL^2$
erweitern mit,
\begin{align*}
\lin{f,g}_{\LL^2} = \int_{[a,b]} f\overline{g}\dmu.
\end{align*}
$(\LL^2,\lin{\cdot,\cdot}_{\LL^2})$ ist ein Hilbertraum und es gilt,
$L\subseteq\LL^2$, $\lin{f,g}_{\LL^2} = \lin{f,g}$ für $f,g\in L$ und $L$ liegt
dicht in $\LL^2$.
\bsphere
\end{enumerate}
\end{bsp}

\begin{prop}
\label{prop:1.7}
Zu jedem Prähilbertraum $(L,\lin{\cdot,\cdot})$ existiert ein bis auf
Isomorphie eindeutiger Hilbertraum $(H,\lin{\cdot,\cdot}_H)$ mit $L\subseteq
H$, $\lin{\cdot,\cdot}_H\big|_L = \lin{\cdot,\cdot}$ und $L$ liegt dicht in
$H$.
$(H,\lin{\cdot,\cdot}_H)$ heißt die \emph{Vervollständigung} von $L$.\fishhere
\end{prop}

\begin{defn}
\label{defn:1.8}
Eine Folge $(e_j)$ im Prähilbertraum $(L,\lin{\cdot,\cdot})$ heißt
\emph{Orthonormalsystem (ONS)}, falls $\lin{e_k,e_j} = \delta_{kj}$.\fishhere
\end{defn}

\begin{prop}
\label{prop:1.9}
Sei $(e_j)$ ein ONS. Dann gilt,
\begin{enumerate}
  \item $x=\sum\limits_{j=1}^\infty x_j e_j \Rightarrow x_j = \lin{x,e_j}$.
  \item $\{e_j\}$ ist linear unabhängig, d.h. jede endliche Teilmenge ist
  linear unabhängig.\fishhere
\end{enumerate}
\end{prop}

\begin{prop}
\label{prop:1.10}
Sei $(e_j)$ ein ONS im Prähilbertraum $(L,\lin{\cdot,\cdot})$, $x\in L$. Dann
gilt
\begin{enumerate}[label=\arabic{*}.)]
\item\label{prop:1.10:1} \emph{Besselsche Ungleichung}
\begin{align*}
\sum\limits_{j=1}^\infty \abs{\lin{x,e_j}}^2 \le \norm{x}^2,
\end{align*}
insbesondere ist die Reihe konvergent.
\item\label{prop:1.10:2} \emph{Parsevallsche Gleichung}
\begin{align*}
x = \sum\limits_{j=1}^\infty \lin{x,e_j}e_j \Leftrightarrow \norm{x}^2 =
\sum\limits_{j=1}^\infty \abs{\lin{x,e_j}}^2.
\end{align*}
\item\label{prop:1.10:3} $M:= \setdef{x\in L}{x = \sum\limits_{j=1}^\infty
\lin{x,e_j}e_j}$ ist abgeschlossen in $L$.
\item\label{prop:1.10:4} Ist $L$ ein Hilbertraum, so ist
$\sum\limits_{j=1}^\infty \lin{x,e_j}e_j$ immer konvergent (nicht zwangsläufig gegen $x$).\fishhere
\end{enumerate}
\end{prop}
\begin{proof}
\ref{prop:1.10:1}+\ref{prop:1.10:2}
\begin{align*}
0 &\le \norm{x-\sum\limits_{j=1}^N\lin{x,e_j}e_j}^2 = \lin{x,x} -
\sum\limits_{j=1}^N \overline{\lin{x,e_j}}\lin{x,e_j} \\
&- \sum\limits_{j=1}^N
\lin{x,e_j}\overline{\lin{x,e_j}} + \sum\limits_{j=1}^N
\lin{x,e_j}\overline{\lin{x,e_j}}
 =\norm{x}^2 - \sum\limits_{j=1}^N \abs{\lin{x,e_j}}^2.
\end{align*}
\ref{prop:1.10:4}
\begin{align*}
\norm{\sum\limits_{j=N}^M \lin{x,e_j}e_j}^2 \overset{\ref{prop:1.10:2}}{=}
\sum\limits_{j=N}^M \abs{\lin{x,e_j}}^2 < \ep, 
\end{align*}
für $N,M > N_\ep$.

\ref{prop:1.10:3}
Sei nun $x\in\overline{M}^L$. Zeige $x\in M$ bzw. $x=\sum\limits_{j=1}^\infty
\lin{x,e_j}e_j$. Die Reihe,
\begin{align*}
\sum\limits_{j=1}^\infty
\underbrace{\lin{x,e_j}e_j}_{\in M}
\end{align*}
konvergiert in $H$ und daher ist $y:=x-\sum\limits_{j=1}^\infty
\lin{x,e_j}e_j\in\overline{M}^H$. Für jedes $k\in\N$ gilt,
\begin{align*}
\lin{y,e_k}_H = \lin{x,e_k} - \sum\limits_{j=1}^{\infty}\lin{x,e_k}\delta_{jk}
= 0.
\end{align*}
$y$ ist also orthogonal zu $e_k$ für $k\in\N$ und damit auch zu allen
Linearkombinationen. Aufgrund der Stetigkeit des Skalarprodukts gilt also auch
\begin{align*}
\lin{y,u} = 0,\forall u\in\overline{M}^H.
\end{align*}
Da $y\in\overline{M}^H$ gilt insbesondere $\lin{y,y} = 0$ und daher ist $y =
0$, also geht
\begin{align*}
\norm{\underbrace{x-\sum\limits_{j=1}^N
\lin{x,e_j}e_j}_{\in L}}_H \to 0,
\end{align*}
für $N\to\infty$, wobei die $H$-Norm hier für endliches $N$ eine $L$-Norm
ist.\qedhere
\end{proof}

\begin{bsp}
\label{bsp:1.11}
Sei $L:=C([0,\pi]\to\C)$ und $e_j: x\mapsto \sqrt{\frac{2}{\pi}}\sin(jx)$.

\ref{prop:1.10:3} besagt nun, dass die Menge
\begin{align*}
\setdef{f\in C([0,\pi]\to\C)}{f(\cdot) = \sum\limits_{j=1}^\infty
\lin{f,e_j}e_j(\cdot)}
\end{align*}
abgeschlossen in $L$ ist.

\ref{prop:1.10:4} besagt, dass für jedes $f\in L$ die Reihe
\begin{align*}
\sum\limits_{j=1}^\infty \lin{f,e_j}\sqrt{\frac{2}{\pi}} \sin(j,\cdot)
\end{align*}
in $\LL^2([0,\pi])$ konvergiert. Das heißt aber nicht, dass sie auch punktweise
gegen $f$ konvergiert!\bsphere
\end{bsp}

\begin{defn}
\label{defn:1.12}
Sei $(L,\lin{\cdot,\cdot})$ ein Prähilbertraum, $(e_j)$ ein ONS.
\begin{enumerate}[label=\arabic{*}.)]
  \item Für $x\in L$ heißt
\begin{align*}
\sum\limits_{j=1}^\infty \lin{x,e_j}e_j
\end{align*}
die \emph{Fourierreihe} von $x$, $\lin{x,e_j}$ heißt $j$-ter
\emph{Fourierkoeffizient}.
\item $(e_j)$ heißt \emph{vollständiges Orthogonalsystem (VONS)}, falls
\begin{align*}
\forall x\in L\; :\; x = \sum\limits_{j=1}^\infty \lin{x,e_j}e_j.\fishhere
\end{align*}
\end{enumerate}
\end{defn}

\begin{prop}
\label{prop:1.13}
Sei $(H,\lin{\cdot,\cdot})$ ein Hilbertraum mit VONS $(e_j)$. Dann ist
\begin{align*}
\phi: H\to l_2,\; x\mapsto \left(\lin{x,e_j}\right)_{j\in\N},
\end{align*}
ein Hilbertraum-Isomorphismus\footnote{Ein Hilbertraumisomorphismus ist eine
bijektive, lineare und Skalarprodukt erhaltende Abbildung.}.\fishhere
\end{prop}

\subsection{Operatoren und Eigenwerte}

Wir wollen jetzt konkrekt mit Fourierreihen rechnen. Dazu werden wir die
Zusammenhänge von Fourierreihen und Operatoren studieren und beispielsweise
Ableitungsoperatoren durch Fourierreihen beschreiben.

\begin{defn}
\label{defn:1.14}
\begin{enumerate}[label=\arabic{*}.)]
  \item Sei $D(A)$ eine lineare Teilmenge von $L$. Ein Operator $A$ in $L$ ist
  eine lineare Abbildung,
\begin{align*}
A: D(A) \to L.
\end{align*}
\item $A$ heißt \emph{symmetrisch}\footnote{Wir fordern hier
hermitesch aber da wir hauptsächlich über $\C$ arbeiten, kürzen wir dies zu
symmetrisch.}, falls $\forall x,y\in D(A)$ gilt,
\begin{align*}
\lin{Ax,y} = \lin{x,Ay}.
\end{align*}
\item $\lambda\in\C$ heißt \emph{Eigenwert} von $A$, falls
\begin{align*}
\exists x\in D(A)\setminus\{0\} : Ax = \lambda x.
\end{align*}
Ein solches $x$ heißt \emph{Eigenvektor} von $A$ zum Eigenwert $\lambda$.
\begin{align*}
N(\lambda) := \dim\ker (A-\lambda\Id),
\end{align*}
heißt \emph{geometrische Vielfachheit} von $\lambda$.\fishhere
\end{enumerate}
\end{defn}
\begin{prop}
\label{prop:1.15}
\begin{enumerate}[label=\arabic{*}.)]
  \item Ist $A$ symmetrisch, sind alle Eigenwerte reell.
  \item Eigenvektoren zu verschiedenen Eigenräumen sind orthogonal.\fishhere
\end{enumerate}
\end{prop}

\begin{bsp}
\label{bsp:1.16}
Sei $(e_n)$ ein ONS im Hilbertraum $(H,\lin{\cdot,\cdot})$ und $(\lambda_j)$
Folge in $\C$ und,
\begin{align*}
&D(A) := \setdef{x\in H}{\sum\limits_{j=1}^\infty \abs{\lambda_j \lin{x,e_j}}^2
< \infty},\\
&Ax := \sum\limits_{j=1}^\infty \lambda_j \lin{x,e_j}e_j,\quad \text{für } x\in
D(A).\tag{*}
\end{align*}
\begin{enumerate}[label=\arabic{*}.)]
  \item $A$ ist linear.
  \item Falls $(\lambda_j)$ reell ist $A$ symmetrisch
\begin{align*}
\lin{Ax,y} &= \lim\limits_{N\to\infty} \sum\limits_{j=1}^N
\lin{\lambda_j\lin{x,e_j}e_j,y} = \lim\limits_{N\to\infty} \sum\limits_{j=1}^N
\lambda_j\lin{x,e_j}\lin{e_j,y}\\
 &= \lim\limits_{N\to\infty} \sum\limits_{j=1}^N \lin{x,\lambda_j
 \lin{y,e_j}e_j}= \lin{x,Ay}.
\end{align*}
\item
\begin{enumerate}[label=(\roman{*})]
\item Alle $\lambda_j$ sind Eigenwerte.
\item Falls $\exists x\in L\setminus\{0\} : \forall j\in\N \lin{x,e_j} = 0$
gilt $Ax = 0$.
\item Es gibt keine weiteren Eigenwerte außer $0,\lambda_j$.

Sei $x\in D(A)\setminus\{0\}$ mit $Ax = \lambda x$ und $\lambda_j \neq\lambda$,
dann
\begin{align*}
\lambda\lin{x,e_k}  = \lin{\lambda x,e_k} = \lin{Ax,e_k} = \lambda_k
\lin{x,e_k}
\end{align*}
Daher ist $\lin{x,e_k} = 0\forall k\in\N$ und $\lambda = 0$.
\end{enumerate}
\item Falls $(\lambda_j)$ beschränkt, gilt $D(A) = H$. (Bessel)
\item Falls $(e_j)$ ein VONS, folgt $D(A)$ liegt dicht in $H$. Denn sei $x\in
H$, wähle $N$ mit
\begin{align*}
\norm{x-\underbrace{\sum\limits_{j=1}^N \lin{x,e_j}e_j}_{\in D(A)}} < \ep.
\end{align*}
\end{enumerate}
\begin{bemn}[Ausblick.]
$(e_j)$ VONS und $(\lambda_j)$ reell $\Rightarrow $ $A$ ist
selbstadjungiert.\bsphere
\end{bemn}
\end{bsp}

\begin{defn}
\label{defn:1.17}
Die Darstellung (*) heißt \emph{Spektraldarstellung} des Operators $A$. Der
Abschluss der Menge aller Eigenwerte,
\begin{align*}
\sigma(A) = \overline{\setdef{\lambda\in\C}{\lambda\text{ ist Eigenwert von }A}}
\end{align*}
heißt \emph{Spektrum} von $A$.

Ein Operator, der eine Spektraldarstellung besitzt, heißt \emph{diskreter
Operator}.\fishhere
\end{defn}

Wir werden später sehen, dass es auch selbstadjungierte Operatoren gibt, die
keine Spektraldarstellung besitzen.

\begin{bsp}
\label{bsp:1.18}
Sei $H=\LL^2([0,\pi])$ sowie,
\begin{align*}
&D(A) = \setdef{f\in C^2([0,\pi]\to\C)}{f(0) = f(\pi) = 0},\\
&Af = -f'',\qquad \text{für } f\in D(A).
\end{align*}
\begin{enumerate}[label=\arabic{*}.)]
  \item $A$ ist symmetrisch, denn für $f,g\in D(A)$ gilt
\begin{align*}
\lin{Af,g} &= -\int_0^\pi f''\overline{g} \dx\\
&= \underbrace{-\left[ f'(x)\overline{g(x)} - f(x)\overline{g'(x)}
\right]\bigg|_0^\pi}_{=0} -\int_0^\pi f\overline{g''} \dx
= \lin{f,Ag}
\end{align*}
wie mit zweifacher partieller Integration folgt.
\item $A$ ist positiv definit, d.h. für $f\in D(A)$ gilt $\lin{Af,f} \ge 0$
und $\lin{Af,f} = 0 \Leftrightarrow f=0$
\begin{align*}
\lin{Af,f} = \int_0^\pi f'(x)\overline{f'(x)}\dx = \int_0^\pi \abs{f'}^2\dx
\end{align*}
wie mit partieller Integration folgt.
\item \textit{Eigenwerte}. Sei $Af = \lambda f$, dann folgt
\begin{align*}
-f'' = \lambda f
\end{align*}
dies ist eine Differentialgleichung mit Lösung,
\begin{align*}
f(x) = c_1 \cos\left(\sqrt{\lambda}x\right) + c_2
\sin\left(\sqrt{\lambda}x\right)
\end{align*}
Die Randbedingungen erzwingen $c_1 = 0$ sowie
$\lambda = j^2$ für $j\in\N$. Normierte Eigenfunktionen sind daher,
\begin{align*}
e_j(x) = \sqrt{\frac{2}{\pi}}\sin(jx),\qquad j\in\N.
\end{align*}
\item Definiere $\tilde{A}$ durch (*) mit $D(A)\subseteq D(\tilde{A})$:
\begin{align*}
\tilde{A}f = Af,\quad \text{für } f\in D(A).
\end{align*}
$\tilde{A}$ heißt die \emph{Fortsetzung} bzw. \emph{Erweiterung} von $A$. Da
$\tilde{A}$ selbstadjungiert ist, spricht man von einer selbstadjungierten
Erweiterung.

Sei $f\in D(A)$, dann ist
\begin{align*}
\lin{f,e_j} = \frac{1}{\lambda_j} \lin{f,Ae_j} =
\frac{1}{\lambda_j}\lin{Af,e_j} = -\frac{1}{\lambda_j}\lin{f'',e_j}  
\end{align*}
Damit ist
\begin{align*}
\sum\limits_{j=1}^\infty \abs{\lambda_j \lin{f,e_j}} = \sum\limits_{j=1}^\infty
\abs{\lin{f'',e_j}} \le \norm{f''}^2 < \infty,
\end{align*}
und daher ist $f\in D(\tilde{A})$.
\begin{align*}
\tilde{A}f = \sum\limits_{j=1}^\infty \lambda_j \lin{f,e_j}e_j = -
\sum\limits_{j=1}^\infty \lin{f'',e_j} = -f''.\bsphere
\end{align*}
\end{enumerate}
\end{bsp}

\begin{bem}
\label{bem:1.19}
$A$ besitze die Darstellung (*), dann gilt $A$ ist positiv definit genau dann
wenn alle Eigenwerte positiv sind,
\begin{align*}
\lin{Ax,x} = \sum\limits_{j=1}^\infty \lambda_j
\lin{x,e_j}\overline{\lin{x,e_j}} \ge 0 \Leftrightarrow \lambda_j \ge 0.\maphere
\end{align*}
\end{bem}

\begin{defn}
\label{defn:1.20}
Ein linearer Operator $A$ in $(L,\lin{\cdot,\cdot})$ mit $D(A)=L$ heißt
\emph{beschränkt}, falls
\begin{align*}
\forall x\in L: \norm{Ax} \le c\norm{x}.
\end{align*}
Dann heißt,
\begin{align*}
\norm{A}&:= \inf\setdef{c> 0}{\forall x\in L: \norm{Ax}\le c\norm{x}}\\
&= \sup\limits_{\norm{x}\neq 0}\frac{\norm{Ax}}{\norm{x}} =
\sup\limits_{\norm{x}=1}\norm{Ax} = \sup\limits_{\norm{x}\le 1} \norm{Ax},
\end{align*}
die \emph{Operatornorm} bzw. Norm von $A$. Insbesondere gilt,
\begin{align*}
\forall x\in L: \norm{Ax}\le\norm{A}\norm{x}.\fishhere
\end{align*}
\end{defn}

\begin{lem}
\label{prop:1.21}
Besitzt $A$ die Darstellung (*), dann ist $A$ genau dann beschränkt, wenn
$(\lambda_j)$ beschränkt ist.\fishhere
\end{lem}
\begin{proof}
$\Rightarrow$: Sei $A$ beschränkt, dann gilt $\abs{\lambda_i} = \norm{Ae_i} \le
\norm{A}$.\\
$\Leftarrow$: Sei $\lambda_j\le M,\;\forall j\in\N$, so folgt mit der
Besselschen Ungleichung,
\begin{align*}
\norm{Ax}^2 &= \lin{Ax,Ax} =
\sum\limits_{j=1}^\infty\abs{\lambda_j}^2\abs{\lin{x,e_j}}^2
\le M^2 \sum\limits_{j=1}^\infty \abs{\lin{x,e_j}}^2 \\
&\le M^2\norm{x}^2.\qedhere
\end{align*}
\end{proof}

\begin{prop}
\label{prop:1.22}
Sei $A$ ein Operator mit $D(A) = L$, dann ist äquivalent.
\begin{enumerate}[label=(\roman{*})]
  \item\label{prop:1.22:1} $A$ ist beschränkt.
  \item\label{prop:1.22:2} $A$ ist stetig in $x=0$.
  \item\label{prop:1.22:3} $A$ ist überall stetig.\fishhere
\end{enumerate}
\end{prop}
\begin{proof}
\ref{prop:1.22:1}$\Rightarrow$\ref{prop:1.22:3}: Sei $(x_j)$ Folge in $L$ und
$x_j\to x$, dann ist
\begin{align*}
\norm{Ax_j-Ax} \le \norm{A}\norm{x_j-x},
\end{align*}
also ist $A$ sogar lipschitz.

\ref{prop:1.22:3}$\Rightarrow$\ref{prop:1.22:2}: Klar, denn Stetigkeit überall
impliziert Stetigkeit in $x=0$.

\ref{prop:1.22:2}$\Rightarrow$\ref{prop:1.22:1}: Sei $A$ stetig in $x=0$,
unbeschränkt und $(x_j)$ eine Folge mit $\norm{Ax_j}>j\norm{x_j}$. Setze $y_j :=
\frac{x_j}{\norm{Ax_j}}$, dann ist $\norm{y_j} < \frac{1}{j}$ und daher ist
$(y_j)$ eine Nullfolge. Aber $\norm{Ay_j} = 1$ und damit ist $(Ay_j)$ keine
Nullfolge und $A$ nicht stetig.\qedhere
\end{proof}

\begin{defn}
\label{defn:1.23}
Ein linearer Operator heißt \emph{kompakt}, falls $(Ax_j)$ eine konvergente
Teilfolge enthält, wenn $(x_j)$ beschränkt ist, oder äquivalent $AM$ präkompakt
ist (d.h. $\overline{AM}$ ist kompakt), wenn $M\subseteq L$ beschränkt
ist.\fishhere
\end{defn}

\begin{prop}
\label{prop:1.24}
Jeder kompakte Operator ist beschränkt.\fishhere
\end{prop}
\begin{proof}
Sei $A$ nicht beschränkt und $(x_j)$ Folge mit $\norm{Ax_j} > j\norm{x_j}$.
Setze $y_j = \frac{x_j}{\norm{x_j}}$, dann ist $\norm{y_j} = 1$, also ist
$(y_j)$ beschränkt. Es gilt jedoch
\begin{align*}
\norm{Ay_j} = \frac{\norm{Ax_j}}{\norm{x_j}} > j \to\infty,
\end{align*}
also divergiert $(Ay_{j_k})$ für jede Teilfolge von $(y_j)$ und  $A$ ist
nicht kompakt.\qedhere
\end{proof}

\begin{bsp}
\label{bsp:1.25}
\begin{enumerate}[label=\arabic{*}.)]
\item Seien $L=C([0,\pi]\to\C)$, $Af = -f''$ und $D(A) = $ wie in
\ref{bsp:1.18}.

Die Eigenwerte von $A$ sind $\lambda_j = j^2, j\in\N$, also ist $A$ nicht
beschränkt und daher nicht kompakt.

Wir werden später sehen, dass $A^{-1}$ existiert und kompakt ist.
\item Seien $K\subseteq\R^n$ kompakt, $G\in C(K\times K\to\R)$,
$L:=C(K\to\C)$, $D(A):=L$, sowie
\begin{align*}
Af(x) := \int_K G(x,y)f(y)\dy.
\end{align*}
$A$ ist ein Integraloperator, $G$ nennt man den Integralkern. Man kann zeigen,
dass $A$ kompakt ist.
\begin{proof}[Beweisskizze.]
Sei $(f_n)$ beschränkt, dann ist
\begin{align*}
\norm{f_n}_2 = \left(\int_K \abs{f_n}^2 \right)^{\frac{1}{2}} \le M.
\end{align*}
Wir sehen daher,
\begin{enumerate}[label=(\roman{*})]
  \item $(Af_j)$ ist gleichmäßig beschränkt auf $K$.
\begin{align*}
\norm{Af_j(x)}^2 &= \norm{\lin{G(x,\cdot),\overline{f_j}}}^2
\le \norm{G(x,\cdot)}^2\norm{\overline{f_j}}^2\\
&= \int_K \underbrace{\abs{G(x,y)}^2}_{\le c}\dy \norm{f_j}^2
\le cM^2\lambda^{(n)}(K).
\end{align*}
\item $(Af_j)$ ist gleichgradig stetig, d.h.
\begin{align*}
\forall\ep > 0 \exists \delta_\ep > 0 &: \forall x,y\in K, \forall j\in\N\\
&: \abs{x-y} <\delta_\ep \Rightarrow \abs{Af_j(x)-Af_j(y)}<\ep.
\end{align*}
\begin{align*}
\norm{Af_j(x)-Af_j(y)}^2 &= \norm{\lin{G(x,\cdot),f_j} -
\lin{G(y,\cdot),f_j}}^2\\ &= \abs{\lin{G(x,\cdot)-G(y,\cdot),f_j}}^2\\
&\le \norm{G(x,\cdot)-G(y,\cdot)}^2\norm{f_j}^2\\
&\le \tilde{\ep}^2M^2\lambda^{(n)}(K) < \ep, 
\end{align*}
für $\abs{x-y}<\tilde{\delta}$. Wir können nun den Satz von Arzelá-Ascoli
anwenden, der besagt, dass $(Af_j)$ eine Teilfolge $(Af_{j_k})$ enthält mit
$Af_{j_k}\unito g\in L$ auf $K$. Damit erhalten wir,
\begin{align*}
\norm{Af_{j_k}-g}^2 = \int_K
\underbrace{\abs{Af_{j_k}(x)-g(x)}^2}_{<\ep\text{ für } k>N_\ep}\dx \le
\ep^2\lambda^{(n)}(K),
\end{align*}
also $Af_{j_k}\to g$ in $L$.\bsphere
\end{enumerate}
\end{proof}
\end{enumerate}
\end{bsp}

\begin{prop}
\label{prop:1.26}
Sei $A$ ein Operator im Hilbertraum $H$. Dann sind äquivalent,
\begin{enumerate}[label=(\roman{*})]
  \item $A$ ist kompakt und symmetrisch.
  \item $A$ besitzt die Darstellung (*) mit $(\lambda_j)\to 0$ und $\lambda_j$
  reell.\fishhere
\end{enumerate}
\end{prop}
\begin{proof}
``$\Leftarrow$'': Übung.\\ 
``$\Rightarrow$'': Nächstes Kapitel.\qedhere
\end{proof}

\subsection{Kompakte Symmetrische Operatoren}

\begin{defn}
\label{defn:1.27}
Sei $M\subseteq L$. Dann heißt
\begin{align*}
M^\bot := \setdef{x\in L}{\forall y\in M: \lin{x,y}=0},
\end{align*}
das \emph{orthogonale Komplement}.\fishhere
\end{defn}

\begin{prop}
\label{prop:1.28}
Sei $(L,\lin{\cdot,\cdot})$ ein Prähilbertraum, $A: L\to L$ symmetrisch und
beschränkt. Dann gilt
\begin{enumerate}[label=(\roman{*})]
  \item\label{prop:1.28:1} $\forall x\in L : \lin{Ax,x}\in\R$.
  \item\label{prop:1.28:2} $(\im A)^\bot = \ker A$.
  \item\label{prop:1.28:3} $A\bigg|_{\overline{\im A}}$ ist injektiv.
  \item\label{prop:1.28:4} $\norm{A} = \sup\limits_{\norm{x} = 1}
  \abs{\lin{Ax,x}}$.
\end{enumerate}
\end{prop}
\begin{proof}
\begin{enumerate}[label=(\roman{*})]
  \item $\lin{Ax,x} = \lin{x,Ax} = \overline{\lin{Ax,x}}$.
  \item Sei $x\in \im A^\bot$, dann gilt für alle $y\in L$:
\begin{align*}
0 = \lin{Ay,x} = \lin{y,Ax} \Rightarrow \lin{Ax,Ax} = 0 \Rightarrow x\in \ker A.
\end{align*}
Sei $x\in \ker A$, dann gilt für alle $y\in L$:
\begin{align*}
\lin{Ay,x} =\lin{y,Ax} = 0 \Rightarrow x\in\im A^\bot.
\end{align*}
\item Seien $x,y\in \overline{\im A}$, dann ist auch $x-y\in\overline{\im A}$.
Sei $Ax = Ay$, dann gilt
\begin{align*}
A(x-y) = 0&\Rightarrow x-y\in \ker A = \im A^\bot\\
&\Rightarrow \lin{x-y,z} = 0, \forall z\in \im A\\
&\Rightarrow \lin{x-y,z} = 0,\forall z\in\overline{\im A}\\
&\Rightarrow x-y = 0.
\end{align*}
\item Sei $A\neq 0$, also $\norm{A} > 0$, 
$d:=\sup\limits_{\norm{x}=1}\abs{\lin{Ax,x}}$. Zeige $d=\norm{A}$.

``$\le$'': Für $\norm{x} = 1$ gilt,
$\abs{\lin{Ax,x}} \le \norm{Ax}\norm{x}\le
\norm{A}\norm{x}^2 = \norm{A} \Rightarrow d\le\norm{A}$.

``$\ge$'': Für $x\neq 0$ gilt,
\begin{align*}
\abs{\lin{Ax,x}} = \norm{x}^2\abs{\lin{A\frac{x}{\norm{x}},\frac{x}{\norm{x}}}}
\le d\norm{x}^2.
\end{align*}
Die Ungleichung gilt offensichtlich für $x\in L$.

Sei nun $\alpha > 0$, dann gilt
\begin{align*}
&\lin{A(\alpha x + \frac{1}{\alpha}Ax),\alpha x + \frac{1}{\alpha}Ax}
- \lin{A(\alpha x - \frac{1}{\alpha}Ax),\alpha x - \frac{1}{\alpha}Ax}
\\ &= 2\lin{Ax,Ax} + 2\lin{Ax,Ax} = 4\norm{Ax}^2
\end{align*}
Somit gilt,
\begin{align*}
4\norm{Ax}^2 &\le d\norm{\alpha x+ \frac{1}{\alpha}Ax}^2 +
d\norm{\alpha x-\frac{1}{\alpha}Ax}\\
&= 2d\left(\alpha^2\norm{x}^2 + \frac{1}{\alpha^2}\norm{Ax}^2\right)
\end{align*}
Sei nun $\norm{x} = 1$ und $\alpha^2 := \norm{Ax} > 0$, dann gilt
\begin{align*}
&4\norm{Ax}^2 \le 2d\left(\norm{Ax} + \norm{Ax}\right) = 4d\norm{Ax}.\\
\Rightarrow\;&\norm{Ax} \le d,\\
\Rightarrow\;&\norm{A} = \sup\limits_{\norm{x}=1}\norm{Ax} \le d.\qedhere
\end{align*}
\end{enumerate}
\end{proof}

\begin{bem}
\label{bem:1.29}
Ist zusätzlich zu \ref{prop:1.28} $(L,\lin{\cdot,\cdot})$ ein Hilbertraum, so
kann \ref{prop:1.28:2} in der Form,
\begin{align*}
L = \overline{\im A}\oplus \ker A,
\end{align*}
geschrieben werden. Dies bedeutet,
\begin{align*}
\forall x\in L \exists! y\in \overline{\im A} \exists! z\in \ker A : x =
y+z.\maphere
\end{align*}
\end{bem}

\begin{prop}[Hauptsatz]
\label{prop:1.30}
Sei $(L,\lin{\cdot,\cdot})$ ein unendlichdimensionaler Prähilbertraum. $A: L\to
L$ symmetrisch und kompakt. Dann gilt
\begin{enumerate}[label=(\roman{*})]
  \item $\lambda = \norm{A}$ oder $\lambda = -\norm{A}$ ist ein Eigenwert.
  \item Jeder Eigenwert $\lambda\neq 0$ ist reell und hat endliche Vielfachheit.
  \item $A$ hat entweder nur endlich viele Eigenwerte, dann ist $\lambda=0$
  Eigenwert mit Vielfachheit $\infty$ oder $A$ hat abzählbar viele Eigenwerte
  $(\lambda_j)$ und $\lambda_j \to 0$.
  \item Sei $(\lambda_j)$ die Folge der Eigenwerte $\lambda_j\neq 0$, in der
  jeder Eigenwert so oft auftritt, wie es seiner Vielfachheit entspricht. Die
  zugehörigen Eigenelemente $(e_j)$ können so normiert werden, dass sie ein 
  ONS bilden. Dieses ONS ist vollständig in $\overline{\im
  A}$.\fishhere
\end{enumerate}
\end{prop}
\begin{proof}
Falls $A=0$, dann ist $\lambda = 0$ einziger Eigenwert mit Vielfachheit
$\infty$, die übrigen Aussagen ergeben sich dann sofort.

Sei nun $A\neq 0$, also $\norm{A} > 0$.
\begin{enumerate}[label=(\roman{*})]
  \item Nach \ref{prop:1.28} existiert eine Folge $(y_j)$ in $L$ mit
\begin{align*}
\norm{y_j} = 1 \land \abs{\lin{Ay_j,y_j}}\to\norm{A}.
\end{align*}
Eventuell Teilfolge bilden, dann erhalten wir
\begin{align*}
\abs{\lin{Ay_j,y_j}}\to\lambda = \pm \norm{A}\neq 0.
\end{align*}
$A$ ist kompakt und $\norm{y_j} = 1$ also $Ay_j \to y\in L$ und damit gilt,
\begin{align*}
0&\le \norm{Ay_j - \lambda y_j}^2 
= \norm{Ay_j}^2  -2\lambda \lin{A y_j,y_j} + \lambda^2\norm{y_j}^2
\\ &\le \underbrace{\norm{A}^2}_{=\lambda^2}\norm{y_j}^2 + \lambda^2 -
2\lambda\underbrace{\lin{Ay_j,y_j}}_{\to\lambda}
\to 0.
\end{align*}
$\lambda y_j \to y$ also $y_j\to\frac{1}{\lambda}y$.
\begin{align*}
\Rightarrow
\begin{cases}
&Ay_j\to y,\\
&Ay_j\to\frac{1}{\lambda}Ay,
\end{cases}
\end{align*}
da $A$ stetig, also ist $Ay = \lambda y$ und $y\neq 0$, da
$1=\norm{y_j}\to\norm{\frac{1}{\lambda}y}$ und daher ist
\begin{align*}
\lambda =\pm \norm{A}
\end{align*}
ein Eigenwert von $A$.
\item Sei $\lambda$ ein Eigenwert, dann ist $\lambda\in\R$.\\
Angenommen $\lambda\neq 0$ mit Vielfachheit $\infty$. Sei $(x_j)$ eine Folge
von linear unabhängigen Eigenelementen. Mit Gram-Schmidt erhalten wir dadurch
eine orthonormale Folge $(e_j)$ mit $Ae_j = \lambda e_j$. Nun ist
$\norm{e_j}=1$ also enthält $(Ae_j)$ eine konvergente Teilfolge aber
\begin{align*}
\norm{Ae_j - Ae_k}^2 = \abs{\lambda}^2\norm{e_j-e_k}^2 = 2\abs{\lambda}^2 =
\const > 0.\dipper
\end{align*}
\item Sei $Ax_1 = \lambda_1x_1$, $\lambda_1 = \pm\norm{A}\neq 0$. Setze
\begin{align*}
&A_1 := \left.A\right|_{\left\{x_1\right\}^\bot} : \{x_1\}^\bot \to
\{x_1\}^\bot,\\
&y\bot x_1 \Rightarrow \lin{Ay,x_1} = \lin{y,Ax_1} =
\lambda\lin{y,x_1} = 0\Rightarrow Ay\bot x_1.
\end{align*}
Also ist $A_1$ kompakt und symmetrisch und hat den Eigenwert $\lambda_2$ mit
$\abs{\lambda_2} = \norm{A_1}\le\norm{A}$, also hat $A$ ebenfalls den Eigenwert
$\lambda_2$.

Führe das Verfahren mit,
\begin{align*}
A_2 := A|_{\{x_1,x_2\}^\bot}
\end{align*}
fort. Induktiv erhält man eine Folge von Eigenwerten
$(\lambda_j)$ mit $(\abs{\lambda_j})$ monoton fallend, wobei jeder Eigenwert
nur endliche Vielfachheit hat.
\begin{enumerate}[label=Fall \alph{*})]
  \item $\exists j\in\N: A_j = 0$, dann ist $\lambda_{j+1}=0$ mit Eigenraum
  $\{x_1,\ldots,x_j\}^\bot$ also Vielfachheit $\infty$.
  \item $\forall j\in\N: A_j\neq 0$ und daher ist $\lambda_{j+1}\neq 0$.
  Eigenwerte zu verschiedenen Eigenwerten sind orthogonal. Führt man nun in
  jedem Eigenraum Gram-Schmidt durch, erhält man ein ONS $(e_j)$ aus
  Eigenfunktionen mit $Ae_j = \lambda_j e_j$ und $(\abs{\lambda_j})$ monton
  fallend.
  
  Angenommen $\neg(\lambda_j \to 0)$ dann existiert ein $\ep > 0$ mit
  $\abs{\lambda_j}\ge \ep, \forall j\in\N$, dann ist $\left(\frac{1}{\lambda_j}
  e_j\right)$ eine beschränkte Folge aber $\left(A\frac{1}{\lambda_j}e_j\right)
  = (e_j)$ enthält keine konvergente Teilfolge.\dipper
\end{enumerate}
Es gibt keine weiteren Eigenwerte als die soeben konstruierten, denn sei
$\lambda$ ein weiterer Eigenwert, dann gilt
\begin{align*}
&Ax = \lambda x,\qquad x\neq 0\\
&\forall j\in\N : \lin{x,e_j}=0.
\end{align*}
Dann gilt
\begin{align*}
\norm{Ax} = \norm{A_{\{e_1,\ldots,e_j\}^\bot}x}
\le  \underbrace{\norm{A_{\{e_1,\ldots,e_j\}^\bot}}}_{\to 0}\norm{x},
\end{align*}
also ist $\norm{Ax} = 0$, d.h. $Ax = 0$ und damit ist $\lambda=0$.
\item Sei $(\lambda_j)$ vorgegeben und $x_j$ Eigenelement zu $\lambda_j$.
Eigenelemente zu verschiedenen Eigenwerten sind bereits orthogonal, wendet man
außerdem Gram-Schmidt in jedem Eigenraum an und normiert alle Eigenelemente,
erhält man ein ONS $(e_j)$ aus Eigenelementen.

Sei $x\in L$, $x_J :=\sum\limits_{j=1}^J \lin{x,e_j}e_j$,
\begin{align*}
\norm{x-x_J}^2 = \underbrace{\norm{x}^2 - \norm{x_J}^2}_{\text{Pythagoras}} \le
\norm{x}^2.
\end{align*}
Nun ist $x-x_J\in\{e_1,\ldots,e_J\}^\bot$ und
\begin{align*}
\norm{Ax - Ax_J} = \norm{A(x-x_J)}\le 
\underbrace{\norm{A|_{\{e_1,\ldots,e_J\}}^\bot}}_{\to
0}\underbrace{\norm{x-x_j}}_{\le\norm{x}}
\end{align*}
Daraus folgt $\forall x\in L : Ax_J\to Ax$ mit,
\begin{align*}
Ax_j = \sum\limits_{j=1}^\infty \lin{x,e_j}\lambda_je_j
= \sum\limits_{j=1}^\infty \lin{x,Ae_j}e_j
= \sum\limits_{j=1}^\infty \lin{Ax,e_j}e_j.
\end{align*} 
Daher gilt für alle $x\in L$,
\begin{align*}
&Ax = \sum\limits_{j=1}^\infty \lin{Ax,e_j}e_j,\\
\overset{\ref{prop:1.10}}{\Rightarrow}\;&
\forall y\in \overline{\im A} : y = \sum\limits_{j=1}^\infty
\lin{y,e_j}e_j.\qedhere
\end{align*}
\end{enumerate}
\end{proof}

\subsection{Das Sturm-Liouville-Problem}
\begin{bsp}
\label{bsp:1.31}
Eine schwingende Saite kann durch die Differentialgleichung 
\begin{align*}
\rho(x)\partial_t^2 u(t,x) - \sigma(x)\partial_x^2u(t,x) = 0,
\end{align*}
charakterisiert werden, wobei $\rho(x)>0$ die \emph{Massendichte} und
$\sigma(x)> 0$ die \emph{Federkonstante} bezeichnen.
\begin{figure}[!b]
  \centering
\begin{pspicture}(-1,-1)(4.5,2.5)
 \psaxes[labels=none,ticks=none]{->}%
 (0,0)(-0.5,-0.5)(4,2)[\color{gdarkgray}$x$,-90][\color{gdarkgray}$u$,0]
 
 \psxTick(3.5){\color{gdarkgray}l}
 
 \psline{<->}(1.75,0.5)(1.75,-0.5)

 \psplot[linewidth=1.2pt,%
	     linecolor=darkblue,%
	     algebraic=true]%
	     {0}{3.5}{sin(3.132/3.5*x)}
 
 \rput[l](3,1){\color{gdarkgray}$u(t,x)$}
 \rput[l](-0.4,-0.4){\color{gdarkgray}$0$}
\end{pspicture}
  \caption{Schwingende Saite.}
\end{figure}
Für die eingespannte Saite sind die Randbedingungen,
\begin{align*}
u(t,0) = u(t,l) = 0,\qquad t\ge 0.
\end{align*}
Geben wir Anfangsbedingungen in der Form,
\begin{align*}
&u(0,x) = u_0(x), &&\text{Auslenkung zur Zeit } t=0,\\
&\partial_t u(t,x)\big|_{t=0} = u_1(x), &&\text{Anfangsgeschwindigkeit zur Zeit
} t=0,
\end{align*}
an, so besagt der Satz von Picard-Lindelöff, dass eine eindeutige
Lösung existiert. Wir sind nun daran interessiert diese Lösung zu finden.
Setzten wir
\begin{align*}
c(x) = \sqrt{\frac{\sigma(x)}{\rho(x)}},
\end{align*}
nimmt die Differentialgleichung die aus der Physik bekannte Form der
Wellengleichung an,
\begin{align*}
\partial_t^2 u(t,x) - c^2\partial_x^2u(t,x) = 0\tag{*}.
\end{align*}
Wir wollen nun beim Lösen ``naiv'' vorgehen und beginnen mit dem
Separationsansatz
\begin{align*}
u(t,x) = v(t)w(x).
\end{align*}
So erhalten wir
\begin{align*}
v''w - c^2 vw'' = 0,
\end{align*}
wordurch sich für  $v,w\neq 0$ ergibt,
\begin{align*}
\frac{v''}{v} = c^2\frac{w''}{w} = \const := \lambda.
\end{align*}
Die Konstanz sieht man so ein, dass die linke Seite lediglich von $t$ und die
rechte lediglich von $x$ abhängt. Wir erhalten dadurch zwei unabhängige
Differentialgleichungen
\begin{align*}
&v''+\lambda v = 0, && \text{mit } v(0),v'(0) \text{ vorgegeben},\\
&c^2w'' + \lambda w = 0, && \text{mit } w(0)=w(l)=0.
\end{align*}
Für den Spezialfall $c(x)=c=\const$ existieren Lösungen von $w$ nur für
\begin{align*}
&\lambda=\lambda_j = \left(\frac{j\pi c}{l}\right)^2,\\
\Rightarrow\;&w_j(x) = \sqrt{\frac{2}{l}}\sin\left(\sqrt{\lambda_j}x\right),
\end{align*}
und Lösungen von $v$, für
\begin{align*}
v_j(t) = v_j(0)\cos\left(\sqrt{\lambda_j}t\right) +
v_j'(0)\frac{\sin\left(\sqrt{\lambda_j}t \right)}{\sqrt{\lambda_j}}.
\end{align*}
Wir erhalten somit abzählbar viele Lösungen von (*),
\begin{align*}
u_j(t,x) = v_j(t)w_j(x).
\end{align*}
Die allgemeine Lösung von (*) kann man durch eine Fourierreihe angeben,
\begin{align*}
&u(t,x) = \sum\limits_{j=1}^\infty w_j(x)v_j(x),\\
&u(0,x) = \sum\limits_{j=1}^\infty w_j(0)v_j(0) \overset{!}{=} u_0 \sim
v_j(0),\\
&\partial_t u(0,x) = \sum\limits_{j=1}^\infty w_j(x)v_j'(0) \overset{!}{=} u_1
\sim v_j'(0).\bsphere
\end{align*}
\end{bsp}

Im Folgenden sei mit $I$ stets das Intervall $[a,b]\subseteq\R$ gemeint. 

\begin{prop}[Sturm-Liouville-Eigenwertproblem]
\label{prop:1.32}
Gesucht sind $u\in C^2([a,b]\to\R)$, $u\neq 0$ und $\lambda\in\R$ mit,
\begin{align*}
&(pu')' - qu + \lambda ru = 0,\tag{1.1}\\
&R_1u := \alpha_1u(a) + \alpha_2p(a)u'(a) = 0,\\
&R_2u := \beta_1u(b) + \beta_2p(b)u'(b) = 0,
\end{align*}
wobei
\begin{align*}
&p\in C^2([a,b]\to\R),\; p>0\text{ auf }[a,b],\\
&q,r\in C^2([a,b]\to\R),\; r > 0,\\
&\alpha_1^2 + \alpha_2^2 > 0 \land \beta_1^2+\beta_2^2 > 0,
\end{align*}
$\lambda$ heißt \emph{Eigenwert} und $u$ zugehörige
\emph{Eigenfunktion}.\fishhere
\end{prop}

\begin{bem}
\label{bem:1.33}
\begin{enumerate}[label=\arabic{*}.)]
  \item Spezialfälle mit besonderen Randbedindungen,
\begin{align*}
&\alpha_1 = \beta_1 = 1 \land \alpha_2 = \beta_2 = 0,&& \text{(Dirichlet)}\\
&u(a) = u(b) = 0,\\
&\alpha_1 = \beta_1 = 0 \land \alpha_2 = \frac{1}{p(a)} \land \beta_2 =
\frac{1}{p(b)},&&\text{(Neumann)}\\
&u'(a) = u'(b) = 0.
\end{align*}
\item $R_1u = 0$ was äquivalent ist mit,
\begin{align*}
\begin{pmatrix}
u(a)\\
u'(a)
\end{pmatrix}
\bot
\begin{pmatrix}
\alpha_1\\
\alpha_2 p(a)
\end{pmatrix}
\end{align*}
\item Jeder Eigenwert von (1.1) ist einfach.
\begin{proof}
Angenommen $u_1,u_2$ sind Eigenfunktionen von (1.1) zum Eigenwert $\lambda$ und
$\{u_1,u_2\}$ ist linear unabhängig, dann ist $\{u_1,u_2\}$ ein
Fundamentalsystem. Die Existenztheorie für Differentialgleichungen sagt, es
existiert eine Lösung mit $u(a) = \alpha_1$ und $u'(a) = \frac{\alpha_2}{p(a)}$.

Da $\{u_1,u_2\}$ ein Fundamentalsystem ist, folgt
\begin{align*}
u = c_1 u_1 + c_2 u_2
\end{align*}
ist ebenfalls Lösung, die die Randbedingungen erfüllt mit
\begin{align*}
R_1u &= \alpha_1u(a) + \alpha_2p(a)u'(a) = \alpha_1^2+\alpha_2^2\neq 0,\\
R_1 u &= c_1R_1u_1 + c_2R_1u_2 \overset{!}{=} 0.\dipper
\end{align*}
Es kann also keine linear unabhängigen Eigenfunktionen zu einem Eigenwert
geben.\qedhere\maphere
\end{proof}
\end{enumerate}
\end{bem}

\begin{prop}[Definition/Satz]
\label{prop:1.34}
Sei $I:=[a,b]$ und $\tilde{A}$ mit
\begin{align*}
\tilde{A}u = -(pu')' + qu,
\end{align*}
eine Abbildngsvorschrift. Sei $L:=C(I\to\R)$ und $\lin{\cdot,\cdot}$ das
Skalarprodukt auf $\LL^2$,
\begin{align*}
&D(A) = \setdef{u\in C^2(I\to\R)}{R_1u=R_2u=0},\tag{1.2}\\
&Au = \tilde{A}u,\qquad \text{für }u\in D(A).
\end{align*}
Dann ist $A$ symmetrisch.\fishhere
\end{prop}
\begin{proof}[Beweisskizze.]
Partiell integrieren, geschickt Null ergänzen, zusammenfassen.\qedhere
\end{proof}

\begin{bem}[Achtung:]
\label{bem:1.35}
Falls $\lambda$ EW von (1.1) und $u$ Eigenfunktion, dann gilt $u\in D(A)$ und
$Au = \lambda r u$. Nur im Fall $r=1$ ist $\lambda$ auch Eigenwert von $A$.
\end{bem}

\begin{defn}
\label{defn:1.36}
Eine Funktion $G: I\times I\to\R$ heißt \emph{Greensche Funktion} zum Operator
$A$ (oder zu (1.1)), falls
\begin{enumerate}[label=\arabic{*}.)]
  \item $G\in C(I\times I\to\R)$
  \item $\partial_1 G, \partial_1^2 G\in C(\Delta_j\to\R)$ und $\partial_1 G,
  \partial_1^2 G$ stetig fortsetzbar auf $\overline{\Delta}_1$ und
  $\overline{\Delta}_2$
  \item Für festes $\zeta\in I$ gilt,
  \begin{align*}
  \tilde{A}G(\cdot,\zeta) = 0,\text{ auf }
  I\setminus\{\zeta\},\text{ sowie } R_1(G(\cdot,\zeta)) = R_2(G(\cdot,\zeta)) =
  0.
  \end{align*}
  \item Für $\zeta\in(a,b) : \partial_1 G(\zeta+0,\zeta) - \partial_1
  G(\zeta-0,\zeta) = \frac{1}{p(\zeta)}$.\fishhere
\end{enumerate}
\end{defn}
\begin{figure}[!ht]
  \centering
\begin{pspicture}(-0.8,-0.8)(4.8,3)

\psaxes[labels=none,ticks=none]{->}%
 (0,0)(-0.5,-0,5)(3,3)[,-90][,0]

\psxTick(0.5){a}
\psxTick(2.5){b}

\psyTick(0.5){a}
\psyTick(2.5){b}

\psdot(0,0)

\psline(0.5,0.5)(2.5,0.5)(2.5,2.5)(0.5,2.5)(0.5,0.5)

\psline[linestyle=dotted](0.5,0.5)(2.5,2.5)

\psline[linecolor=darkblue]{->}(2.4,1.8)(2,1.8)

\psline[linecolor=darkblue]{->}(1.4,0.8)(1.4,1.2)

\psline[linecolor=darkblue]{->}(1.7,2.3)(2.1,2.3)

\rput[l](2.7,1.8){\color{darkblue}$\partial_1 G(\zeta+0,\zeta)$}
\rput[l](1.2,0.25){\color{darkblue}$\partial_1 G(\zeta,\zeta-0)$}
\rput[l](0.6,2.8){\color{darkblue}$\partial_1 G(\zeta-0,\zeta)$}

\rput(0.8,2.2){\color{yellow}$\Delta_1$}
\rput(2.2,0.8){\color{yellow}$\Delta_2$}

\end{pspicture} 
  \caption{Definitionsbereich der Greenschen Funktion.}
\end{figure}

Die Greensche Funktion soll also auf die Dreiecke $\overline{\Delta}_1$ und
$\overline{\Delta}_2$ zweimal stetig differenzierbar fortsetzbar sein, die
homogene Differentialgleichung erfüllen und ihre erste partielle Ableitung soll
auf der Diagonalen einen Sprung haben.

\begin{prop}
\label{prop:1.37}
Sei $G$ eine Greensche Funktion zu $A$ und $\ph\in C(I\to\R)$. Dann sind
äquivalent,
\begin{enumerate}[label=(\roman{*})]
  \item\label{prop:1.37:1} $Au = \ph$,
  \item\label{prop:1.37:2} $u(x) = \int\limits_a^b G(x,y)\ph(y)\dy$.\fishhere
\end{enumerate}
\end{prop}
Existiert $G$, so erhalten wir einen Umkehroperator zu $A$ und so für jedes
$\ph$ mit $Au=\ph$ sofort $u$.
\begin{proof}
``\ref{prop:1.37:2}$\Rightarrow$\ref{prop:1.37:1}'':
\begin{align*}
u(x) = \int\limits_a^x G(x,y)\ph(y)\dy + \int\limits_x^b G(x,y)\ph(y)\dy.
\end{align*}
\begin{align*}
u'(x) &= G(x,y)\ph(x) - G(x,y)\ph(x) \\
&+\int\limits_a^x \frac{\partial}{\partial x}G(x,y)\ph(y)\dy +
\int\limits_x^b \frac{\partial}{\partial x}G(x,y)\ph(y)\dy\\
&= \int\limits_a^x \frac{\partial}{\partial x}G(x,y)\ph(y)\dy +
\int\limits_x^b \frac{\partial}{\partial x}G(x,y)\ph(y)\dy
\end{align*}
\begin{align*}
u''(x) &=
\frac{\partial}{\partial x} G(x,x-0)\ph(x-0) + 
\int\limits_a^x \frac{\partial^2}{\partial x^2} G(x,y)\ph(y)\dy \\
 &-\frac{\partial}{\partial x} G(x,x+0)\ph(x+0) + 
\int\limits_x^b \frac{\partial^2}{\partial x^2} G(x,y)\ph(y)\dy\\
&= -\frac{\ph(x)}{p(x)} + 
\int\limits_a^x \frac{\partial^2}{\partial x^2} G(x,y)\ph(y)\dy +
\int\limits_x^b \frac{\partial^2}{\partial x^2} G(x,y)\ph(y)\dy\\
&= -\frac{\ph(x)}{p(x)} + 
\int\limits_a^b \frac{\partial^2}{\partial x^2} G(x,y)\ph(y)\dy
\end{align*}
Wir können damit $\tilde{A}u$ wie folgt auflösen,
\begin{align*}
\tilde{A}u &= -pu'' - p'u' + qu
= p\frac{\ph(x)}{p} + \int\limits_a^b
\underbrace{\tilde{A}G(x,y)\ph(y)}_{3.) \Rightarrow = 0}\dy = \ph(x).
\end{align*}
Offensichtlich ist $u\in D(A)$, denn $u\in C^2(I\to\R)$ und
\begin{align*}
R_1u &= \alpha_1 u(a) + \alpha_2p(a)u'(a) \\
&=\int\limits_a^b \underbrace{\left(\alpha_1G(a,y) + \alpha_2p(a) \partial_1
G(a,y) \right)\ph(y)\dy}_{=R_1G(\cdot,y)=0} = 0.
\end{align*}
``\ref{prop:1.37:1}$\Rightarrow$\ref{prop:1.37:2}'': Die Umkehrung zeigen wir
später.\qedhere
\end{proof}

\begin{prop}
\label{prop:1.38}
\begin{enumerate}[label=(\roman{*})]
  \item $G$ ist symmetrisch, d.h. $G(x,y)=G(y,x)$.
  \item $G$ ist eindeutig, d.h. zu gegebenen Operator $A$ existiert höchstens
  eine Eigenfunktion.\fishhere
\end{enumerate}
\end{prop}
\begin{proof}
Seien $G,H$ Greensche Funktionen zu $A$ und $\ph,\psi\in C(I\to\R)$.
\begin{align*}
u(x) &:= \int\limits_a^b G(x,y)\ph(y)\dy,\\
v(x) &:= \int\limits_a^b H(x,y)\psi(y)\dy.
\end{align*}
Mit \ref{prop:1.37} folgt,
\begin{align*}
Au = \ph,\text{ und } Av = \psi.
\end{align*}
\begin{align*}
0 &= \lin{Au,v} - \lin{Au,v} = \lin{\ph,v} - \lin{u,\psi}
\\ &= \int\limits_a^b \ph(x)\int\limits_a^b H(x,y)\psi(y)\dy \dx
- \int\limits_a^b \int\limits_a^b G(x,y)\ph(y)\dy \psi(x)\dx.
\end{align*}
Wir können im ersten Term die Integrationsreihenfolge ändern und im zweiten die
Variablen $x$ und $y$ vertauschen,
\begin{align*}
\ldots &= \int\limits_a^b\int\limits_a^b \ph(x)H(x,y)\psi(y) -
G(y,x)\ph(x)\psi(y)\dx\dy
\\ &= \int\limits_a^b \ph(x)\underbrace{\int\limits_a^b
\underbrace{\left(H(x,y)-G(y,x)\right)}_{(*)}\psi(y)\dy}_{(**)}\dx.
\end{align*}
Da $\ph\in C(I\to\R)$ beliebig ist, kann man 
$\ph=(**)$ setzten, dann verschwindet das Betragsquadrat, also ist (**) Null.
$\psi\in C(I\to\R)$ ist ebenfalls beliebig und daher ist ebenfalls $(H-G)$
Null, es gilt also $H(x,y) = G(y,x)$.
\begin{enumerate}[label=(\roman{*})]
  \item Setze $H(x,y)=G(x,y)$, dann gilt $G(x,y)=G(y,x)$.
  \item $G$ ist symmetrisch es gilt daher $H(x,y) = G(y,x) = G(x,y)$.\qedhere
\end{enumerate}
\end{proof}

\begin{proof}[Beweis von \ref{prop:1.37}.]
``\ref{prop:1.37:1}$\Rightarrow$\ref{prop:1.37:2}'': Sei $Au = \ph$.
\begin{align*}
\int\limits_a^b G(x,y)\ph(y)\dy &= \int\limits_a^b G(y,x)((-pu')'+qu)(y)\dy\\
&= \int\limits_a^x G(y,x)((-pu')'+qu)(y)\dy \\
&+ \int\limits_x^b G(y,x)((-pu')'+qu)(y)\dy \\
&\overset{\text{2x part.}}{=}
\left[-G(y,x)p(y)u(y) + \partial_1 G(y,x)
p(y)u(y)\right]\bigg|_{y=a}^x\bigg|_{y=x}^b\\ &+ \int\limits_a^b
\underbrace{\left(\frac{\partial}{\partial y}(p(y)\partial_1 G(y,x)) - q(y)G(y,x) \right)}_{:=\tilde{A}G(\cdot,x)=0}u(y)\dy.
\end{align*}
Wir wissen aus dem Beweis der Symmetrie von $G(x,y)$, dass der Term aufgrund
der Randbedingung auf dem Rand verschwindet,
\begin{align*}
&\underbrace{\left[\ldots\right]_{y=a}^b}_{=0,\text{ wegen
}R_ju=R_jG(\cdot,x)=0} +
\underbrace{(-G(y,x)p(y))u(y)\bigg|_{y=x}^x}_{=0,\text{ da stetig}.}\\
&+ \partial_1G(x-0,x)p(x)u(x) - \partial_1 G(x+0,x)p(x)u(x) \\
&= -\left(-\frac{1}{p}\right)p u(x) = u(x).\qedhere
\end{align*}
\end{proof}

\begin{prop}[Konstruktion der Greenschen Funktion]
\label{prop:1.39}
Folgende Aussagen sind äquivalent,
\begin{enumerate}[label=(\roman{*})]
  \item\label{prop:1.39:1} Zu $A$ existiert eine Greensche Funktion.
  \item\label{prop:1.39:2} $\lambda = 0$ ist kein Eigenwert von $A$
  $\Leftrightarrow$ $\lambda=0$ ist kein Eigenwert von (1.1).\fishhere
\end{enumerate}
\end{prop}
\begin{proof}
``\ref{prop:1.39:1}$\Rightarrow$\ref{prop:1.39:2}'':
Aus $Au=0$ folgt, $u=\int G\cdot 0\dy = 0$. $u=0$ ist die eindeutige Lösung
dieser Gleichung aber kein Eigenvektor und daher ist $\lambda=0$ kein Eigenwert.

``\ref{prop:1.39:2}$\Rightarrow$\ref{prop:1.39:1}'':
Sei $\lambda=0$ kein Eigenwert von $A$ und $u_1,u_2\in C^2(I\to\R)$ Lösungen von
$\tilde{A}u=0$ mit $u_1,u_2\neq 0$ und $R_1u_1 = 0$ und $R_2u_2 = 0$. Die
Existenz folgt aus dem Satz von Picard-Lindelöff.

Angenommen $u_1 = c u_2$, dann gilt $R_1u_1 = 0$ nach Voraussetzung und
\begin{align*}
&R_2u_1 = cR_2 u_2 = 0,\\
\Rightarrow\;& Au_1 = 0,
\Rightarrow\; u_1 = 0.\dipper
\end{align*}

Die \emph{Wronskideterminante} ist gegeben durch
\begin{align*}
W(x) = \begin{vmatrix}
u_1(x) & u_2(x)\\
u_1'(x) & u_2'(x)
\end{vmatrix}.
\end{align*}
Nun ist
\begin{align*}
W'(x) &= \frac{\partial}{\partial x}\left(u_1(x)u_2'(x) - u_2(x)u_1'(x) \right)
= u_1u_2'' - u_2u_1''\\
&= u_1\frac{1}{p}\left(-p'u_2' + qu_2 -\lambda ru_2\right) -
u_2\frac{1}{p}\left(-p'u_1'+qu_1 -\lambda ru_1\right)\\
&= -\frac{1}{p}\left(u_1u_2' p' - u_1'u_2 p'\right)
= -\frac{p'}{p}W(x),\\
W'(x) &= -\frac{p'}{p}W(x).
\end{align*}
Dies ist eine spearierbare Differentialgleichung. Durch Nachrechnen erhält man
$W(x) = \frac{c}{p(x)}$. $\{u_1,u_2\}$ ist linear unabhängig, also ist
$W(x)\neq 0$ und daher $c\neq 0$. Setze
\begin{align*}
G(x,\xi) =
\begin{cases}
-\frac{1}{c}u_2(\xi)u_1(x), & a \le x \le \xi \le b,\\
-\frac{1}{c}u_2(x)u_1(\xi), & a \le \xi \le x \le b.
\end{cases}
\end{align*}
Dann ist $G$ eine Greensche Funktion, denn
\begin{enumerate}[label=\arabic{*}.)]
  \item $G\in C(I\times I\to\R)$ ist offensichtlich.
  \item $\partial_1 G,\partial_1^2 G$ ist stetig auf $\Delta_1$, $\Delta_2$ und
  stetig fortsetzbar auf $\overline{\Delta}_1$, $\overline{\Delta}_2$.
  \item Sei $\xi\in I$ fest, dann gilt
\begin{align*}
&x < \xi \Rightarrow G(\cdot,\xi) = -\frac{1}{c}u_2(\xi)u_1 \Rightarrow
\tilde{A}G(\cdot,\xi) = 0,\; R_1G(\cdot,\xi) = 0,\\
&x > \xi \Rightarrow G(\cdot,\xi) = -\frac{1}{c}u_1(\xi)u_2 \Rightarrow
\tilde{A}G(\cdot,\xi) = 0,\; R_2G(\cdot,\xi) = 0.
\end{align*}
\item Für den Sprung an der Diagonlen erhält man,
\begin{align*}
\partial_1 G(\xi+0,\xi)- \partial_1G(\xi-0,\xi) &=
-\frac{1}{c}\left(u_1(\xi)u_2'(\xi) - u_1'(\xi)u_2(\xi)\right) \\ &=
-\frac{1}{p(x)}.\qedhere
\end{align*}
\end{enumerate}
\end{proof}

\begin{lem}
\label{prop:1.40}
$\exists c\in\R \forall u\in D(A) : \lin{Au,u} \ge -c\norm{u}^2$.\fishhere
\end{lem}
\begin{proof}
Übung.\qedhere
\end{proof}

\begin{prop}
\label{prop:1.41}
\begin{enumerate}[label=\arabic{*}.)]
\item Es gilt $\inf\setdef{\lambda\in\R}{\lambda\text{ ist EW von (1.1)}} >
-\infty$.
\item Falls $\alpha_1 = \beta_1 = 0$ (Neumann) oder $\alpha_2=\beta_2=0$
(Dirichlet), so gilt
\begin{align*}
\inf\setdef{\lambda\in\R}{\lambda\text{ ist EW von (1.1)}} \ge \frac{\min
q(x)}{\max r(x)}.\fishhere
\end{align*}
\end{enumerate}
\end{prop}

\begin{prop}
\label{prop:1.42}
Sei $d\in\R$ kein Eigenwert von (1.1) und $G'$ die Greensche Funktion von
\begin{align*}
A' := A- dr.
\end{align*}
und seien
\begin{align*}
&G(x,y) := \sqrt{r(x)}G'(x,y)\sqrt{r(y)},\\
&D(K) := C(I\to\R),\\
&K\ph(x) := \int_I G(x,y)\ph(y)\dy,
\end{align*}
dann ist $\mu=0$ kein Eigenwert von $K$ und es sind äquivalent
\begin{enumerate}[label=(\roman{*})]
\item\label{prop:1.42:1} $\lambda\left(=\frac{1}{\mu}+d\right)$ ist Eigenwert
von (1.1) mit Eigenfunktion $u\left(=\frac{1}{\sqrt{r}}v\right)$.
\item\label{prop:1.42:2} $\mu\left(=\frac{1}{\lambda - d}\right)$ ist Eigenwert
von $K$ mit Eigenfunktion $v=\left(\sqrt{r}u\right)$.\fishhere
\end{enumerate}
\end{prop}
\begin{proof}
$G'$ existiert da das Sturm-Liouville nicht $d$ als Eigenwert hat und
daher das ``verschobene" Problem
\begin{align*}
(pu')' - qu + \lambda ru - dru = 0,
\end{align*}
nicht $(\lambda-d)=0$ als Eigenwert besitzt.

Angenommen $K\ph(x) = 0$, dann gilt
\begin{align*}
&0 = \int_I G(x,y)\ph(y)\dy = \underbrace{\sqrt{r(x)}}_{>0}\int_I
G'(x,y)\sqrt{r(y)}\ph(y)\dy\\ \Leftrightarrow & 0=\int_I
G'(x,y)\sqrt{r(y)}\ph(y)\dy.
\end{align*}
Aus \ref{prop:1.37} folgt, $A0 = \sqrt{r}\ph$ und daher ist $\ph=0$ also kann
$\mu=0$ kein Eigenwert von $K$ sein.
\begin{align*}
\ref{prop:1.42:1}
&\Leftrightarrow Au = \lambda ru
\Leftrightarrow (A-dr)u = (\lambda - d)ru\\
&\Leftrightarrow u(x) = \int_I G'(x,y)(\lambda-d)r(y)u(y)\dy\\
&\Leftrightarrow \underbrace{\sqrt{r(x)}u(x)}_{:=v(x)} =
(\lambda-d)\int_I\sqrt{r(x)}G'(x,y)\sqrt{r(y)}\sqrt{r(y)} u(y)\dy\\
&\Leftrightarrow v(x) = \int_I G(x,y)v(y)\dy
\Leftrightarrow v = (\lambda-d)Kv\\
&\Leftrightarrow \underbrace{\frac{1}{\lambda-d}}_{:=\mu}v = Kv
\Leftrightarrow \ref{prop:1.42:2}\qedhere
\end{align*}
\end{proof}

\begin{lem}
\label{prop:1.43}
\begin{enumerate}[label=\arabic{*}.)]
  \item $K$ ist symmetrisch und kompakt in $L:=C(I\to\R)$.
  \item $\im K = \setdef{\sqrt{r}u}{u\in D(A)}$.\fishhere
\end{enumerate}
\end{lem}
\begin{proof}
\begin{enumerate}[label=\arabic{*}.)]
  \item Die Symmetrie folgt direkt aus $G(x,y)=G(y,x)$. Zur Kompaktheit
  siehe \ref{bsp:1.25}.
  \item
\begin{align*}
v\in \im K &\Leftrightarrow \exists \ph \in C(I\to\R) : v(x) = \int_I
G(x,y)\ph(y)\dy\\
& \Leftrightarrow v(x)=\sqrt{r(x)}\int_I
G'(x,y)\underbrace{\sqrt{r(y)}\ph(y)}_{\in L(I\to\R)}\dy.
\end{align*}
In \ref{prop:1.37} haben wir gesehen, dass $\int_I
G'(x,y)\underbrace{\sqrt{r(y)}\ph(y)}_{\in L(I\to\R)}\dy\in D(A)$.\qedhere
\end{enumerate}
\end{proof}

\begin{prop}[Satz von Sturm-Liouville]
\label{prop:1.44}
(1.1) besitzt abzählbar viele Eigenwerte $\lambda_j, j\in\N$. Es gilt
\begin{enumerate}[label=\arabic{*}.)]
  \item\label{prop:1.44:1} $\lambda_j\to+\infty$ für $j\to\infty$.
  \item\label{prop:1.44:2} Jeder Eigenwert hat Vielfachheit $1$.
  \item\label{prop:1.44:3} Die Eigenfunktionen $(u_j)$ können so normiert
  werden, dass die $(\sqrt{r}u_j)$ ein ONS bilden. Dieses ist vollständig in
  $\LL^2(I)$.\fishhere
\end{enumerate}
\end{prop}
\begin{proof}
``\ref{prop:1.44:2}'': Siehe \ref{bem:1.33}.

``\ref{prop:1.44:1}'': $\lambda$ ist Eigenwert von (1.1) ist äquivalent zu
$\mu=\frac{1}{\lambda-d}$ ist Eigenwert von $K$. $K$ ist symmetrisch und
kompakt besitzt daher abzählbar viele Eigenwerte $\mu_j$ und $\mu_j\to0$.

``\ref{prop:1.44:3}'': Da Eigenwerte von (1.1) einfach sind, sind auch die
Eigenwerte von $K$ einfach. Da $K$ symmetrisch, sind die Eigenfunktionen $(v_j)$
orthogonal. Normiert man diese, bilden sie ein ONS. Da $r>0$ bilden auch die
$v_j=\sqrt{r}u_j$ ein ONS. 
Der Hauptsatz sagt nun, dass dieses ONS vollständig in $\overline{\im K}$ ist.
Satz \ref{prop:1.37} besagt,
\begin{align*}
\overline{\im K} =
\overline{\setdef{\sqrt{r}u}{u\in D(A)}}
\end{align*}
Der Abschluss hängt nun vom umgebenden Raum ab,
\begin{align*}
&\overline{\im K}^{C(I\to\R)} = C(I\to\R),\\
&\overline{\im K}^{\LL^2(I\to\R)} = \LL^2(I\to\R).\qedhere
\end{align*}
\end{proof}

\begin{prop}[Bessere Konvergenz]
\label{prop:1.45}
Sei $K\subseteq \R^n$ kompakt, $G\in C(K\times K\to\R)$, $L:=C(K\to\C)$,
$G(x,y)=G(y,x)$,
\begin{align*}
Bf(x) := \int_K G(x,y)f(y)\dy,\quad y\in D(B) := L,
\end{align*}
$(e_j)$ Folge der orthonormierten Eigenfunktionen mit zugehörigen Eigenwerten
$\lambda_j$. Dann konvergiert für $f\in\im B$ die Fourierreihe
\begin{align*}
f = \sum\limits_{j=1}^\infty \lin{f,e_j}e_j,
\end{align*}
gleichmäßig und absolut in $K$.\fishhere
\end{prop}
\begin{proof}
\begin{enumerate}[label=\arabic{*}.)]
  \item $e_j\in C(K\to\C)$, denn
\begin{align*}
\lambda_j e_j(x) = \int_K G(x,y)e_j(y)\dy,
\end{align*}
ist stetig in $x$ und daher ist $e_j$ stetig.
\item Sei $f=Bg$ und $g\in L$, dann ist
\begin{align*}
\lin{f,e_j} = \lin{Bg,e_j} = \lin{g,Be_j} = \lambda_j\lin{g,e_j} 
\end{align*}
\begin{align*}
\abs{\sum\limits_{j=N}^{M+N} \abs{\lin{f,e_j}e_j(x)}}^2
&= \abs{\sum\limits_{j=N}^{M+N} \abs{\lambda_j\lin{g,e_j}e_j(x)}}^2\\
&\le \underbrace{\sum\limits_{j=N}^{N+M} \abs{\lin{g,e_j}}^2}_{\le \ep
\text{ für } N>N_\ep \text{ Bessel}} \underbrace{\sum\limits_{j=N}^{M+N}
\abs{\lambda_j e_j(x)}^2}_{\le c s.u.}
\end{align*}
Betrachte zur Abschätzung des 2. Faktors,
\begin{align*}
&\lambda_j e_j(x) = Be_j(x) = \int_K G(x,y)e_j(y)\dy = \lin{G(x,\cdot),e_j},\\
&\sum\limits_{j=N}^{M+N} \abs{\lambda_j e_j(x)}^2
= \sum\limits_{j=N}^{M+N}\abs{\lin{G(x,\cdot),e_j}}^2
\le \norm{G(x,\cdot)}_{\LL^2(K)} < \infty.\qedhere
\end{align*}
\end{enumerate}
\end{proof}

\subsubsection{Lösung der Wellengleichung}

Wir wollen nun die entwickelte Theorie dazu verwenden, das Motivationsproblem
der schwingenden Saite zu Beginn des Kapitels zu lösen. Betrachte dazu das
Rand- und Anfangswertproblem,
\begin{align*}
&\partial_t^2 u(t,x) - c(x)^2\partial_x^2u(t,x) = 0,\quad t\ge 0, x\in
I,\tag{1.3}\\
&\partial_x u(t,a) = \partial_x u(t,b) = 0,\\
&u(0,x) = g_0(x),\quad \partial_t u(0,x) = 0,\qquad t\ge 0, x\in I.
\end{align*}
Die Randbedingungen entsprechen also einer Saite mit freien Enden. $g_0$
gibt ihre Anfangsauslenkung an.

Zur Lösung betrachten wir folgendes Sturm-Liouville Problem
\begin{align*}
&u''(x) + \lambda\frac{1}{c^2(x)}u(x) = 0,\tag{1.4}\\
&u'(a) = u'(b)
\end{align*}
Obwohl dieses Problem nicht mehr zeitabhängig ist, werden wir sehen, dass wir
damit (1.3) lösen können.

Wenden wir den Satz von Sturm-Liouville auf (1.4) an, so erhalten wir als
Lösungen die Eigenfunktionen $(u_j)$ mit Eigenwerten $(\lambda_j)$ und
$\lambda_j\to\infty$. Die $(\frac{1}{c}u_j)$ bilden ein VONS von $\LL^2(I)$,
wir können also $g_0$ entwickeln zu,
\begin{align*}
&\frac{1}{c}g_0 = \sum\limits_{j=1}^\infty
\lin{\frac{1}{c}g_0,\frac{1}{c}u_j}\frac{1}{c}u_j\\
\Leftrightarrow & g_0 = \sum\limits_{j=1}^\infty
\lin{\frac{1}{c}g_0,\frac{1}{c}u_j}u_j.
\end{align*}
Wir wissen von die Fourierreihe jedoch nur, dass sie bezüglich $\norm{\cdot}_2$
konvergiert, über die punktweise Konvergenz können wir noch keine Aussage
machen.

\begin{propn}[Proposition]
Die Reihe
\begin{align*}
u(t,x) = \sum\limits_{j=1}^\infty
\cos\left(\sqrt{\lambda_j}t\right)\lin{\frac{1}{c}g_0,\frac{1}{c}u_j}u_j(x)\tag{1.5},
\end{align*}
konvergiert unter möglichen Zusatzvoraussetzungen punktweise für $t\ge 0$ und
$x\in I$ und löst (1.3).\fishhere
\end{propn}

Im Folgenden wollen wir die Aussage beweisen und die Zusatzvoraussetzungen,
die wir an das Problem stellen müssen, erarbeiten. Wir werden daher in einigen
Fällen annehmen, dass Eigenschaften erfüllt sind, über die wir noch nichts
aussagen können.

\begin{proof}[Beweis der Propostion.]
\begin{enumerate}[label=\arabic{*}.)]
\item \textit{Anfangsbedinung}.
\begin{align*}
u(0,x) = \sum\limits_{j=1}^\infty
1\cdot \lin{\frac{1}{c}g_0,\frac{1}{c}u_j}u_j(x) \overset{!}{=} g_0(x),
\end{align*} 
falls die Reihe punktweise für jedes $x\in I$ konvergiert.
\begin{align*}
\partial_t u(t,x)\bigg|_{t=0} \overset{!}{=} -\sum\limits_{j=1}^\infty
\sqrt{\lambda_j}\sin\left(\sqrt{\lambda_j}t\right)\lin{\frac{1}{c}g_0,\frac{1}{c}u_j}u_j(x)\bigg|_{t=0}
= 0,
\end{align*}
falls die Reihe mit $\partial_t$ vertauscht.
\item \textit{Randbedingungen}.
\begin{align*}
u(t,a) = \sum\limits_{j=1}^\infty
\cos\left(\sqrt{\lambda_j}t\right)\lin{\frac{1}{c}g_0,\frac{1}{c}u_j}\underbrace{u_j(a)}_{=0}
= 0,
\end{align*}
da $u_j$ Lösung von (1.4) ist. $u(t,b)$ analog.
\item \textit{Differentialgleichung}. Falls $\partial_t^2$ und $\partial_x^2$
mit der Reihe vertauschen, gilt
\begin{align*}
&\partial_t^2 u(t,x) = -\sum\limits_{j=1}^\infty \lambda_j
\cos\left(\sqrt{\lambda_j}t\right)\lin{\frac{1}{c}g_0,\frac{1}{c}u_j}u_j(x),\\
&\partial_x^2 u(t,x) = \sum\limits_{j=1}^\infty 
\cos\left(\sqrt{\lambda_j}t\right)\lin{\frac{1}{c}g_0,\frac{1}{c}u_j}u_j''(x).
\end{align*}
$u_j''$ erfüllt die Gleichung (1.4) und daher gilt
\begin{align*}
&\partial_x^2 u(t,x) = -\sum\limits_{j=1}^\infty 
\cos\left(\sqrt{\lambda_j}t\right)\lin{\frac{1}{c}g_0,\frac{1}{c}u_j}\frac{\lambda_j}{c^2(x)}u_j(x).
\end{align*}
Damit ist $\partial_t^2 u(t,x) - c(x)^2\partial_x^2u(t,x) = 0$ und $u$ löst
(1.3).
\item \textit{Konvergenzfragen}. Wir betrachten zunächst die Vertauschung der
Reihe mit $\partial_t^2$. Es gilt 
\begin{align*}
\lambda_j\lin{\frac{1}{c}g_0,\frac{1}{c}u_j}
&= \lin{\frac{1}{c}g_0,\frac{\lambda_j}{c}u_j}
= -\lin{\frac{1}{c}g_0,cu_j''}
= -\lin{g_0,u_j''} \\ 
&= \lin{g_0,Au_j} = -\lin{g_0'',u_j},
\end{align*}
falls $g_0\in D(A)$. Somit gilt,
\begin{align*}
&\abs{\cos\left(\sqrt{\lambda_j}t\right)\lambda_j\lin{\frac{1}{c}g_0,\frac{1}{c}u_j}u_j(x)}
\le 1\abs{\lin{g_0'',u_j}u_j(x)}.\\
\Rightarrow & \sum \abs{\lin{g_0'',u_j}u_j(x)}
= c(x)\sum
\abs{\lin{cg_0'',\underbrace{\frac{1}{c}u_j}_{=e_j}}\underbrace{\frac{1}{c}u_j}_{=e_j}}
\end{align*}
% ist eine konvergente Majorante. \ref{prop:1.45} besagt nun, sie konvergiert
% gleichmäßig. Setzte nun
% \begin{align*}
% \sum \lin{g_0'',u_j}u_j(x) = 
% c(x)\sum\lin{cg_0'',\underbrace{\frac{1}{c}u_j}_{=e_j}}\underbrace{\frac{1}{c}u_j}_{=e_j},
% \end{align*}
ist eine Majorante, wobei $e_j$ das ONS aus Eigenfunktionen zu $K$
ist\footnote{Siehe Beweis zu Satz von Sturm Liouville}. Mit \ref{prop:1.45} folgt, dass
\begin{align*}
\sum\lin{cg_0'',e_j}e_j(x)
\end{align*} 
gleichmäßig bezüglich $x\in I$ konvergiert, falls $cg_0''\in\im K$. Nun ist
\begin{align*}
\im K = \setdef{\sqrt{r}u}{u\in D(A)},
\end{align*}
wobei $\sqrt{r}=\frac{1}{c}$, also muss $g_0\in D(A)$
vorausgesetzt werden. Mit dem Weierstraß-Kriterium folgt daher, dass die Reihe,
\begin{align*}
-\underbrace{\sum\limits_{j=1}^\infty \lambda_j
\cos\left(\sqrt{\lambda_j}t\right)\lin{\frac{1}{c}g_0,\frac{1}{c}u_j}u_j(x)}_{(*)},
\end{align*}
gleichmäßig bezüglich $x\in I$ konvergiert. Die gefundene Majorante hängt aber
nicht mehr von $t$ ab, also konvergiert (*) auch gleichmäßig für $t\ge 0$. Wir
haben bereits gezeigt, dass die Reihe
\begin{align*}
-\underbrace{\sum\limits_{j=1}^\infty
\sqrt{\lambda_j}\sin\left(\sqrt{\lambda_j}t\right)\lin{\frac{1}{c}g_0,\frac{1}{c}u_j}u_j(x)}_{(**)}
\end{align*}
für $t=0$ konvergiert. Mit dem Satz aus der Analysis folgt somit, dass man
$\partial_t$ und (**) vertauschen kann und (**) ebenfalls gleichmäßig
konvergiert. Nun konvergiert auch (1.5) absolut und gleichmäßig bezüglich
$x\in I$ und $t\ge 0$, da sich (1.5) und (*) nur durch den Faktor $\lambda_j$
unterscheiden und $\lambda_j\to\infty$, also ist (*) eine Majorante von (1.5). $\partial_t^2$
lässt sich somit mit (1.5) vertauschen.
\begin{align*}
\sum\limits_{j=1}^\infty
\cos\left(\sqrt{\lambda_j}t\right)\lin{\frac{1}{c}g_0,\frac{1}{c}u_j}u_j''(x)
\end{align*}
konvergiert ebenfalls gleichmäßig (siehe (*)), da $u_j$ die
Differentialgleichung efüllt. Setze
\begin{align*}
\underbrace{\sum\limits_{j=1}^\infty
\cos\left(\sqrt{\lambda_j}t\right)\lin{\frac{1}{c}g_0,\frac{1}{c}u_j}u_j'(x)}_{(***)},
\end{align*}
dann konvergiert (***) aufgrund der Randbedingung für $x=a$ und $x=b$, wir
können also wieder den Satz aus der Analysis anwenden und erhalten, dass (***)
absolut und gleichmäßig konvergiert. Wir können somit $\partial_x^2$ und (1.5)
vertauschen.\qedhere
\end{enumerate}
\end{proof}

Der folgende Satz fasst die Voraussetzungen an (1.5) zusammen.
\begin{prop}
\label{prop:1.46}
Seien
\begin{align*}
D(A):= \setdef{u\in C^2(I\to\R)}{u'(a) = u'(b)= 0},
\end{align*}
$g_0$ und $c^2g_0''\in D(A)$, dann besitzt (1.3) eine Lösung gegeben
durch (1.5).\fishhere
\end{prop}

\begin{prop}[Offene Fragen]
\label{prop:1.47}
\begin{enumerate}[label=\arabic{*}.)]
  \item Ist die Lösung von (1.3) eindeutig?
  \item Wie verhält es sich mit anderen Randbedingungen?
  \item Können die Voraussetzungen weiter abgeschwächt werden?\fishhere
\end{enumerate}
\end{prop}
\begin{proof}
\begin{enumerate}[label=\arabic{*}.)]
  \item Ja, man kann dies zeigen.
  \item Für unseren Beweis war es maßgeblich, dass $u'(a) = 0$. Der Beweis
  klappt jedoch auch für andere Randbedingungen, man muss in diesem Fall $u'$
  durch $u$ und $u''$ interpolieren.
  \item Ja, näheres sagt der Satz von Merces.\qedhere
\end{enumerate}
\end{proof}

\begin{bem}
\label{bem:1.48}
Die Funktionen $\cos\left(\sqrt{\lambda_j}t\right)u_j(x)$ sind die
Eigenfunktionen des Systems. Die Lösungsstruktur (1.5) ist sehr allgemein und
ergibt sich auch bei viel allgemeineren Problemen.\maphere
\end{bem}

\subsection{Sinus-Cosinus-Reihen}

Sei $L:=\LL^2([-\pi,\pi])$ und
\begin{align*}
e_j = \begin{cases}
\frac{1}{\sqrt{2\pi}}, & j=0,\\
\frac{1}{\sqrt{\pi}}\cos\left(\frac{j+1}{2}x\right), & \frac{j-1}{2}\in\N,\\
\frac{1}{\sqrt{\pi}}\sin\left(\frac{j}{2}x\right), & \frac{j}{2}\in\N.
\end{cases}
\end{align*}
Die $(e_j)$ bilden ein ONS in $L$. Man kann mit Hilfe des
Sturm-Liouville-Problems zeigen, dass dieses ONS vollständig in $L$ ist, d.h.
\begin{align*}
x = \sum\limits_{j=1}^\infty \lin{x,e_j}e_j,\quad \forall x\in L.
\end{align*}
Dies ist äquivalent dazu, dass die Parsevallsche Gleichung für jedes $x\in L$
gilt,
\begin{align*}
\sum\limits_{j=1}^\infty \abs{\lin{x,e_j}}^2 = \norm{x}^2,\quad \forall x\in L.
\end{align*}
Die für unseren Fall übliche Notation für Fourier-Koeffizienten der
Fourierentwicklung von $f$ ist
\begin{align*}
&a_j = \frac{1}{\pi}\int\limits_{-\pi}^\pi f(t)\cos(jt)\dt,\qquad j\in\N,\\
&b_j = \frac{1}{\pi}\int\limits_{-\pi}^\pi f(t)\sin(jt)\dt,\qquad j\in\N.
\end{align*}
Die $n$-te Partialsumme der Fourierreihe ist gegeben durch
\begin{align*}
&s_n(x) := \frac{a_0}{2} + \sum\limits_{j=1}^n \left(a_j\cos(jx) + b_j\sin(jx)
\right)\\
&s(x) = \lim\limits_{j\to\infty} s_n(x).
\end{align*}
Die Vollständigkeit besagt nun, dass $\norm{s_n-f}\to0$. Im Folgenden wollen
wir zusätzlich die punktweise Konvergenz $s_n(x)\to f(x)$ betrachten. Für
$f\in C^2$ haben wir bereits gesehen, dass die Reihe auch punktweise
konvergiert.

\begin{bem}[Beobachtungen.]
\label{bem:1.49}
\begin{enumerate}[label=\arabic{*}.)]
  \item Ist $f$ auf $\R$ definiert und $2\pi$-periodisch, so können die
  Integrale beliebig verschoben werden,
\begin{align*}
a_j = \frac{1}{\pi}\int\limits_{-\pi}^\pi f(t)\cos(jt)\dt
= \frac{1}{\pi}\int\limits_{-\pi+c}^{\pi+c} f(t)\cos(jt)\dt,\quad c\in\R.
\end{align*}
\item $a_j,b_j$ und damit $s$ ändern sich nicht, wenn $f$ auf einer Nullmenge
abgeändert wird.
\item Zur Definition von $a_j,b_j$ zu $f$ reicht $f\in
\LL^1([-\pi,\pi])\varsupsetneq \LL^2([-\pi,\pi])$.
\end{enumerate}
\end{bem}

\begin{lem}
\label{prop:1.50}
Falls $f\in C^1([-\pi,\pi]\to\C)$, dann gilt
\begin{align*}
&\lim\limits_{\omega\to\infty} \int\limits_{-\pi}^\pi f(t)\cos(\omega t)\dt =
0,\tag{*}\\
&\lim\limits_{\omega\to\infty} \int\limits_{-\pi}^\pi f(t)\sin(\omega t)\dt =
0.\fishhere
\end{align*} 
\end{lem}
\begin{proof}
Wir zeigen die Behauptung nur für den Sinus, sie folgt für den Cosinus analog.
\begin{align*}
\int\limits_{-\pi}^\pi f(t)\sin(\omega t)\dt
= -\frac{1}{\omega}\underbrace{f(t)}_{\abs{\cdot}\le C_f}\underbrace{\cos(\omega
t)}_{\le 1} + \int\limits_{-\pi}^\pi \frac{1}{\omega}
\underbrace{f'(t)}_{\abs{\cdot}\le C_{f'}}\underbrace{\cos(\omega t)}_{\le
1}\dt\to 0\qedhere
\end{align*}
\end{proof}

\begin{lem}[Lemma von Riemann]
\label{prop:1.51}
Falls $f\in\LL^1([-\pi,\pi])$ gilt ebenfalls (*).\fishhere
\end{lem}
\begin{proof}
Sei $f\in\LL^1([-\pi,\pi])$, $\ep>0$ fest. $C^1([-\pi,\pi])$ liegt dicht in
$\LL^1([-\pi,\pi])$, d.h.
\begin{align*}
\exists f_\ep\in C^1([-\pi,\pi]) : \norm{f-f_\ep} <\ep.
\end{align*}
Es gilt nun unabhängig für $\omega > 0$,
\begin{align*}
\abs{\int\limits_{-\pi}^\pi (f(t)-f_\ep(t))\sin(\omega t)\dt}
\le \int\limits_{-\pi}^\pi \abs{f(t)-f_\ep(t)}\dt = \norm{f-f_\ep}_1 < \ep.
\end{align*}
Aus \ref{prop:1.50} folgt,
\begin{align*}
\abs{\int\limits_{-\pi}^\pi f_\ep(t)\sin(\omega t)\dt} <\ep,
\end{align*}
für $\omega > \omega_\ep$. Insgesamt ergibt sich,
\begin{align*}
\abs{\int\limits_{-\pi}^\pi f(t)\sin(\omega t)\dt}
&= \abs{\int\limits_{-\pi}^\pi (f-f_\ep)(t)\sin(\omega t)\dt +
\int\limits_{-\pi}^\pi f_\ep(t)\sin(\omega t)\dt} \\ &< 2\ep,
\end{align*}
für $\omega > \omega_\ep$.\qedhere
\end{proof}

\begin{lem}
\label{prop:1.52}
Für $f\in \LL^1([-\pi,\pi])$, $2\pi$-periodisch fortgesetzt gilt,
\begin{align*}
s_n(x)-f(x) = \int\limits_{-\pi}^\pi \left[f(x+t) -
f(x)\right]\underbrace{\frac{\sin\left(\left(n+\frac{1}{2}\right)t\right)}
{2\pi\sin\left(\frac{t}{2}\right)}}_{(**)},
\end{align*}
wobei (**) stetig ergänzbar bei $t=0$.\fishhere
\end{lem}

Für den Beweis benötigen wir noch etwas Vorbereitung.

\begin{prop}[Krierium von Dini]
\label{prop:1.53}
Es sei $f\in\LL^1([-\pi,\pi])$, $2\pi$-periodisch fortgesetzt, $x\in[-\pi,\pi]$
und
\begin{align*}
\exists \delta > 0 : \int\limits_{-\delta}^\delta \frac{f(x+t)-f(x)}{t}\dt <
\infty.
\end{align*}
Dann gilt $f(x) = \lim\limits_{n\to\infty} s_n(x)$.\fishhere
\end{prop}
\begin{proof}
Wegen $f\in\LL^1$ und der Integrationsbedingung gilt
\begin{align*}
g(t) = \frac{f(x+t)-f(x)}{t} \in \LL^1([-\pi,\pi]).
\end{align*}
Mit \ref{prop:1.52} folgt
\begin{align*}
s_n(x)-f(x) = \frac{1}{2\pi}\int\limits_{-\pi}^\pi
g(t)\frac{t}{\sin\left(\frac{t}{2}\right)}\sin\left(\left(n+\frac{1}{2}\right)t\right)\dt
\end{align*}
Nun ist $g\in\LL^1$ und $\dfrac{t}{\sin\left(\frac{t}{2}\right)}$ ist stetig
fortsetzbar, also ist auch das Produkt in $\LL^1$ und damit ist der ganze
Integrand $\LL^1$ und das Integral verschwindet nach dem Lemma von Riemann.\qedhere
\end{proof}

\begin{proof}[Beweis von \ref{prop:1.52}.]
\begin{align*}
s_n(x) &= \frac{1}{2\pi}\int\limits_{-\pi}^\pi f(t)\dt\\
&+ \frac{1}{\pi}\sum\limits_{j=1}^n \int\limits_{-\pi}^\pi
f(t)\cos(jt)\cos(jx) + f(t)\sin(jt)\sin(jx)\dt\\
&= \frac{1}{2\pi}\left[ 
\int\limits_{-\pi}^\pi f(t)\dt + \sum\limits_{j=1}^n \int\limits_{-\pi}^\pi 
f(t)\left(e^{ij(x-t)} + e^{ij(t-x)}\right)\dt\right]\\
&= \int\limits_{-\pi}^\pi \underbrace{\frac{1}{2\pi}\sum\limits_{j=-n}^n
e^{ij(x-t)}}_{=: D_n(x-t)}f(t)\dt
\end{align*}
Nun gilt für $D_n$,
\begin{align*}
2\pi D_n(\xi) &= \sum\limits_{j=-n}^n e^{ij\xi}
\overset{j\mapsto j+n}{=} e^{-in\xi}\sum\limits_{j=0}^{2n} e^{ij\xi}
= e^{-in\xi}\left(\frac{1-e^{i(2n+1)\xi}}{1-e^{i\xi}}\right)\\
&= \underbrace{e^{-in\xi}
\frac{e^{i(n+\frac{1}{2})\xi}}{e^{i\frac{1}{2}\xi}}}_{=1} 
\frac{e^{-i(n+\frac{1}{2})\xi}-e^{i(n+\frac{1}{2})\xi}}{e^{-i\frac{1}{2}\xi}-e^{i\frac{1}{2}\xi}}
= \frac{-2i\sin\left(\left(n+\frac{1}{2}\right)\xi\right)}
{-2i\sin\left(\frac{1}{2}\xi\right)}\\
&= \frac{\sin\left(\left(n+\frac{1}{2}\right)\xi\right)}
{\sin\left(\frac{1}{2}\xi\right)}
\end{align*}
$D_n(\xi)$ nennt man den \emph{Dirichlet-Kern}. Für die Partialsumme gilt daher
\begin{align*}
s_n(x) = \int\limits_{-\pi}^\pi f(t)D_n(x-t)\dt.
\end{align*}
Für den Spezialfall $f=1$ konstant, sind alle $a_j,b_j = 0$ außer
$\frac{a_0}{2}$, d.h.
\begin{align*}
s_n(x) = 1 \Rightarrow \int\limits_{-\pi}^\pi D_n(x-t)\dt = 1.
\end{align*}
Nun ist $D_n$ $2\pi$-periodisch, d.h. $D_n(\xi+2\pi) = D_n(\xi)$, also gilt auch
\begin{align*}
\int\limits_{-\pi}^\pi D_n(\xi)\dxi = 1.\tag{*}
\end{align*}
$f$ ist ebenfalls $2\pi$-periodisch, wir können also das Integral verschieben,
\begin{align*}
s_n(x) &= \int\limits_{-\pi}^\pi f(t)D_n(x-t)\dt
= \int\limits_{-\pi+x}^{\pi+x} f(t)D_n(x-t)\dt\\
&\overset{\xi=t-x}{=} \int\limits_{-\pi}^{\pi}
f(\xi+x)\underbrace{D_n(-\xi)}_{=D_n(\xi)}\dxi\qedhere
\end{align*}
\end{proof}

\begin{prop}
\label{prop:1.54}
Das Kriterium von Dini ist erfüllt, falls $f\in\LL^1([-\pi,\pi])$,
$2\pi$-periodisch fortgesetzt, \emph{hölderstetig} in $x$ ist, d.h.
\begin{align*}
\exists\delta > 0 \exists\alpha > 0 \exists c > 0 \forall t\in[-\delta,\delta]
: \abs{f(x-t)-f(x)} \le c\abs{t}^\alpha.
\end{align*}
und
\begin{align*}
\int\limits_{-\delta}^\delta \abs{\frac{f(x+t)-f(x)}{t}}\dt \le
c\int\limits_{-\delta}^\delta \abs{t}^{\alpha-1}\dt =
2c\frac{\delta^\alpha}{\alpha} <\infty.\fishhere
\end{align*}
\end{prop}

\begin{prop}[Erweitertes Kriterium von Dini]
\label{prop:1.55}
Sei $f\in\LL^1([-\pi,\pi])$, $2\pi$-periodisch fortgesetzt und $x\in[-\pi,\pi]$
eine \emph{Unstetigkeitsstelle 1. Art}, d.h.
\begin{align*}
&\int\limits_{0}^\delta \abs{\frac{f(x+t)-f(x+0)}{t}} \dt < \infty,\\
&\int\limits_{-\delta}^0 \abs{\frac{f(x+t)-f(x-0)}{t}} \dt < \infty,
\end{align*}
so konvergiert die Fourierreihe in $x$ und es gilt,
\begin{align*}
\lim\limits_{n\to\infty} s_n(x) =
\frac{1}{2}\left[f(x+0)+f(x-0)\right].\fishhere
\end{align*}
\end{prop}
%TODO: Bild, Unstetigkeitsstelle
\begin{bspn}
\begin{align*}
&f(x) =
\begin{cases}
0, & -\pi\le x <0,\\
1, & 0\le x\le \pi,
\end{cases}\\
\Rightarrow &
\lim\limits_{n\to\infty} s_n(x)
=
\begin{cases}
\frac{1}{2}, & x = -\pi,0,\pi,\\
0, & -\pi< x <0,\\
1, & 0< x< \pi.\bsphere 
\end{cases}
\end{align*}
\end{bspn}
\begin{proof}
Der Dirichletkern ist symmetrisch, $D(\xi) = D(-\xi)$ und es gilt,
\begin{align*}
\int\limits_{-\pi}^\pi D_n(\xi)\dxi = 1\Rightarrow \int\limits_{0}^\pi =
\frac{1}{2} = \int\limits_{-\pi}^0 =\frac{1}{2}.
\end{align*}
Wir können somit schreiben,
\begin{align*}
\frac{1}{2}\left[f(x+0)+f(x-0)\right] = \int\limits_0^\pi f(x+0)D_n(\xi)\dxi
 + \int\limits_{-\pi}^0 f(x-0)D_n(\xi)\dxi.
 \end{align*}
Satz \ref{prop:1.52} besagt nun,
\begin{align*}
&s_n(x) = \int\limits_{-\pi}^\pi f(x+t)D_n(t)\dt,
\end{align*}
es folgt somit,
\begin{align*}
&s_n(x) - \frac{1}{2}\left[f(x+0)+f(x-0)\right]\\
&= \int\limits_0^\pi \left[f(x+t) - f(x+0)\right]D_n(t)\dt
+ \int\limits_{-\pi}^0 \left[f(x+t) - f(x-0)\right]D_n(t)\dt\\
&\overset{\ref{prop:1.53}}{\to} 0 \text{ für } n\to\infty.\qedhere
\end{align*}
\end{proof}

\begin{prop}[Lipschitzkriterium]
\label{prop:1.56}
Sei $f\in C(\R\to\R)$, $2\pi$-periodisch fortgesetzt und
\begin{align*}
\exists \alpha > 0, c > 0 : \forall x,x'\in\R : \abs{f(x)-f(x')} \le
c\abs{x-x'}^\alpha.
\end{align*}
Dann konvergiert $(s_n)$ auf $\R$ gleichmäßig gegen $f$.\fishhere
\end{prop}
\begin{proof}
Sei $x\in[-\pi,\pi]$, und $\ep > 0$.
\begin{enumerate}[label=\arabic{*}.)]
  \item Aus \ref{prop:1.52} folgt
\begin{align*}
\int\limits_{-\delta}^\delta
&\abs{f(x+t)-f(x)}\abs{\frac{\sin\left(\left(n+\frac{1}{2}\right)t
\right)}{2\pi\sin\frac{t}{2}}}\dt\le \frac{c}{2\pi} \int\limits_{-\delta}^\delta
\abs{t}^{\alpha-1} \frac{\abs{t}}{\abs{\sin\frac{t}{2}}} \dt.
\end{align*}
$\dfrac{\abs{t}}{\abs{\sin\frac{t}{2}}}$ kann stetig in Null fortgesetzt werden
und hat daher ein Maximum $d$ auf $[-\delta,\delta]$. Es gilt
\begin{align*}
\le \frac{cd}{2\pi}\int\limits_{-\delta}^\delta \abs{t}^{\alpha-1}\dt =
\frac{2cd}{2\pi}\frac{\delta^\alpha}{\alpha}.
\end{align*}
Und daher ist
\begin{align*}
\int\limits_{-\delta}^\delta \abs{f(x+t)-f(x)}\abs{D_n(t)}\dt < \ep,
\end{align*}
für $0<\delta<\delta_\ep$.
\item Wähle nun $\delta\in(0,\delta_\ep)$ fest. $f$ ist stetig auf
$[-\delta,\delta]$, das Intervall ist kompakt, also ist $f$ dort gleichmäßig
stetig, die $2\pi$-periodische Fortsetzung ist somit gleichmäßig stetig auf
ganz $\R$. Wähle nun eine Zerlegung des Intervalls,
\begin{align*}
-\pi = \tau_0 < \tau_1 < \ldots < \tau_m = -\delta < \delta < \tau_{m+1} <
\ldots < \tau_{2m+1} = \pi,
\end{align*}
so dass
\begin{align*}
\abs{f(x+\tau_j)- f(x+t)} < \ep,
\end{align*}
für $\tau_j \le t \le \tau_{j+1}$, $j\neq m$.
% Definiere die Treppenfunktion,
% \begin{align*}
% g(t) = f(x-\tau_j),\quad\text{für } \tau_j \le t\le  
% \end{align*}
Dann gilt
\begin{align*}
&\sum\limits_{j=1}^{m-1} \int\limits_{\tau_j}^{\tau_{j+1}}
\abs{f(x+\tau_j)-f(x+t)}\abs{D_n(t)}\dt\\
 +& \sum\limits_{j=m}^{2m+1}
\int\limits_{\tau_j}^{\tau_{j+1}} \abs{f(x+\tau_j)-f(x+t)}\abs{D_n(t)}\dt\\
 \le&\ep \frac{1}{\abs{\sin\frac{\delta}{2}}}
\end{align*}
\item Definiere für festes $x$ die Treppenfunktion
\begin{align*}
g_x(t) = f(x+\tau_j),\quad\text{für } \tau_j \le t < \tau_{j+1}.
\end{align*}
Dann erhalten wir,
\begin{align*}
&\abs{\int\limits_{[-\pi,-\delta]}^{[\delta,\pi]} \left(g_x(t) - f(x)\right)
D_n(t)\dt}\\
&= \bigg|\sum\limits_{j=1}^{m-1}\int\limits_{\tau_j}^{\tau_{j+1}}
\left(f(x+\tau_j)-f(x)\right)D_n(t) + 
\sum\limits_{j=m}^{2m-1}\int\limits_{\tau_j}^{\tau_{j+1}}
\ldots
\bigg|\\
&\le 2\max\limits_{\R} \abs{f}
\left[
\sum\limits_{j=1}^{m-1} \abs{\int\limits_{\tau_j}^{\tau_{j+1}} D_n(t)\dt}
+
\sum\limits_{j=m}^{2m-1}\abs{\int\limits_{\tau_j}^{\tau_{j+1}} D_n(t)\dt}
\right]\tag{*}
\end{align*}
 %= \abs{\sum\limits_{j=1}^{m-1}\int\limits_{\tau_j}^{\tau_{j+1}}
%\ldots + \sum\limits_{j=m}^{2m-1}\int\limits_{\tau_j}^{\tau_{j+1}} \ldots}\\
Der Dirichletkern ist unabhängig von $x$ und es gilt,
\begin{align*}
\int\limits_{\tau_j}^{\tau_{j+1}} D_n(t)\dt =
\int\limits_{\tau_j}^{\tau_{j+1}}
\underbrace{\frac{1}{2\pi\sin\frac{t}{2}}}_{\in\C^\infty}
\sin\left(\left(n+\frac{1}{2}\right)t\right) \dt \to 0
\end{align*}
nach dem Lemma von Riemann. Damit ist $\abs{*} < \ep$ für $n > N_\ep$ und
\begin{align*}
\abs{s_n(x)-f(x)} <\ep,
\end{align*}
für $n > N_\ep$ und unabhängig von $x$.\qedhere
\end{enumerate}
\end{proof}

\begin{bem}
\label{bem:1.57}
Wenn $f\in C^1(\R^\to\R)$ und $2\pi$-periodisch, dann gilt
\begin{align*}
\abs{f(x)-f(x')} \overset{\text{MWS}}{\le} \abs{f'(xi)}\abs{x-x'} \le
\max\limits_\R \abs{f'}\abs{x-x'},
\end{align*}
und damit ist die Lipschitz-Bedingung erfüllt.\maphere
\end{bem}

\begin{prop}[Komplexe Darstellung]
\label{prop:1.58}
\begin{align*}
&s_n(x) = \sum\limits_{j=-n}^n \lin{f,e_j}e_j(x),
\end{align*}
mit
\begin{align*}
&e_{j}(x) = \frac{1}{\sqrt{2\pi}}e^{ijx}.
\end{align*}
Die $(e_j)$ bilden ein vollständiges ONS von $\LL^2([-\pi,\pi])$.\fishhere
\end{prop}
\begin{proof}

Aus dem Beweis von \ref{prop:1.52} wissen wir,
\begin{align*}
s_n(x) = \int\limits_{-\pi}^\pi \frac{1}{2\pi}\sum\limits_{j=-n}^n
e^{ij(x-t)}f(t)\dt
= \sum\limits_{j=-n}^n \int\limits_{-\pi}^\pi \frac{e^{-ijt}}{\sqrt{2\pi}}
f(t) \frac{e^{ijx}}{\sqrt{2\pi}}\dt.
\end{align*}
Anders angeschaut erhalten wir so
\begin{align*}
&e_{j}(x) = \frac{1}{\sqrt{2\pi}}e^{ijx},\\
&s_n(x) = \sum\limits_{j=-n}^n \lin{f,e_j}e_j(x).
\end{align*}
$(e_j)$ ist ein ONS in $\LL^2([-\pi,\pi])$, denn
\begin{align*}
\lin{e_j,e_k} = \frac{1}{2\pi}\int\limits_{-\pi}^\pi e^{i(j-k)x} = 
\begin{cases}
1, & j=k,\\
\frac{1}{2\pi}\frac{e^{(j-k)x}}{i(j-k)}\bigg|_{x=-\pi}^\pi = 0, & j\neq k.
\end{cases}
\end{align*}

$C_0^1((-\pi,\pi)\to\C)$ liegt dicht in $\LL^2([-\pi,\pi])$, wähle nun zu
$f\in\LL^2$ ein $\tilde{f}\in C_0^1$ mit $\norm{f-\tilde{f}}_2<\ep$. Setzt man
$\tilde{f}$ $2\pi$-periodisch fort, erhält man $f\in C^1(\R\to\C)$. Sei
$\tilde{s}_n$ die Fouriersumme zu $\tilde{f}$, dann gilt,
\begin{align*}
\abs{\tilde{s}_n - \tilde{f}} < \frac{\ep}{2\pi}, 
\end{align*}
für $N>N_\ep$ und daher ist,
\begin{align*}
\norm{\tilde{s}_n - \tilde{f}}^2 = \int\limits_{[-\pi,\pi]} \abs{\tilde{s}_n(x)
-\tilde{f}(x)}\dx \le \ep^2.
\end{align*}
Damit gilt,
\begin{align*}
\norm{f-s_n} \le \underbrace{\norm{f-\tilde{f}}}_{<\ep} +
\underbrace{\norm{\tilde{f}-\tilde{s}_n}}_{\le \ep, n>N_\ep} +
\norm{\tilde{s}_n - s_n}.
\end{align*}
Der letzte Summand ist die Fouriersumme von $\tilde{f}-f$, da die Fouriersumme
linear in $f$ ist. Die Besselungleichung besagt nun, dass
\begin{align*}
\norm{\tilde{s}_n - s_n} \le \norm{\tilde{f}-f} < \ep,
\end{align*}
somit ist
\begin{align*}
\norm{f-f_n} < 3\ep,\quad\text{für } n> N_\ep.
\end{align*}
Also ist das ONS $(e_j)$ vollständig.\qedhere
\end{proof}
Wir haben also drei Konvergenzmöglichkeiten.
\begin{enumerate}
  \item Ist $f\in\LL^2$, so konvergiert $s_n\to f$ bezüglich $\norm{\cdot}_2$.
  \item Erfüllt $f$ das Dinikriterium in $x$, so konvergiert
  $s_n(x)\to f(x)$ punktweise.
  \item Erfüllt $f$ das Lipschitzkriterium, so konvergiert $s_n\unito f$ glm.
  bezüglich $x$.
\end{enumerate}

\begin{prop}[Andere Intervalle]
\label{prop:1.59}
Sei $f$ $2\pi$-periodisch und erfülle das Lipschitzkriterium, so ist auch
\begin{align*}
g(x) = f\left(\frac{Lx}{\pi}\right)
\end{align*}
$2\pi$-periodisch und erfüllt ebenfalls das Lipschitzkriterium.\fishhere
\end{prop}
\begin{proof}
Sei $s_n$ die Fouriersumme von $g$,
\begin{align*}
f\left(\frac{Lx}{\pi}\right) := g(x) = \lim\limits_{n\to\infty} s_n(x) = 
\lim\limits_{n\to\infty} \sum\limits_{j=-n}^n \frac{1}{2\pi}
\int\limits_{-\pi}^\pi e^{-ijt}f\left(\frac{Lt}{\pi}\right)e^{ijx}\dt.
\end{align*}
Substituiere $s = \frac{Lt}{\pi}$, so ist $\dt = \frac{\pi}{L}\ds$,
\begin{align*}
f\left(\frac{Lx}{\pi}\right) = \lim\limits_{n\to\infty} \sum\limits_{j=-n}^n
\frac{1}{2L}\int\limits_{-L}^L
e^{-ij\frac{\pi s}{L}}f(s)e^{ijx}\ds.
\end{align*}
Setzen wir nun $y = \frac{Lx}{\pi}$,
\begin{align*}
f(y) = \lim\limits_{n\to\infty} \sum\limits_{j=-n}^n
\frac{1}{2L}\int\limits_{-L}^L
e^{-ij\frac{\pi s}{L}}f(s)\ds e^{ij\frac{\pi y}{L}}.
\end{align*}
Es gelten nun dieselben Sätze mit $f_j(y) := e^{ij\frac{\pi}{L}y}$ anstelle von
$e_j$. $(f_j)$ ist ein VONS in $\LL^2([-L,L])$.\qedhere
\end{proof}

\begin{bsp}
\label{bsp:1.60}
Löse die Differentialgleichung,
\begin{align*}
ay''(x) + by'(x) + y(x) = f(x),\quad -L\le x\le L,
\end{align*}
mit
\begin{align*}
y(-L) = y(L),\quad y'(-L) = y'(L),\quad a,b\in\R.
\end{align*}
Entwickle $f,y,y'$ und $y''$ und rechne formal, d.h. man geht davon aus, dass
die Objekte die Voraussetzungen für alle verwendeten Operationen erfüllen und prüft erst
abschließend welche Bedingungen gestellt werden müssen.
\begin{align*}
&f(x) = \sum\limits_{j=-\infty}^\infty \lin{f,f_j} f_j(x),\\
&y(x) = \sum\limits_{j=-\infty}^\infty \lin{y,f_j} f_j(x),\\
&y'(x) = \sum\limits_{j=-\infty}^\infty \lin{y,f_j} f_j'(x) =
\sum\limits_{j=-\infty}^\infty \lin{y,f_j} \frac{ij\pi}{L}f_j(x),\\
&y'(x) = \sum\limits_{j=-\infty}^\infty \lin{y,f_j} f_j''(x) =
\sum\limits_{j=-\infty}^\infty \lin{y,f_j} \frac{i^2j^2\pi^2}{L^2}f_j(x).
\end{align*}
Wir gehen hier davon aus, dass Ableitung und Reihe vertauschbar sind. Setze
nun die Entwicklung in die Differentialgleichung ein,
\begin{align*}
\sum\limits_{j=-\infty}^\infty \left(-\frac{j^2\pi^2}{L^2}a +
\frac{ij\pi}{L}b + 1\right)\lin{y,f_j} = \sum\limits_{j=-\infty}^\infty
\lin{f,f_j}f_j(x).
\end{align*}
Wir wissen bereits, dass wenn die Fourierreihen in jedem $x$ übereinstimmen,
die Koeffizienten gleich sind. Es gilt daher,
\begin{align*}
&\lin{y,f_j} = \underbrace{\frac{1}{-a\frac{j^2\pi^2}{L^2} +
b\frac{ij\pi}{L}+1}}_{:=\alpha_j}\lin{f,f_j},\\
\Rightarrow & y(x) = \sum\limits_{-\infty}^\infty \alpha_j \lin{f,f_j}f_j(x).
\end{align*}
Wir wollen nun die Voraussetzungen an $f$ erarbeiten. Sei $f\in\C^1(\R\to\R)$
$2\pi$-periodisch, dann konvergiert die Fouriersumme von $f$ gleichmäßig. Durch
partielle Integration erhält man, $\lin{f,f_j} \le \frac{c}{j}$. Die Reihe
\begin{align*}
y(x) = \sum\limits_{j=-\infty}^\infty \alpha_j \lin{f,f_j} f_j(x),
\end{align*}
konvergiert gleichmäßig, denn 
\begin{align*}
\abs{\alpha_j \lin{f,f_j}f_j(x)} \le \frac{c}{j^3},
\end{align*}
Weierstrass Kriterium. Die Reihe
\begin{align*}
y'(x) = \sum\limits_{j=-\infty}^\infty \frac{ij\pi}{L}\alpha_j \lin{f,f_j}
f_j(x),
\end{align*}
konvergiert gleichmäßig, denn 
\begin{align*}
\abs{\frac{ij\pi}{L}\alpha_j \lin{f,f_j}f_j(x)} \le \frac{c}{j^2}.
\end{align*}
Nun ist
\begin{align*}
y'' = \frac{1}{a}\left(f - by' - y\right)
\end{align*}
und die rechte Seite konvergiert gleichmäßig, also konvergiert auch $y''$
gleichmäßig.

Somit sind alle Voraussetzungen zur Vertauschung von Ableitung und Reihe
erfüllt und $y\in C^2$ erfüllt die Differentialgleichung.
\begin{align*}
y(x) = \int\limits_{-L}^L \underbrace{\left(\sum\limits_{j=-\infty}^\infty
\alpha_j e^{ij(x-t)} \right)}_{:=G(x,t)}f(t)\dt.
\end{align*}
$G(x,t)$ ist die Greensche Funktion zum Operator,
\begin{align*}
Lu := au'' + bu' + u.
\end{align*}
Die Greensche Funktion ist im Komplexen nicht mehr symmetrisch sondern
hermitesch.\bsphere
\end{bsp}

