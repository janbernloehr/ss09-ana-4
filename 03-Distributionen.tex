\section{Distributionen}

Betrachte die Funktion
\begin{align*}
&f: \R^3\setminus\setd{0} \to \R,\; x\mapsto \frac{1}{\abs{x}},\\
&\nabla f = -\frac{x}{\abs{x}^3}\\
&\Delta f = 0.
\end{align*}
Aus der Physik wissen wir jedoch, dass
\begin{align*}
-\Delta \frac{1}{\abs{x}} = 4\pi \delta_0,
\end{align*}
die \emph{Dirac-Delta-Funktion} ist, d.h.
\begin{align*}
\int_{\R^3}\delta_0(x)\ph(x)\dx = \ph(0).
\end{align*}
Die Delta-Funktion weißt also jeder stetigen Funktion eine Zahl zu.

Im Folgenden wollen wir das Notwendige erarbeiten, um die Funktion zu
beschreiben.
\begin{bemn}[Arbeitsgebiet.]
Erweitere den Raum $C^1(\R^n\to\C)$ zu einem linearen topologischen Raum
$D'(\R^n)$ mit dem Ziel,
\begin{enumerate}[label=(\roman{*})]
  \item $D_j : C^1(\R^n\to\C) \to C^0(\R^n\to\C),\; \ph\mapsto \partial_{x_j}
  \ph$ ist eine stetige Abbildung,
  \item Setze $D_j$ auf $D'(\R^n)$ fort.
  \item Zeige $f: x\mapsto \frac{1}{\abs{x}}$ und $\delta_0\in D'(\R^n)$, sowie
\begin{align*}
-\Delta f = 4\pi \delta_0.
\end{align*}
\end{enumerate}
\end{bemn}

\subsection{Konstruktion des Raums}

\begin{defn}[Notation]
\label{defn:3.1}
Der Träger einer Funktion $\ph$ ist definiert als,
\begin{align*}
\supp\ph = \overline{\setdef{x}{\ph(x)\neq 0}}.
\end{align*}
Der Raum der $C^\infty$-Funktionen mit kompaktem Träger ist
\begin{align*}
C_0^\infty(\R^n) := \setdef{\ph\in C^\infty(\R^n\to\C)}{\supp \ph \text{ ist
kompakt}}.
\end{align*}
Für die Ableitung schreiben wir,
\begin{align*}
\nabla^\alpha \ph :=
\frac{\partial^{\abs{\alpha}}}{\partial_{x_1}^{\alpha_1}\cdots
\partial_{x_n}^{\alpha_n}}.\fishhere
\end{align*}
\end{defn}

\begin{defn}
\label{defn:3.2}
Seien $\ph_j,\ph\in C_0^\infty(\R^n)$. Dann heißt $(\ph_j)$
\emph{$D$-konvergent} gegen $\ph$, falls gilt
\begin{enumerate}[label=(\roman{*})]
  \item $\exists K\subseteq \R^n$ kompakt, so dass $\forall j\in\N_0$ gilt
  $\supp \ph_j \subseteq K$,
  \item $\forall \alpha\in\N_0^n$ gilt $\nabla^\alpha \ph_j(x) \to \nabla^\alpha
  \ph(x)$ gleichmäßig bezüglich $x\in\R^n$.
\end{enumerate}
Schreibe $\ph_j \Dto \ph$. Der lineare Raum
$C_0^\infty(\R^n)$ versehen mit diesen Konvergenzeigenschaften heißt \emph{Raum
der Testfunktionen} $D(\R^n)$.\fishhere
\end{defn}

\begin{prop}
\label{prop:3.3}
$D(\R^n)$ ist vollständig.\fishhere
\end{prop}
\begin{proof}
Sei $(\ph_j)$ Cauchyfolge in $D(\R^n)$, d.h.
\begin{enumerate}[label=(\roman{*})]
  \item $\exists K\subseteq \R^n$ kompakt, so dass $\forall j\in\N_0$ gilt
  $\supp \ph_j \subseteq K$,
  \item $\forall \ep > 0 \forall \alpha\in\N_0^n\exists J_\ep \in\N \forall j,k
  > J_\ep\forall x\in\R^n$.
\begin{align*}
\abs{\nabla^\alpha \ph_j(x) - \nabla^\alpha \ph_k(x)} <\ep.
\end{align*}
\end{enumerate}
D.h. $(\nabla^\alpha \ph_j(x))$ ist Cauchy in $\C$, also existiert
\begin{align*}
\psi_\alpha(x) = \lim\limits_{j\to\infty} \nabla^\alpha \ph_j(x) \in \C.
\end{align*}
Da $\nabla^\alpha \ph_j$ stetig und die Konvergenz bezüglich $x$ gleichmäßig
ist, its $\psi_\alpha$ stetig, mehr noch, alle Ableitungen konvergieren
gleichmäßig, d.h. $\nabla^\alpha \psi_0 = \psi_\alpha$.

Für alle $j\in\N$ ist $\supp\ph_j\subseteq K$ und daher ist auch
$\supp\psi_\alpha\subseteq K$. Somit folgt $\ph_j \Dto \psi_0$.\qedhere
\end{proof}

$D$ ist ein Beispiel für einen vollständigen aber nicht metrisierbaren Raum.

\begin{defn}
\label{defn:3.4}
Eine \emph{Schwartzsche Distribution} ist eine lineare stetige Abbildung
\begin{align*}
T: D(\R^n)\to \C,
\end{align*}
d.h.
\begin{align*}
&T(\alpha \ph + \beta \psi) = \alpha T(\ph) + \beta T(\psi),\\
&\ph_j \Dto \ph \Rightarrow T(\ph_j) \to T(\ph).
\end{align*}
Durch $(T+S)(\ph) = T(\ph) + S(\ph)$ und $(\alpha T)(\ph) = \alpha (T(\ph))$
wird
\begin{align*}
\setdef{T: D(\R^n)\to \C}{T\text{ ist linear und stetig}}
\end{align*}
zu einem linearen Raum $D'(\R^n)$ mit dem Konvergenzbegriff
\begin{align*}
T_n\to T \Leftrightarrow \forall \ph\in C_0^\infty(\R^n) : T_n(\ph) \to
T(\ph).\fishhere
\end{align*}
\end{defn}

\begin{bsp}
\label{bsp:3.5}
\begin{enumerate}[label=\alph{*})]
  \item Diracsche $\delta$-Distribution $\delta_{x_0}$ mit $x_0\in\R^n$ fest,
\begin{align*}
\delta_{x_0}\ph = \ph(x_0).
\end{align*}
Sie ist offenbar linear, denn
\begin{align*}
\delta_{x_0}(\alpha\ph + \beta\psi) = \alpha\ph(x_0) + \beta\psi(x_0) = \alpha
\delta_{x_0}(\ph) + \beta \delta_{x_0}(\psi),
\end{align*}
und stetig denn sei $(\ph_j)$ Folge in $D$ mit $\ph_j \Dto \ph$, so gilt
\begin{align*}
\delta_{x_0}(\ph_j) = \ph_j(x_0) \to \ph(x_0) = \delta_{x_0}(\ph).
\end{align*}
\item Betrachte den Raum der lokalen $\LL^1$-Funktionen,
\begin{align*}
\LL^1_\loc(\R^n) = \setdef{u\in \R^n \to \C \text{ messbar}}{\forall
K\subseteq \R^n\text{ kompakt} : u\in\LL^1(K)}.
\end{align*}
Beachte, dass $u\in\LL^p(K)\Rightarrow u\in\LL^1(K)$.

Zu $u\in \LL^1_{\text{loc}}$ setze
\begin{align*}
T_u(\ph) := \int_{\R^n} u\ph\dmu.
\end{align*}
Die Linearität der Abbildung folgt direkt aus der Linearität des Integrals, zur
Stetigkeit betrachte eine Folge $(\ph_j)$ in $D$ mit $\ph_j \Dto \ph$ und
\begin{align*}
\abs{T_u(\ph_j)-T_u(\ph)} &\le \int_{\R^n} \abs{u}\abs{\ph_j - \ph}\dmu = 
\int_K \abs{u}\abs{\ph_j-\ph}\dmu \\ &\le
\underbrace{\norm{\ph_j - \ph}_\infty}_{\to 0} \underbrace{\int_K
\abs{u}\dmu}_{< \infty} \to 0.\bsphere
\end{align*}
\end{enumerate}
\end{bsp}

\begin{prop}
\label{prop:3.6}
$D'(\R^n)$ ist vollständig.\fishhere
\end{prop}
\begin{proof}
Siehe Walter.\qedhere
\end{proof}

\subsection{Einbettung der klassischen Funktionen}

\begin{prop}
\label{prop:3.7}
Die Abbildung $\LL_\loc^1(\R^n)\ni u\mapsto T_u\in D'(\R^n)$ ist linear,
injektiv und stetig.\fishhere
\end{prop}
\begin{proof}
\textit{Linearität}.
\begin{align*}
T_{\alpha u + \beta v}(\ph)  &= \int_{\R^n} \left(\alpha u + \beta v\right)\ph
\dmu = \alpha\int_{\R^n} u\ph \dmu + \beta\int_{\R^n} v\ph\dmu \\ &= \alpha
T_u(\ph) + \beta T_v(\ph).
\end{align*}
\textit{Injektivität}. Zeige $T_u = T_v \Leftrightarrow u=v$.

``$\Leftarrow$'': Offensichtlich.

``$\Rightarrow$'': Sei $T_u = T_v$, so ist dies aufgrund der Linearität 
äquivalent mit $T_{u-v} = 0$. Wir müssen also lediglich zeigen, dass der Kern
der Abbildung trivial ist.

Sei $T_w(\ph) = 0$, für alle $\ph\in C_0^\infty$.

\begin{enumerate}[label=\arabic{*})]
  \item \textit{Abschneiden}. Sei $\psi_R\in C_0^\infty(\R^n)$ mit $\psi(x) =
  1$ für $\abs{x}\le R$, dann ist
  \begin{align*}
  \forall \ph\in C_0^\infty 0 = T_w(\psi_R \ph) = T_{\psi_R w}(\ph),
  \end{align*}
und $\psi_R w\in \LL^1(\R^n)$.
\item \textit{Approximieren}. Für festes $x\in\R^n$ ist $j_\ep(x-\cdot)\in
C_0^\infty(\R^n)$
\begin{align*}
\Rightarrow &0 = T_{\psi_R w}(j_\ep)(x-\cdot) = \int\limits_{\R^n}
\psi_R(y)w(y)j_\ep(x-y)\dy \\ &= (j_\ep*(\psi_Rw))(x).
\end{align*}
Da $\psi_r w\in \LL^1$ folgt mit \ref{prop:2.22},
\begin{align*}
&\norm{\psi_Rw - j_\ep(\psi_Rw)}_1 \to 0,\quad \text{für }\ep\downarrow 0,\\
\Rightarrow &\psi_R w = 0,\mufu\\
\Rightarrow &w = 0 \mufu\text{ in }\setdef{x}{\abs{x}<}R.
\end{align*}
\item \textit{Zusammenfassen}.
\begin{align*}
\mu\setdef{x\in\R^n}{w(x)\neq 0} &= \mu\left(\bigcup\limits_{j\in\N}
\setdef{x\in\R^n}{\abs{x}\le j\land w(x)\neq 0}\right)
\\ &\le \sum\limits_{j=1}^\infty \mu\setdef{x\in\R^n}{\abs{x}\le j\land w(x)\neq
0} = 0.\qedhere
\end{align*}
\end{enumerate}
\end{proof}

\begin{defn}
\label{defn:3.8}
Identifiziere $u$ und $Tu$, dadurch ist $\LL_\loc^1\subseteq D'(\R^n)$.
$T\in D'(\R^n)$ heißt \emph{regulär}, falls $\exists u\in \LL_\loc^1(\R^n) : T
= T_u$, sonst \emph{singulär}. Falls $T=T_u$ schreibe auch $T_u(\ph) =: (u,\ph)
=: (T_u,\ph)$.
\fishhere
\end{defn}

\begin{bsp}
\label{bsp:3.9}
\begin{enumerate}[label=\arabic{*}.)]
  \item $\delta_{x_0}(\ph) = \ph(x_0)$ ist eine singuläre Distribution.
  \item Sei $S\subseteq \R^n$ eine $k$-dimensionale Mannigfaltigkeit $k\le
  n-1$, so ist
  \begin{align*}
  T\ph := \int_S \ph\dV^{(k)}      
  \end{align*}
eine singuläre Distribution.
\item Sei $n=1$ und
\begin{align*}
T\ph := \lim\limits_{\ep\downarrow 0} \left( \int\limits_{-\infty}^\ep
\frac{\ph(x)}{x}\dx + \int\limits_{\ep}^\infty \frac{\ph(x)}{x}\dx \right)
= \mathrm{CH}\left(\int\limits_{-\infty}^\infty \frac{\ph(x)}{x} \right), 
\end{align*}
der Cauchysche Hauptwert. So ist $T$ eine singuläre Distribution.\bsphere
\end{enumerate}
\end{bsp}

\begin{prop}
\label{prop:3.10}
$C_0^\infty(\R^n)$ ist dicht in $D'(\R^n)$.\fishhere
\end{prop}
\begin{proof}
Sei $T\in D'(\R^n)$ und $\psi_k\in C_0^\infty$ eine Abschneidefunktion mit
\begin{align*}
\psi_k(x) =
\begin{cases}
1, & \abs{x}\le k,\\
0, & \abs{x}\ge k+1,\\
\in[0,1], &\text{sonst}.
\end{cases}
\end{align*}
Setze nun
\begin{align*}
t_k(x) := \psi_k(x) T\left(j_{1/k}(x-\cdot) \right) =
T\left(\psi_k(x)j_{1/k}(x-\cdot) \right).
\end{align*}
Dann gilt
\begin{enumerate}[label=\arabic{*}.)]
  \item $t_k\in C_0^\infty(\R^n)$.
  \item $t_k\to T$ in $D'(\R^n)$, d.h. $\forall\ph\in C_0^\infty$ gilt
\begin{align*}
T_{t_k}\ph \to T\ph.
\end{align*}
\begin{enumerate}[label=\alph{*})]
  \item $\supp t_k \subseteq \supp \psi_k \subseteq K_{k+1}(0)$, also hat $t_k$
  kompakten Träger. Zur Differenzierbarkeit von $t_k$ betrachte
  \begin{align*}
  &\frac{T\left(j_{1/k}(x + he_j-\cdot)
  \right)-T\left(j_{1/k}(x-\cdot) \right)}{h} \\ 
  &\quad = T\left(\frac{j_{1/k}(x +
  he_j-\cdot)-j_{1/k}(x-\cdot) }{h}\right)\\
  &\quad
  \overset{\text{z.Z.}}{\longrightarrow} T\left(\partial_{x_j}
  j_{1/k}(x-\cdot)\right).
  \end{align*} 
Da $T$ stetig ist, genügt es zu zeigen,
\begin{align*}
\frac{j_{1/k}(x + he_j-\cdot)-j_{1/k}(x-\cdot)
  }{h} \longrightarrow \partial_{x_j}
  j_{1/k}(x-\cdot). 
\end{align*}
Dann folgt
\begin{align*}
\nabla^\alpha t_k(x) = T\left(\nabla^\alpha j_{1/k}(x-\cdot) \right),
\end{align*}
insbesondere ist $t_k\in C_0^\infty(\R^n)$.
\item Zur Konvergenz von $T_{t_k}$ betrachte
\begin{align*}
T_{t_k}(\ph) &= \int_{\R^n} t_k(x)\ph(x)\dx\\
&= \int_{\R^n}
T\left(j_{1/k}(x-\cdot)\right)\psi_k(x)\ph(x)
\dx\\
&= \int_{\R^n}T\left(j_{1/k}(x-\cdot)\right)\ph(x)
\dx,\qquad \text{für $k$ groß}.
\end{align*}
interpretiere dies als Riemannsumme und verwende die Stetigkeit von $T$,
\begin{align*}
=T\left(\underbrace{\int_{\R^n} j_{1/k}(x-\cdot)\ph(x)
\dx}_{\Dto \ph(\cdot)}\right) \to T(\ph).\qedhere
\end{align*}
\end{enumerate}
\end{enumerate}
\end{proof}

\subsection{Differentiation}

\begin{lem}
\label{prop:3.11}
Für $u\in C_0^\infty(\R^n)$ und $\alpha\in\N_0^n$ gilt
\begin{align*}
T_{\nabla^\alpha u}(\ph) = (-1)^\alpha T_u(\nabla^\alpha \ph).\fishhere
\end{align*}
\end{lem}
\begin{proof}
Sei $\alpha=e_j$,
\begin{align*}
T_{\partial_{x_j} u}(\ph) = \int\limits_{\R^n} (\partial_{x_j} u)\ph\dx
\end{align*}
nun ist $\supp\ph\subseteq [-R,R]^n$, es gilt also
\begin{align*}
&= \int\limits_{[-R,R]^{n-1}}\int\limits_{[-R,R]} (\partial_{x_j} u)\ph
\dx_{j}\dx_{j+1}\cdots\dx_{n}\\
&= \int\limits_{[-R,R]^{n-1}}\left[\underbrace{u\ph\bigg|_{-R}^R}_{=0}
-\int\limits_{-R}^R u\partial_{x_j} \ph\dx_j\right]\dx_{j+1}\cdots\dx_{n}\\
&= -T_u(\partial_{x_j}\ph ).\qedhere
\end{align*}
\end{proof}

\begin{prop}
\label{prop:3.12}
Die Abbildung
\begin{align*}
\nabla^\alpha : C_0^\infty(\R^n)\to C_0^\infty(\R^n)
\end{align*}
ist stetig bezüglich der Topologie auf $D'(\R^n)$.\fishhere
\end{prop}
\begin{proof}
Zu zeigen ist,
\begin{align*}
u_j\Dto u \Rightarrow \nabla^\alpha u_j \to \nabla^\alpha u.
\end{align*}
Sei also $(u_j)$ Folge in $C_0^\infty(\R^n)$ und $u_j\Dto u$, d.h.
$T_{u_j}\to Tu$.
\begin{align*}
\Rightarrow T_{\nabla^\alpha u_j}(\ph) \overset{\ref{prop:3.11}}{=}
(-1)^{\abs{\alpha}}
T_{u_j}(\underbrace{\nabla^\alpha \ph}_{\in C_0^\infty(\R^n)}) 
\end{align*}
D.h. $\nabla^\alpha u_j \to \nabla^\alpha u$ in $D'(\R^n)$.\qedhere
\end{proof}

\begin{prop}
\label{prop:3.13}
Für $\alpha\in\N_0^n$ ist die Abbildung $\nabla^\alpha$ eindeutig fortsetzbar
zu einer stetigen Abbildung
\begin{align*}
\D^\alpha : D'(\R^n)\to D'(\R^n).
\end{align*}
Für $\ph\in D'(\R^n)$ gilt,
\begin{align*}
\D^\alpha T(\ph) = (-1)^{\abs{\alpha}}T(\nabla^\alpha \ph)\tag{*}.
\end{align*}
$\D^\alpha$ heißt \emph{Distributionenableitung} oder \emph{schwache
Ableitung}.\fishhere
\end{prop}
\begin{proof}
\textit{Wohldefiniertheit}. Durch (*) wird eine Abbildung $\D^\alpha T\in
D'(\R^n)$ definiert.

\textit{Linearität}. Seien $\ph,\psi\in D(\R^n)$ und $a,b\in\C$,
\begin{align*}
\D^\alpha T(a\ph + b\psi) &= (-1)^{\abs{\alpha}} T(a\nabla^\alpha \ph +
b\nabla^\alpha \psi) \\ &= a (-1)^{\abs{\alpha}} T(\nabla^\alpha \ph) +
b(-1)^{\abs{\alpha}} T(\nabla^\alpha \psi) \\ &= a\D^\alpha T(\ph) +
b\D^\alpha(\psi)
\end{align*}

\textit{Stetigkeit}. Sei $\ph_n\Dto \ph$, dann gilt für alle $\alpha\in\N_0^n$
$\nabla^\alpha \ph_n \Dto \nabla^\alpha \ph$ und daher,
\begin{align*}
\D^\alpha(\ph_n) = (-1)^{\abs{\alpha}}
T(\nabla^\alpha \ph_n) \overset{T\text{
stet.}}{\to} (-1)^{\abs{\alpha}} T(\nabla^\alpha \ph) = \D^\alpha T(\ph).
\end{align*}

\textit{Eindeutigkeit}. Aufgrund der Stetigkeit ist die Fortsetzung
eindeutig.\qedhere
\end{proof}

\begin{bem}[Bemerkungen.]
\label{bem:3.14}
\begin{enumerate}[label=\arabic{*}.)]
  \item Jede Distribution ist beliebig oft differenzierbar.
  \item Es gilt der Satz von H.A. Schwarz\footnote{Hermann Amandus Schwarz (*
  25. Januar 1843 in Hermsdorf, Schlesien; † 30. November 1921 in Berlin)}.
\begin{align*}
\forall \alpha,\beta \in \N_0^n : \D^\alpha (\D^\beta T) = \D^\beta(\D^\alpha T)
= \D^{\alpha+\beta}T.
\end{align*}
\begin{proof}
Seien $T\in D'(\R^n)$ und $\ph\in D(\R^n)$,
\begin{align*}
\D^\alpha (\D^\beta T)(\ph) &= (-1)^{\abs{\alpha}}(\D^\beta T)(\nabla^\alpha
\ph) = (-1)^{\abs{\alpha}+\abs{\beta}}T(\nabla^\beta(\nabla^\alpha\ph))\\
&= (-1^{\abs{\alpha+\beta}} T(\nabla^{\alpha+\beta}\ph) =
\D^{\alpha+\beta}T(\ph).\qedhere\maphere
\end{align*}
\end{proof}
\end{enumerate}
\end{bem}

\begin{bsp}
\label{bps:3.15}
\begin{enumerate}[label=\arabic{*}.)]
  \item Setze $f(x) = \begin{cases} x, & x > 0,\\ 0, & x\le 0,\end{cases}$ dann
  gilt für die Distributionenableitung,
  \begin{align*}
  \D f = h,\qquad h(x) = \begin{cases}
                         1, & x> 0,\\
                         0, & x< 0.
                         \end{cases}
  \end{align*}
$h$ ist die \emph{Heaviside-Funktion}\footnote{Oliver Heaviside (* 18. Mai 1850
in London; † 3. Februar 1925 in Homefield bei Torquay) war ein britischer
Mathematiker und Physiker.}. Der Wert an der Stelle $x=0$ ist ohne Bedeutung
da $f$ eine reguläre Distribution ist, d.h. $h\in\LL_\loc^1$.
\begin{proof}
\begin{align*}
\D T_f(\ph) &= -T_f(\ph' = -\int\limits_0^\infty x\ph'(x)\dx = 
-\underbrace{x\ph(x)\bigg|_{0}^\infty}_{=0} + \int\limits_{0}^\infty \ph(x)\dx\\
&= T_h(\ph).\qedhere
\end{align*}
\end{proof}
Für die zweite Ableitung ergibt sich, $\D^2 f = \D h = \delta_0$.
\begin{proof}
$\D T_h(\ph) = -T_h(\ph') = -\int\limits_0^\infty 1 \ph'(x)\dx =
\delta_0(\ph)$.\qedhere
\end{proof}
Die $k$-te Ableitung ist $\D^k f = (-1)^k\ph^{(k-2)}(0)$ für $k\ge2$.
\item Sei $f: \R\to\C$, $f\in C^1((-\infty,x_0))$ und $f\in C^1((x_0,\infty))$
und $f(x_0+0)$ sowie $f(x_0-0)$ existieren. Dann gilt
\begin{align*}
\D f = f' + \left(f(x_0 +0) - f(x_0-0)\right)\delta_{x_0}.
\end{align*}
$f'$ bezeichnet hier die klassische Abbleitung von $f$. Der Wert bei $x_0$
spielt wieder keine Rolle.

Wir erkennen das folgende Schema,
\begin{align*}
&\text{Knick} &&\overset{\text{Ableitung}}{\longrightarrow} &&\text{Sprung},\\
&\text{Endlicher Sprung} &&\overset{\text{Ableitung}}{\longrightarrow}
&&\delta\text{-Distribution}.
\end{align*}
\item $\D^\alpha \delta_0(\ph) = (-1)^\alpha \delta_0(\nabla^\alpha \ph$.
\item Setze $f(x) = \begin{cases}\frac{1}{\abs{x}},&\text{für
}x\in\R^3\setminus\setd{0}\\ 0, & x = 0\end{cases}$.
\begin{enumerate}[label=\alph{*})]
  \item $f$ ist stetig für $x\neq 0$ und es gilt,
\begin{align*}
\int_{K_1(0)} f(x)\dx \overset{\text{Kugelkoord.}}{=} \int\limits_{r=0}^1
\frac{1}{r} r^2 \dr \int\limits_{\ph=0}^{2\pi}\int\limits_{\th=0}^\pi \sin\th
\dth\dph < \infty.
\end{align*}
Also ist $f\in\LL_\loc^1$ und daher die Distribution $f\in D'(\R^n)$.
\item 
\begin{align*}
(\Delta f)(\ph) &= \Delta T_f (\ph) = \partial_{x_1}^2 T_f(\ph) + \ldots +
\partial_{x_3}^2 T_f(\ph)\\
&= (-1)^2 T_f(\partial_{x_1}^2 \ph+ \ldots +\partial_{x_3}^2 \ph) = T_f(\Delta
\ph)\\
&= \int\limits_{\R^3} \frac{1}{\abs{x}}\Delta \ph(x)\dx
\end{align*}
Wähle $R$ so, dass $\supp\ph \subseteq K_{R}(0)$ und verwende $\Delta
\frac{1}{\abs{x}}  = 0$,
\begin{align*}
\ldots &= \lim\limits_{\ep\downarrow 0} \int\limits_{\ep < \abs{x} < R}
\frac{1}{\abs{x}}\Delta \ph(x) - \Delta\frac{1}{\abs{x}}\ph(x)\dx\\
&\overset{\text{Green}}{=}
\lim\limits_{\ep\downarrow 0} \int\limits_{\abs{x}=\ep,R}
\frac{1}{\abs{x}}n_0(x) \nabla\ph(x) -
n_0(x)\nabla\frac{1}{\abs{x}}\ph(x)\dV^{(2)}
\end{align*}
Da $\supp\ph \subseteq K_{R}(0)$, verschwindet der Ausdruck auf $\abs{x}=R$,
\begin{align*}
\ldots = \lim\limits_{\ep\downarrow 0} \int\limits_{\abs{x}=\ep}
\frac{1}{\abs{x}}n_0(x) \nabla\ph(x) -
n_0(x)\nabla\frac{1}{\abs{x}}\ph(x)\dV^{(2)}
\end{align*}
%TODO: Skizze Normalenvektor
Nun ist $n_0(x) = -\frac{x}{\abs{x}}$ für $\abs{x}=\ep$. Für den ersten
Summanden ergibt sich,
\begin{align*}
&\abs{\int\limits_{\abs{x}=\ep}
\frac{1}{\abs{x}}n_0(x) \nabla\ph(x) \dV^{(2)}} \le
\int\limits_{\abs{x}=\ep} \frac{\abs{x}\abs{\nabla \ph}}{\abs{x}^2}\dV^{(2)}\\
&\le \norm{\nabla \ph} \int\limits_{\abs{x}=\ep} \frac{1}{\abs{x}}\dV^{(2)} =
\norm{\nabla \ph}\frac{4\pi\ep^2}{\ep} \to 0.
\end{align*}
Für den zweiten Summanden verwende $\nabla \frac{1}{\abs{x}} =
-\frac{x}{\abs{x}^3}$,
\begin{align*}
&-\int\limits_{\abs{x}=\ep}
n_0(x)\nabla\frac{1}{\abs{x}}\ph(x)\dV^{(2)}
= -\int\limits_{\abs{x}=\ep}
\frac{x}{\abs{x}}\frac{x}{\abs{x}^3}\ph(x)\dV^{(2)}\\
&-\int\limits_{\abs{x}=\ep}
\frac{1}{\abs{x}^2}\ph(x)\dV^{(2)}\\
&= -\frac{1}{\ep^2}\left[ 
\underbrace{\int\limits_{\abs{x}=\ep}
\ph(x)-\ph(0)\dV^{(2)}}_{\to 0}
+ \int\limits_{\abs{x}=\ep}
\ph(0)\dV^{(2)}
 \right]\\
 &\to -4\pi \ph(0)
\end{align*}
Somit ist $\Delta T_f(\ph) = -4\pi\ph(0) = -4\pi \delta_0(\ph)$, d.h. $-\Delta
f = 4\pi\delta_0$.\bsphere
\end{enumerate}
\end{enumerate}
\end{bsp}

\begin{prop}
\label{prop:3.16}
Sei $(T_j)$ Folge in $D'(\R^n)$.
\begin{enumerate}[label=(\roman{*})]
  \item\label{prop:3.16:1} Sei $T_j\to T$, so gilt
\begin{align*}
\forall \alpha\in\N_0^n : \D^\alpha T_j \to \D^\alpha T.
\end{align*}
\item\label{prop:3.16:2} Sei $\sum_j T_j = T$, so gilt
\begin{align*}
\forall \alpha\in\N_0^n : \D^\alpha \sum_j T_j \to \D^\alpha T.\fishhere
\end{align*}
\end{enumerate}
\end{prop}
\begin{proof}
``\ref{prop:3.16:1}'': $T_j\to T$ heißt,
\begin{align*}
\forall \ph\in C_0^\infty(\R^n) : T_j(\ph)\to T(\ph).
\end{align*}
Sei also $\ph\in C_0^\infty(\R^n)$, so gilt
\begin{align*}
\D^\alpha T_j(\ph) = (-1)^{\abs{\alpha}}T_j(\nabla^\alpha \ph) =
\to  (-1)^{\abs{\alpha}}T(\nabla^\alpha \ph) = \D^\alpha T(\ph).
\end{align*}
``\ref{prop:3.16:2}'': Folgt aus \ref{prop:3.16:1} mit
\begin{align*}
\D^\alpha \sum_j T_j = \D^\alpha \lim\limits_{J\to\infty} \sum\limits_{j=1}^J
T_j
\overset{\ref{prop:3.16:1}}{=}
 \lim\limits_{J\to\infty} \D^\alpha  \sum\limits_{j=1}^J T_j = \D^\alpha
 T.\qedhere
\end{align*}
\end{proof}

\begin{bsp}
\label{bsp:3.17}
\begin{enumerate}[label=\arabic{*}.)]
  \item Betrachte
\begin{align*}
f_j(x) = 
\begin{cases}
0, & x < 0,\\
x^j, & x \le 1,\\
1, & x > 1,
\end{cases}
\overset{\text{punktw.}}{\to}
f(x) =
\begin{cases}
0, & x < 1,\\
1, & x\ge 1.
\end{cases}
\end{align*}
\begin{enumerate}[label=\alph{*})]
  \item $f_j\to f$ in $D'(\R)$.
\begin{align*}
T_{f_j}(\ph) = \int\limits_1^\infty \ph(x)\dx + \int\limits_0^1
x^j\ph(x)\dx
\end{align*}
Nun gilt
\begin{align*}
\abs{\int\limits_0^1
x^j\ph(x)\dx} \le \norm{\ph}_\infty \int\limits_0^1 x^j\dx =
\frac{\norm{\ph}_\infty}{j+1} \to 0.
\end{align*}
\item $\D f_j \to \D f = \delta_1$.
\begin{align*}
(\D f_j)(\ph) = -T_{f_j}(\ph') = -\underbrace{\int\limits_0^1
x^j\ph'(x)\dx}_{\to 0} + \underbrace{\int\limits_{1}^\infty
\ph'(x)\dx}_{=\ph(1)} \to \delta_1(\ph).
\end{align*} 
\end{enumerate}
\item Betrachte $f(x) = \frac{\pi}{4}\abs{x}$ für $-\pi\le \pi$,
$2\pi$-periodisch fortgesetzt.
\begin{figure}[H]
\centering
\begin{pspicture}(-4,-1)(4,2)
\psaxes[labels=none,ticks=none]{->}%
 (0,0)(-3.5,-0.5)(3.5,1.5)%
 [\color{gdarkgray}$x$,-90][\color{gdarkgray}$f(x)$,0]

\psplot[linewidth=1.2pt,%
	     linecolor=darkblue,%
	     algebraic=true]%
	     {-3}{-2}{-x-2}
\psplot[linewidth=1.2pt,%
	     linecolor=darkblue,%
	     algebraic=true]%
	     {-2}{-1}{x+2}
\psplot[linewidth=1.2pt,%
	     linecolor=darkblue,%
	     algebraic=true]%
	     {-1}{0}{-x}
\psplot[linewidth=1.2pt,%
	     linecolor=darkblue,%
	     algebraic=true]%
	     {0}{1}{x}
 \psplot[linewidth=1.2pt,%
	     linecolor=darkblue,%
	     algebraic=true]%
	     {1}{2}{2-x}
\psplot[linewidth=1.2pt,%
	     linecolor=darkblue,%
	     algebraic=true]%
	     {2}{3}{x-2}

 \psxTick(-2){\color{gdarkgray}-2\pi}
 \psxTick(-1){\color{gdarkgray}-\pi} 
 \psxTick(1){\color{gdarkgray}\pi}
 \psxTick(2){\color{gdarkgray}2\pi}
 \psyTick(1){\color{gdarkgray}\frac{\pi}{4}}
\end{pspicture} 
  \caption{Graph der Funktion $f$.}
\end{figure}

Die Fourierreihe von $f$ ist gegeben durch,
\begin{align*}
g_n(x) = \frac{\pi^2}{8} - \left(\cos x + \frac{\cos 3x}{x} + \frac{\cos 5x}{x}
+  \ldots\right).
\end{align*}
$f$ ist hölderstetig, d.h. $g_n\unito f$ glm.
\begin{align*}
T_{g_n}(\ph) &= \int_\R g_n(x)\ph(x)\dx \to
\lim\limits_{n\to\infty} \int_\R g_n(x)\ph(x)\dx\\
&\overset{\text{glm. konv.}}{=} \int_\R f(x)\ph(x)\dx = T_f(\ph).
\end{align*}
D.h. $g_n\to f$ in $D'(\R)$. Mit dem erweiterten Dini Kriterium folgt,
\begin{align*}
\D g = \sin x + \frac{\sin 3x}{3} + \frac{\sin 5x}{5} + \ldots
= 
\begin{cases}
\frac{\pi}{4}, & 0< x < \pi,\\
-\frac{\pi}{4}, & -\pi < x < 0,\\
0, & \text{sonst}.
\end{cases}
\end{align*}
Mit dem soeben gezeigten gilt,
\begin{align*}
\D g = \D f = 
\begin{cases}
\frac{\pi}{4}, & 0< x < \pi,\\
-\frac{\pi}{4}, & -\pi < x < 0.
\end{cases}
\end{align*}
Für die zweite Ableitung gilt,
\begin{align*}
\D^2 g= \cos x + \cos 3x + \cos 5x + \ldots
\end{align*}
offensichtlich haben wir keine Konvergenz im Definitionsbereich. Im
Distributionensinn lässt sich die Ableitung jedoch darstellen,
\begin{align*}
\D^2 g = \D^2 f = \frac{\pi}{2} \sum\limits_{j=-\infty}^\infty (-1)^j
\delta_{j\pi}.\bsphere
\end{align*}
\end{enumerate}
\end{bsp}

\subsubsection{Multiplikation von Distributionen}

\begin{prop}[Definition/Satz]
\label{prop:3.18}
Sei $a\in C^\infty(\R^n\to\C)$
\begin{enumerate}[label=\arabic{*}.)]
  \item\label{prop:3.18:1} Für $T\in D'(\R^n)$ ist $a\cdot T$ definiert durch,
\begin{align*}
a\cdot T(\ph) := T(a\ph).
\end{align*}
\item Für $u\in\LL_\loc^1(\R^n)$ gilt $a T_u = T_{au}$, d.h. auf der Menge der
regulären Distributionen ist die Multiplikation wohlbekannt.
\item Die Abbildung $T\mapsto a\cdot T$ ist stetig, d.h. die Definition in
\ref{prop:3.18:1} ist die stetige Fortsetzung der Multiplikation.\fishhere
\end{enumerate}
\end{prop}
\begin{proof}
\begin{enumerate}[label=\arabic{*}.)]
  \item Zu zeigen ist, dass $a\cdot T$ wieder eine 
 Distribution ist.
 
 \textit{Linearität.} Ist offensichtlich.
  
  \text{Stetigkeit}. Sei $\ph_j\Dto\ph$, dann ist auch $a\ph_j\Dto a\ph$ und
  daher,
\begin{align*}
aT(\ph_J) = T(a\ph_j) \overset{T\text{ stet.}}{=} T(a\ph) = aT(\ph).
\end{align*}
\item $a\cdot T_u(\ph) = T_u(a\ph) = \int_{\R^n} u(x)a(x)\ph(x)\dx =
T_{ua}(\ph)$.
\item Sei $T_j\to T$ in $D'(\R^n)$, so gilt,
\begin{align*}
a\cdot T_j(\ph) = T_j(a\ph) \to T(a\ph) = a\cdot T(\ph).\qedhere
\end{align*}
\end{enumerate}
\end{proof}

\begin{prop}[Leibnitzregel]
\label{prop:3.19}
$\forall \alpha\in\N_0^n : D^\alpha(a\cdot T) = \sum\limits_{\beta\le \alpha}
\binom{\alpha}{\beta} \left(\nabla^{\alpha-\beta} a\right)\D^\beta T$.\fishhere
\end{prop}
\begin{proof}
Zu $T\in D'(\R^n)$ sei $(\ph_j)\in C_0^\infty(\R^n)$ mit $\ph_j\to T$.
Mit \ref{prop:3.18} folgt, $a\ph_j \to aT$. Aufgrund der Stetigkeit der
Distributionenableitung gilt nun,
\begin{align*}
&\D^\alpha (a\ph_j) \to \D^\alpha(a\ph)\\ \Rightarrow &
\sum\limits_{\beta\le\alpha} \binom{\alpha}{\beta} \left(\nabla^{\alpha-\beta} a
\right)\nabla^\beta \ph_j \to \sum\limits_{\beta\le\alpha}
\binom{\alpha}{\beta} \left(\nabla^{\beta-\alpha} a \right)\D^\beta \ph.\qedhere
\end{align*}
\end{proof}

\subsection{Lokales Verhalten}

\begin{defn}
\label{defn:3.20}
\begin{enumerate}[label=\arabic{*}.)]
  \item Für $M\subseteq \R^n$ sei,
\begin{align*}
C_0^\infty(M) := \setdef{\ph\in C^\infty(M\to C)}{\supp \ph \text{ kompakt und
}\supp\ph \subseteq M}.
\end{align*}
\item Für $O\subseteq \R^n$ offen und $\ph\in C_0^\infty(O)$ sei
\begin{align*}
\tilde{\ph}(x) := 
\begin{cases}
\ph(x), & x\in O,\\
0, &\text{sonst}.
\end{cases}
\end{align*}
Es gilt $\tilde{\ph}\in C_0^\infty(\R^n)$.
\item Seien $T,S\in D'(\R^n)$, $O\subseteq \R^n$ offen. Dann heißt $T=S$ in
$O$, falls,
\begin{align*}
\forall \ph\in C_0^\infty(O) : T(\tilde{\ph}) = S(\tilde{\ph}).\fishhere
\end{align*}
\end{enumerate}
\end{defn}

\begin{bsp}
\label{bsp:3.21}
\begin{enumerate}[label=\arabic{*}.)]
  \item Seien $u,v\in\LL_\loc^1(\R^n)$, $u=v$ $\fu$ in $O$. Für $\ph\in
  C_0^\infty(O)$ gilt dann,
\begin{align*}
T_u(\tilde{\ph}) - T_v(\tilde{\ph}) = T_{u-v}(\tilde{\ph}) = \int\limits_{\R^n}
(u-v)\tilde{\ph}\dmu = \int\limits_O (u-v)\tilde{\ph}\dmu.
\end{align*}
D.h. $T_u = T_v$ in $O$.
\item Sei $O:= \R^n\setminus\setd{0}$, dann ist $\delta_0 = 0$ in $O$, d.h.
\begin{align*}
\ph\in C_0^\infty(O) \Rightarrow \delta_0(\tilde{\ph}) = \tilde{\ph}(0) =
0.\bsphere
\end{align*}
\end{enumerate}
\end{bsp}

\begin{prop}[Ableitung ist lokale Opperation]
\label{prop:3.22}
Wenn $T=S$ in $O$, dann gilt
\begin{align*}
\forall \alpha\in\N_0^n : \D^\alpha T = \D^\alpha
S \text{ in }O.\fishhere
\end{align*}
\end{prop}
\begin{proof}
Offensichtlich ist $\nabla^\alpha \tilde{\ph} = \widetilde{\nabla^\alpha \ph}$
und daher,
\begin{align*}
\D^\alpha T(\tilde{\ph}) &= (-1)^{\abs{\alpha}} T(\nabla^\alpha\tph) =
(-1)^{\abs{\alpha}} T\left(\widetilde{\nabla^\alpha \ph} \right) =
(-1)^{\abs{\alpha}} S\left(\widetilde{\nabla^\alpha \ph} \right) \\ &= \D^\alpha
S(\tph).\qedhere
\end{align*}
\end{proof}

\begin{prop}
\label{prop:3.23}
Sei $T\in D'(\R^n)$, $G:= \bigcup\setdef{O\in\R^n\text{ offen}}{T = 0\text{
in } O}$. Dann ist $T$ offen und $T=0$ in $G$.\fishhere
\end{prop}
\begin{proof}
Die Offenheit von $G$ ist klar. Sei $\ph\in C_0^\infty(G)$. Da der Träger von
$\ph$ kompakt ist existiert eine endliche Überdeckung von $\supp\ph$,
\begin{align*}
\supp\ph \subseteq \bigcup\limits_{j=1}^J O_j,\quad T=0\text{ in }O_j.
\end{align*}
Verwende nun eine Zerlegung der Eins: $\exists \psi_j \in C_0^\infty(O_j) :
\sum\limits_{j=1}^J \psi_j = 1$ auf $\supp\ph$.
\begin{align*}
\Rightarrow \tph = \sum\limits_{j=1}^J \tph\psi_j \Rightarrow
T\tph = \sum\limits_{j=1}^J T\left(\widetilde{\ph\psi_j}\right) = 0.\qedhere
\end{align*}
\end{proof}

\begin{defn}
\label{defn:3.24}
Für $T\in D'(\R^n)$ ist
\begin{align*}
\supp T:= \R^n\setminus \bigcup \setdef{O\subseteq
\R^n\text{ offen}}{T=0\text{ in }O}
\end{align*}
der Träger von $T$. $x\in \supp T$ heißt \emph{wesentlicher Punkt} von $T$.
Falls $\supp T$ kompakt, heißt $T$ \emph{finit}.\fishhere
\end{defn}

\begin{bsp}
\label{bsp:3.25}
\begin{enumerate}[label=\arabic{*}.)]
  \item $\supp \delta_0 = \setd{0}$.
  \item $\ph\in C_0^\infty(\R^n) \Rightarrow \supp T_\ph = \supp\ph$.
  \item Sei $u\in\LL_\loc^1(\R^n)$.
  \begin{align*}
  \supp T_u &=
  \R^n\setminus\setdef{O\subseteq\R^n\text{ offen}}{u=0\mufu \text{ auf } O} \\
  &= \R^n\setminus\setdef{x\in\R^n}{\exists U(x)  : u=0\mufu \text{ in }
  U(x)}.\bsphere
  \end{align*}
\end{enumerate}
\end{bsp}

\begin{prop}
\label{prop:3.26}
Sei $T\in D'(\R^n)$, $K\subseteq\R^n$ kompakt. Dann gilt
\begin{align*}
\exists k\in\N \exists c > 0 \forall\ph\in C_0^\infty(\R^n) : \supp
\ph\subseteq K\Rightarrow \abs{T(\ph)} \le c \max\limits_{\abs{\alpha}\le k}
\norm{\nabla^\alpha\ph}_\infty.\fishhere
\end{align*}
\end{prop}
\begin{proof}
Angenommen die Aussage wäre falsch, d.h.
\begin{align*}
\forall k\in\N \forall c > 0 \exists \ph\in C_0^\infty(\R^n) : \supp\ph
\subseteq K \land \abs{T(\ph)} > c\max\limits_{\abs{\alpha}\le k}
\norm{\nabla^\alpha \ph}_\infty.
\end{align*}
Wähle $c=k$, dann gibt es ein $\ph_k\in C_0^\infty(\R^n)$ mit
$\supp\ph_k\subseteq K$ und
\begin{align*}
\abs{T(\ph_k)} > k\max\limits_{\abs{\alpha}=k} \norm{\nabla^\alpha
\ph_k}_\infty.
\end{align*}
Setze $\psi_k = \frac{1}{k\max\norm{\nabla^\alpha \ph_k}_\infty}\ph_k$, dann
ist $\psi_k\in C_0^\infty(\R^n)$, $\psi_k\subseteq K$ und
\begin{align*}
\max\limits_{\abs{\alpha}\le k} \norm{\nabla^\alpha \psi_k}_\infty = \frac{1}{k}
\to 0.
\end{align*}
Insbesondere geht für festes $\alpha$, $\nabla^\alpha \psi_k \unito 0$ glm. auf
$\R^n$, d.h. $\psi_k \Dto 0$ und daher $T(\psi_k)\to 0$. Aber $T(\psi_k) >
1$.\dipper\qedhere
\end{proof}

\begin{cor}
\label{prop:3.27}
Ist $T$ finit, so gilt \ref{prop:3.26} ohne die Einschränkung $\supp
\ph\subseteq K$.\fishhere
\end{cor}
\begin{proof}
Wähle $\psi\in C_0^\infty(\R^n)$ mit $\psi = 1$ auf $\supp T$. Dann ist
\begin{align*}
T(\ph) = T(\psi\ph) + T(\underbrace{(1-\psi)\ph}_{\in
C_0^\infty(\R^n\setminus\supp T))}) = T(\psi\ph).
\end{align*}
Wende nun \ref{prop:3.26} an mit $K=\supp\psi$, $\supp\ph\psi\subseteq
\supp\psi = K$, dann folgt
\begin{align*}
\abs{T(\ph)} \le c\max\limits_{\abs{\alpha}\le k} \norm{\nabla^\alpha
(\psi\ph)}_\infty \le c \max\limits_{\abs{\alpha}\le k} \norm{\nabla^\alpha
\ph}_\infty.\qedhere
\end{align*}
\end{proof}

\begin{defn}
\label{defn:3.28}
$T\in D'(\R^n)$ heißt von \emph{endlicher Ordnung}, falls \ref{prop:3.26} ohne
die Einschränkung $\supp\ph\subseteq K$ gilt.
Das kleinstmögliche $k\in\N$ heißt \emph{Ordnung} von $T$.\fishhere
\end{defn}

\begin{bsp}
\label{bsp:3.29}
\begin{enumerate}[label=\arabic{*}.)]
  \item $\abs{\delta_{x_0}(\ph)} = \abs{\ph(x_0)} \le \norm{\ph}_\infty$. D.h.
  die Ordnung ist $0$.
  \item Sei $u\in\LL_\loc^1(\R^n)$, dann ist
\begin{align*}
\abs{T_u(\ph)} \le \int\limits_{\R^n} \abs{u}\abs{\ph}\dmu \le
\norm{\ph}_\infty \int\limits_{\R^n} \abs{u}\dmu = \norm{\ph}_\infty\norm{u}_1.
\end{align*}
\item In $\R$ ist für
\begin{align*}
T = \sum\limits_{j=0}^J c_j D^j\delta_0,
\end{align*}
die Ordnung $J$, falls $c_J\neq 0$.
\item Betrachte $u: \R\to\R$ mit $u(x) = x$. Setze $\ph_k = j_1(\cdot-k)$, dann
folgt
\begin{align*}
\abs{T_u(\ph_k)} = \abs{\int\limits_\R j_1(x-k)x\dx} =
\abs{\int\limits_{k-1}^{k+1} j_1(x-k)x\dx } > k-1 \to \infty.
\end{align*}
Das heißt,
\begin{align*}
\norm{\nabla^\alpha \ph_k}_\infty = \norm{\nabla^\alpha j_1(\cdot-k)}_\infty =
\norm{\nabla^\alpha j_1}_\infty = \const,
\end{align*}
unabhängig von $k$. D.h. die Ordnung von $T_u$ ist $\infty$.\bsphere
\end{enumerate}
\end{bsp}

\begin{bem}
\label{bem:3.30}
Sei $T\in D'(\R^n)$ und $O\in\R^n$ offen und beschränkt. Dann existiert ein
$\alpha\in \N_0$ und ein $u\in C(\R^n\to\C)$ mit $T = D^\alpha u$.\maphere
\end{bem}

\subsection{Faltung}

\begin{bemn}[Erinnerung.]
Die Faltung von zwei Funktionen $f,g$ ist definiert als,
\begin{align*}
f*g := \int\limits_{\R^n} f(x-y)g(y)\dy.
\end{align*}
Sind $f,g\in\LL^1$ folgt mit Fubini, dass $f*g\in\LL^1$. Ist aber $f=g=1$, so
ist $f*g$ nicht mehr definiert.

\textit{Abhilfe}. Definiere $f*g$ nur für solche $f,g$ für die bei festem $z$,
\begin{align*}
&\setdef{(z,y)\in\R^{2n}}{f(z-y) \neq 0\land g(y)\neq 0}\text{beschränkt}\\
\Leftrightarrow &
\setdef{(x,y)\in\R^{2n}}{f(x) \neq 0\land g(y)\neq 0\land
x=z-y}\text{beschränkt}\\ \Leftrightarrow &
\supp f\times \supp g \cap
\setdef{(x,y)\in\R^{2n}}{x=z-y}\text{beschränkt}\maphere
\end{align*}
\end{bemn}

\begin{defn}
\label{defn:3.31}
$(T,S)\in D'(\R^n)^2$ erfüllt die \emph{Streifenbedingung}, falls
\begin{align*}
\forall a> 0 : \supp T \times \supp S \cap \sigma_a\text{ beschränkt},
\end{align*}
wobei $\sigma_a := \setdef{(x,y)\in\R^{2n}}{\abs{x+y}\le a}$.\fishhere
\end{defn}

\begin{bsp}
\label{bsp:3.32}
$\supp T$ beschränkt $\Rightarrow$ $(T,S)$ erfüllt die Streifenbedingung.
%TODO: Bild Streifenbedingung \supp T\times \R^n.
\end{bsp}

\begin{bem}
\label{bem:3.33}
Seien $f,g,\ph\in C_0^\infty(\R^n)$, dann ist
\begin{align*}
T(\ph) &= \int_{\R^n} (f*g)(x)\ph(x)\dx
= \int_{\R^n}\int_{\R^n} f(x-y)g(y)\ph(x)\dy\dx\\
&= \int_{\R^n}\int_{\R^n} f(x-y)g(y)\ph(x)\dx\dy\\
&\overset{x-y = z}{=}
\int_{\R^n}\int_{\R^n} f(z)\ph(z+y)g(y)\diffd(z,y)
\end{align*}
Wähle $R$ so, dass $\supp\ph\in K_R(0)$. Dann ist der Integrand höchstens für
$\abs{z+y}<R$ von Null verschieden und $f(z)g(y)\neq 0$ höchstens in $\supp
f\times\supp g$.\maphere
\end{bem}

\begin{defn}
\label{defn:3.34}
$(T,S)\in D'(\R^n)^2$ erfülle die Streifenbedingung, dann ist $T*S$ definiert
durch,
\begin{align*}
(T*S)(\ph) := T_x\left(S_y(\ph(x+y)) \right) = \left(T_x \times
S_y\right)(\ph(x+y)),
\end{align*}
für $\ph\in C_0^\infty(\R^n)$.\fishhere
\end{defn}
\begin{bemn}[Probleme.]
Was ist $T\times S$? $\tilde{\ph}: (x,y)\mapsto \ph(x+y)$ hat stets
unbeschränkten Träger.
\end{bemn}

\begin{lem}
\label{prop:3.35}
Sei $S\in D'(\R^n)$, $\ph\in C_0^\infty(\R^{n+m})$. Setze
\begin{align*}
\psi(x) := S(\ph(x,\cdot)),\qquad\text{für }x\in \R^n. 
\end{align*}
Dann gilt
\begin{enumerate}[label=(\roman{*})]
  \item\label{prop:3.35:1} $\nabla^\alpha\psi(x) = S\left(\nabla^\alpha
  \ph(x,\cdot)\right)$.
  \item $\psi\in C_0^\infty(\R^n)$.
  \item $\ph_j\Dto 0\Rightarrow \psi_j \Dto 0$.
\end{enumerate}
\end{lem}
\begin{proof}
\begin{enumerate}[label=(\roman{*})]
  \item Es genügt die Behauptung für die erste partielle Ableitung zu zeigen,
\begin{align*}
&\frac{\psi(x+he_j)-\psi(x)}{h} =
S\left(\frac{\ph(x+he_j,\cdot)-\ph(x,\cdot)}{h} \right)\\ & \overset{S\text{
stetig}}{\to} S(\partial_{x_j}\ph(x,\cdot) ).\\ \Rightarrow &
\partial_{x_j}\psi(x) = S\left(\partial_{x_j} \ph(x,\cdot) \right)
= S\left(\nabla^{e_j,0} \ph(x,\cdot) \right)
\end{align*}
\item Mit \ref{prop:3.35:1} folgt, $\psi\in C^\infty(\R^n)$.  Sei
$\supp\ph\subseteq [-R,R]^{n+m}$, dann ist $\ph(x,\cdot) =0$ für $\abs{x_j}>R$
unabhängig von $\cdot$ und daher ist auch $\psi(x) = S(\ph(x,\cdot)) = 0$ außerhalb von
$[-R,R]^n$.
\item $\supp\ph_j \subseteq [-R,R]^{n+m}$ daher ist $\supp\psi_j\subseteq
[-R,R]^n$. Mit \ref{prop:3.36} folgt,
\begin{align*}
S(\xi) \le c\max\limits_{\abs{\beta}\le k}\norm{\nabla^\beta \xi}_\infty,
\end{align*}
für $\xi\in C_0^\infty(\R^n)$ mit $\supp\xi\subseteq [-R,R]^m$. Setze $\xi_j :=
\nabla^{(\alpha,0)}\ph_j(x,\cdot)$, so folgt,
\begin{align*}
\abs{\nabla^\alpha \ph_j(x)} &= \abs{S(\nabla^{(\alpha,0)} \ph_j(x,\cdot))} \le
c \max\limits_{\abs{\beta}\le k}\norm{\nabla^{(\alpha,\beta)}
\ph_j(x,\cdot)}_\infty\\
&\to 0\text{ glm. bezgl. } x.\qedhere
\end{align*}
\end{enumerate}
\end{proof}

\begin{prop}[Definition/Satz]
\label{prop:3.36}
Seien $T\in D'(\R^n)$, $S\in D'(\R^m)$. Dann ist das direkte Produkt definiert
als,
\begin{align*}
(T\times S)(\ph) := T_x(S_y(\ph)),\quad \text{für }\ph\in C_0^\infty(\R^{n+m}).
\end{align*}
Es gelten,
\begin{enumerate}[label=(\roman{*})]
  \item\label{prop:3.36:1} $T\times S\in D'(\R^n\times \R^m)$.
  \item\label{prop:3.36:2} Für $u\in\LL_\loc^1(\R^n)$, $v\in\LL_\loc^1(\R^m)$
  gilt,
\begin{align*}
T_u\times T_v = T_{u(x)\cdot v(y)}.
\end{align*}
\item\label{prop:3.36:3} Seien $\ph_1\in C_0^\infty(\R^n)$, $\ph_2\in
C_0^\infty(\R^m)$ und $\ph(x,y)=\ph_1(x)\ph_2(y)$, so gilt
\begin{align*}
T(\ph_1)S(\ph_2).
\end{align*}
\item\label{prop:3.36:4} $\supp T\times S = \supp T\times \supp S$.\fishhere
\end{enumerate}
\end{prop}
\begin{proof}
``\ref{prop:3.36:1}'': $T\times S$ ist nach \ref{prop:3.35} definiert. Die
Lineartität ist offensichtlich. Zum Nachweis der Stetigkeit sei $\ph_j\Dto 0$,
so folgt mit \ref{prop:3.35}, dass $\psi_j(x):= S_y(\ph(x,\cdot)\Dto 0$ und
damit,
\begin{align*}
T\times S(\ph_j) = T(\psi_j) \to 0.
\end{align*}
$T\times S$ ist linear, d.h. Stetigkeit in $0$ impliziert die Stetigkeit
überall.

``\ref{prop:3.36:2}, \ref{prop:3.36:3}'': Übung.

``\ref{prop:3.36:4}'': 
\begin{enumerate}[label=(\alph{*})]
  \item Sei $(x,y)\in\supp T\times \supp S$ und $U(x,y)$ Umgebung im
  $\R^{n+m}$, dann gilt
\begin{align*}
\exists U_1(x), U_2(y) : U_1(x)\times U_2(y) \subseteq U(x,y).
\end{align*}
Nun gilt,
\begin{align*}
&x\in\supp T\Rightarrow \exists \ph_1 \in C_0^\infty(U_1(x)) : T(\ph_1)\neq 0,\\
&y\in \supp S\Rightarrow \exists \ph_2 \in C_0^\infty(U_2(y)) : S(\ph_2)\neq 0.
\end{align*}
Setze $\ph(x,y) = \ph_1(x)\ph_2(y)$, dann gilt,
\begin{align*}
&\ph\in C_0^\infty(U(x,y)) \land
T\times S(\ph)  = T(\ph_1)S(\ph_2) \neq 0\\
\Rightarrow &(x,y)\in \supp T\times S.
\end{align*}
\item Sei $(x,y)\notin \supp T\times \supp S$. Ohne Einschränkung sei $x\notin
\supp T$, dann gilt
\begin{align*}
\exists U(x) \subseteq \R^n\text{ offen } : \forall \ph\in C_0^\infty(U(x)) :
T\ph = 0.
\end{align*}
$U(x)\times \R^m$ ist Umgebung von $(x,y)$ und $T=0$ auf $U(x)\times \R^m$,
denn für $\phi\in C_0^\infty(U(x)\times\R^m)$ gilt
\begin{align*}
T\times S(\phi) = T_x(\underbrace{S_y(\phi(x,y))}_{\in C_0^\infty(U(x))}) = 0.
\end{align*}
Also ist $(x,y)\notin \supp T\times S$.\qedhere
\end{enumerate}
\end{proof}

\begin{prop}[Fortsetzung der Distributionen]
\label{prop:3.37}
\begin{enumerate}[label=(\roman{*})]
  \item Zu $M\subseteq \R^n$ abgeschlossen sei,
\begin{align*}
C_M^\infty := \setdef{\ph\in C^\infty(\R^n\to C)}{\supp\ph\cap M\text{
beschränkt}}.
\end{align*}
$C_M^\infty$ ist ein linearer Raum und $C_0^\infty(\R^n)\subseteq C_M^\infty$.

Ist $M$ kompakt so ist $C_M^\infty = C^\infty(\R^n\to\C)$. Ist $M=\R^n$, so ist
$C_M^\infty = C_0^\infty(\R^n)$.
\item Sei $T\in D'(\R^n)$, $M:=\supp T$. Zu $\ph\in C_M^\infty$ wähle $\psi\in
C_0^\infty(\R^n)$ mit $\psi=1$ auf $K=\supp T\cap \supp\ph$. Setze
\begin{align*}
T\ph = T(\psi\ph).
\end{align*}
Diese Definition ist unabhängig vom gewählten $\psi$, denn sei $\tilde{\psi}$
ebenfalls $=1$ auf $K$, so gilt
\begin{align*}
T(\psi\ph) - T(\tilde{\psi}\ph) =T((\psi-\tilde{\psi})\ph)  = 0.
\end{align*}
\item \textit{Eigenschaften}. 
\begin{enumerate}[label=(\alph{*})]
  \item $T: C_M^\infty \to \C$ ist linear.
\item Die Ableitungsdefinition gilt weiterhin,
\begin{align*}
\D^\alpha T(\ph) &:= \D^\alpha T(\psi\ph) = (-1)^{\abs{\alpha}}
T(\underbrace{\nabla^\alpha \psi\ph}_{=\psi\nabla^\alpha\ph\text{ in }K}) \\ &=
(-1)^{\abs{\alpha}} T(\nabla^\alpha \ph).
\end{align*}
\item $T_j\to T$ in $D'(\R^n)$ und $\exists K$ kompakt mit $\supp T_j\subseteq
K$, so gilt,
\begin{align*}
\forall \ph\in C_K^\infty : T_j(\ph)\to T.
\end{align*}
\item $\supp \ph\cap \supp T=\varnothing \Rightarrow T(\ph) = 0$.\fishhere
\end{enumerate}
\end{enumerate}
\end{prop}

\begin{prop}
\label{prop:3.38}
Falls $(T,S)\in D'(\R^n)^2$ die Streifenbedingung erfüllt, dann ist $T*S\in
D'(\R^n)$.\fishhere
\end{prop}
\begin{proof}
$T*S(\ph) = (T_x\times S_y)(\ph(x+y))$
\begin{enumerate}[label=\arabic{*}.)]
  \item Sei $\ph\in C_0^\infty(\R^n)$, $\supp\ph\subseteq K_R(0)$,
  $\tilde{\ph}:= \ph(x+y)$, so ist $\supp\tph \subseteq \sigma_R$ und daher,
\begin{align*}
\supp T\times S\cap \supp\tph \subseteq \supp T\times\supp S \cap \sigma_R
\text{ beschränkt}.
\end{align*}
D.h. $\tph\in C_{\supp T\times S}^\infty\Rightarrow T*S(\tph)$ ist definiert.
\item $T*S$ ist offensichtlich linear.
\item Sei $\ph_j\Dto 0$, dann
\begin{align*}
\exists R > 0: \forall j\in\N : \supp \ph_j \subseteq K_R(0).
\end{align*}
Also ist $\supp T\times S\cap \supp\ph_j\subseteq K:=\supp T\times S\cap
\sigma_R$.

Sei $\psi\in C_0^\infty(\R^n)$, $\psi=1$ auf $K$ und $\tph_j(x,y) =
\ph_j(x+y)$, dann gilt
\begin{align*}
T*S(\ph_j) = T\times S(\tph_j) = T\times S(\underbrace{\psi\tphi_j}_{\Dto 0})
\to 0,
\end{align*}
denn $\supp\psi\tph_j\subseteq \supp\psi$ und $\nabla^\alpha (\psi\ph_j)
(x,y)\unito 0$ glm.\qedhere
\end{enumerate}
\end{proof}

\begin{bsp}
\label{bsp:3.39}
\begin{enumerate}[label=\arabic{*}.)]
  \item $T*\delta_0 = T$: $\delta_0$ ist finit und erfüllt daher die
  Streifenbedingung. Es gilt
\begin{align*}
&\delta_{0,y} \ph(x+y) = \ph(x),\\
&T*\delta_{0}(\ph) = T(\ph).
\end{align*}
\item Seien $\ph_1,\ph_2\in C_0^\infty(\R^n)$, so gilt $T_{\ph_1}* T_{\ph_2} =
T_{\ph_1*\ph_2}$.\bsphere
\end{enumerate}
\end{bsp}

\begin{prop}[Eigenschaften]
\label{prop:3.40}
\begin{enumerate}[label=\arabic{*}.)]
  \item $T*S = S*T$.
  \item $\forall\alpha\in\N_0^k : \D^\alpha T*S = (\D^\alpha T)*S =
  T*(\D^\alpha S)$.\fishhere
\end{enumerate}
\end{prop}
\begin{proof}[Beweisskizze.]
\begin{enumerate}[label=\arabic{*}.)]
  \item Zu $\ph\in C_0^\infty(\R^n)$ sei $\psi\in C_0^\infty(\R^{2n})$ mit
  $\psi=1$ auf $K\cup \setdef{(x,y)}{(y,x)\in K}$, $K=\supp T\times \supp S\cap
  \sigma_R$, wobei $\supp \ph \subseteq K_R(0)$.
\begin{align*}
T*S(\ph) &= T_x\times T_y(\psi(x,y)\ph(x+y)) \\ &\overset{\text{z.Z.}}{=}
S_y(T_x(\psi(x,y)\ph(y+x))) = S*T(\ph).
\end{align*}
Für $\ph_1,\ph_2\in C_0^\infty(\R^n)$ gilt
\begin{align*}
T_x(S_y(\ph_1(x)\ph_2(y))) &\overset{\ref{prop:3.30:3}}{=} T(\ph_1)S(\ph_2) =
S(\ph_2)T(\ph_1) \\ &= S_y\times T_x(\ph_1(x)\ph_2(y)),
\end{align*}
$\Rightarrow T_x\times S_y(\ph) = S_y\times T_x(\ph)$
für alle Linearkombinationen. Die Menge aller
\begin{align*}
\setdef{\sum\limits_{j=1}^N c_j \ph_j(x) \psi_j(y)}{N\in\N, \ph_j,\psi_j\in
C_0^\infty(\R^n)}
\end{align*}
ist dicht in $C_0^\infty(\R^n)$. (Siehe Walter Kap. 7)
\item
\begin{align*}
\D^\alpha (T*S) &= (-1)^{\abs{\alpha}}(T*S)(\nabla^\alpha \ph) \\ &=
(-1)^{\abs{\alpha}} T_x\times S_y(\psi(x,y)\nabla^\alpha\psi(x+y)) \\ &=
(-1)^{\abs{\alpha}} T_x(S_y(\nabla_y^\alpha(\psi(x,y)\psi(x+y)))) \\ &=
T_x(\D^\alpha S_y(\psi\ph)) = T*\D^\alpha S.\qedhere
\end{align*}
\end{enumerate}
\end{proof}

Es lässt sich also für $1\le p\le \infty$ jeder $\LL^p$ in
$D'(\R^n)$ einbetten und daher lässt sich jede $\LL^p$-Funktion im $\D$-Sinne
ableiten. Diejenigen $\LL^p$-Funktionen, deren Ableitungen wieder im $\LL^p$
sind, heißen \emph{Sobolev-Räume}.
