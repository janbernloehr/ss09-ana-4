\section{Fouriertransformation}

\subsection{Grundlagen}

\begin{defn}
\label{defn:2.1}
Für $f\in C^\infty(\R\to\C)$ und $j,k\in\N_0$ sei
\begin{align*}
\norm{f}_{j,k} := \sup\limits_{x\in\R} \abs{x^jf^{(k)}(x)}.
\end{align*}
Für $k\ge 1$ ist $\norm{f}_{j,k}$ eine Seminorm.
Der \emph{Schwartzraum}\footnote{Laurent Schwartz (* 5. März 1915 in Paris; †
4. Juli 2002 ebenda)} über $\R$ ist definiert als
\begin{align*}
\SS(\R) := \setdef{f\in C^\infty(\R\to\C)}{\forall j,k\in\N_0 :
\norm{f}_{j,k}<\infty}.\fishhere
\end{align*}
\end{defn}

\begin{bem}
\label{bem:2.2}
Mit
\begin{align*}
d(f,g) = \sum\limits_{j,k=0}^\infty
\frac{1}{2^{j+k}}\frac{\norm{f-g}_{j,k}}{1-\norm{f-g}_{j,k}}
\end{align*}
wird $\SS(\R)$ zu einem vollständigen metrischen Raum.\maphere
\end{bem}

\begin{bsp}
\label{bsp:2.3}
\begin{enumerate}[label=\arabic{*}.)]
  \item $f_1(x) = x^j e^{-x^2} \in \SS(\R)$, $\forall j\in\N_0$.
  \item $f_2(x) = \frac{1}{(1+x^2)^j}\notin \SS(\R)$, $\forall j\in\N_0$.
  \item Betrachten wir die Funktion,
  
\begin{align*}
f_3(x) = \begin{cases}e^{-\frac{1}{1-x^2}}, &
  x<1\\ 0, & x \ge 1\end{cases}.
\end{align*}
\begin{figure}[H]
\centering
\begin{pspicture}(-2.5,-1)(2.5,2)
\psaxes[labels=none,ticks=none]{->}%
 (0,0)(-2,-0.5)(2,1.5)%
 [\color{gdarkgray}$x$,-90][\color{gdarkgray}$y$,0]

 \psplot[linewidth=1.2pt,%
	     linecolor=darkblue,%
	     algebraic=true]%
	     {-0.999}{0.999}{2.71828183^(-1/(1-x^2))}
 \psline[linewidth=1.2pt,linecolor=darkblue](-1.5,0)(-1,0)
 \psline[linewidth=1.2pt,linecolor=darkblue](1,0)(1.5,0)
 
 \psxTick(-1){\color{gdarkgray}-1}
 \psxTick(1){\color{gdarkgray}1}
 \psyTick(0.367879441){}
 \rput(-0.4,0.6){\color{gdarkgray}1/e}
\end{pspicture} 
  \caption{Graph der Funktion $f_3$.}
\end{figure}

Offensichtlich ist $f\in C^\infty(\R\to\C)$ und $\supp f = [-1,1]$. Damit haben
auch alle Ableitungen kompakten Träger und es gilt $f\in\SS(\R)$.\bsphere
\end{enumerate}
\end{bsp}

\begin{prop}[Eigenschaften]
\label{prop:2.4}
\begin{enumerate}[label=\arabic{*}.)]
  \item $\SS(\R)$ ist eine $\R$-Algebra ohne Einselement.
  \item Seien $f\in\SS(\R)$, $j,k\in\N_0$ und
\begin{align*}
g(x) = x^jf^{(k)}(x),
\end{align*}
so ist auch $g\in\SS(\R)$.\fishhere
\end{enumerate}
\end{prop}
\begin{proof}
Der Beweis ist eine leichte Übung.\qedhere
\end{proof}

\begin{defn}
\label{defn:2.5}
Für $f: \R\to\C$ heißt
\begin{align*}
\supp f := \overline{\setdef{x\in\R}{f(x)\neq 0}}
\end{align*}
der \emph{Träger} bzw. \emph{Support} von $f$.
\begin{align*}
C_0^\infty(\R\to\C) := \setdef{f\in C^\infty(\R\to\C)}{\supp f\text{ ist
kompakt}}.
\end{align*}
Offensichtlich ist $C_0^\infty(\R\to\C)\subseteq \SS(\R)$.\fishhere
\end{defn}

\begin{defn}[Fouriertransformation]
\label{defn:2.6}
Für $f\in\SS(\R)$ heißt
\begin{align*}
\FF f(x) = \hat{f}(\omega) := \frac{1}{\sqrt{2\pi}}\int\limits_{-\infty}^\infty
f(x)e^{-i\omega x}\dx
\end{align*}
die \emph{Fouriertransformierte} von $f$.\fishhere
\end{defn}

\begin{bsp}
\label{bsp:2.7}
Wir wollen die Fouriertransformation der Funktion
\begin{align*}
f(x) = e^{-\frac{x^2}{2}}
\end{align*}
berechnen,
\begin{align*}
\sqrt{2\pi} \hat{f}(\omega) = \int\limits_{-\infty}^\infty
e^{-\frac{x^2}{2}}e^{-i\omega x}\dx.
\end{align*}
Quadratische Ergänzung ergibt,
\begin{align*}
\sqrt{2\pi} \hat{f}(\omega) &= \int\limits_{-\infty}^\infty
e^{-\frac{1}{2}(x^2 + 2i\omega x - \omega^2)}e^{-\frac{\omega^2}{2}}\dx
= e^{-\frac{\omega^2}{2}} 
\underbrace{\int\limits_{-\infty}^\infty
e^{-\frac{1}{2}((x-i\omega)^2)}\dx}_{:= \sqrt{2\pi}\text{ (s.u.)}}\\
&=\sqrt{2\pi}f(\omega).
\end{align*}
$f$ ist also Eigenfunktion der Fouriertransformation mit Eigenwert
$1$.

Wir wollen das $\sqrt{2\pi}$ nun Integral explizit berechnen. Dazu bedienen wir
uns der Funktionentheorie, denn $f(z)$ ist ganze Funktion. Es gilt somit
\begin{align*}
\int_{\gamma_R} f(z)\dz = \int_{\gamma_1} f(z)\dz + \int_{\gamma_2} f(z)\dz +
\int_{\tilde{\gamma}_R}f(z)\dz
\end{align*}

\begin{figure}[h]
\centering
\begin{pspicture}(0,-1.0589062)(4.5,1.0573437)
\psline{->}(0.0,-0.68109375)(4.48,-0.6610938)
\psline{->}(2.18,-0.86109376)(2.18,1.0189062)
\psline[linecolor=darkblue]{->}(0.76,0.5389063)(3.28,0.5389063)
\psline[linecolor=darkblue]{->}(0.78,0.5389063)(0.78,-0.12109375)
\psline[linecolor=darkblue]{->}(0.78,-0.68109375)(1.36,-0.68109375)
\psline[linecolor=darkblue]{->}(3.58,-0.68109375)(3.58,0.21890625)
\psline[linecolor=darkblue](0.78,-0.02109375)(0.78,-0.68109375)
\psline[linecolor=darkblue](3.04,0.5389063)(3.6,0.5389063)
\psline[linecolor=darkblue](3.58,-0.04109375)(3.58,0.5389063)
\psline[linecolor=darkblue](1.16,-0.68109375)(3.58,-0.6610938)

\rput(2.0271876,0.76890624){\color{gdarkgray}$\omega$}
\rput(3.4764063,0.8){\color{gdarkgray}$\gamma_R$}
\rput(1.4357812,-0.9){\color{gdarkgray}$\tilde{\gamma}_R$}
\rput(3.9571874,0.14890625){\color{gdarkgray}$\gamma_2$}
\rput(0.40265626,0.1){\color{gdarkgray}$\gamma_1$}
\end{pspicture} 
  \caption{Konstruktion der geschlossenen Kurve}
\end{figure}

\begin{align*}
\abs{\int_{\gamma_1} f(z)\dz} &\le \sup\limits_{z\in\im\gamma_1}
\abs{f(z)}L(\gamma_1)
= \sup\limits_{0\le t\le\omega} \abs{e^{-\frac{(it-R)^2}{2}}}\omega\\
&= \sup\limits_{0\le t\le\omega} \abs{e^{\frac{-R^2+2iRt+t^2}{2}}}\omega
= e^{-\frac{R^2}{2}}e^{\frac{\omega^2}{2}}\omega\overset{R\to\infty}{\longrightarrow}
0.
\end{align*}
Analog erhält man
\begin{align*}
\abs{\int_{\gamma_2} f(z)\dz} \to 0,\quad \text{für } R\to\infty,
\end{align*}
somit gilt
\begin{align*}
\int_{\gamma_R} f(z)\dz = \int_{\tilde{\gamma}_R} f(z)\dz
= \int\limits_{-\infty}^\infty e^{-\frac{t^2}{2}}\dt = \sqrt{2\pi}.\bsphere
\end{align*}
\end{bsp}

\subsection{Eigenschaften der Fouriertransformation}

\begin{prop}
\label{prop:2.8}
Für $f\in\SS(\R)$, $j,k\in\N_0$ gelten,
\begin{align*}
&\widehat{f^{(j)}}(\omega) = (i\omega)^j\hat{f}(\omega)\\
&g(x) = x^kf(x) \Rightarrow \hat{g}(\omega) = i^k\hat{f}^{(k)}(\omega)\\
&g(x) = x^kf^{(j)}(x) \Rightarrow \hat{g}(\omega) =
i^{k+j}x^j\hat{f}^{(k)}(\omega).\fishhere
\end{align*}
\end{prop}
\begin{proof}
\begin{enumerate}[label=\arabic{*}.)]
  \item Sei $j=1$.
\begin{align*}
\hat{f'}(\omega) &= \frac{1}{\sqrt{2\pi}}\int\limits_{-\infty}^\infty
f'(x)e^{-i\omega x}\dx\\
&\overset{\small\text{part.int.}} =
\frac{1}{\sqrt{2\pi}} \left[ \underbrace{f(x)e^{-i\omega
x}\bigg|_{-\infty}^\infty}_{=0} + i\omega \int\limits_{-\infty}^\infty
f(x)e^{-i\omega x}\dx \right]\\
&= i\omega \hat{f}(\omega).
\end{align*}
Rest folgt mit Induktion über $j$.
\item Sei $k=1$.
\begin{align*}
&g(x) = xf(x)\\
&\hat{g}(x) = \frac{1}{\sqrt{2\pi}}\int\limits_{-\infty}^\infty
f(x)xe^{-i\omega x}\dx = i\frac{1}{\sqrt{2\pi}}\int\limits_{-\infty}^\infty
f(x)\frac{\diffd}{\domega} e^{-i\omega x}\dx.
\end{align*}
Der abgeleitete Integrand konvergiert gleichmäßig, wir können also Ableitung
und Integral vertauschen,
\begin{align*}
\hat{g}(x) =
i \frac{\diffd}{\domega} \frac{1}{\sqrt{2\pi}}\int\limits_{-\infty}^\infty f(x)
e^{-i\omega x}\dx
= i\frac{\diffd}{\domega} \hat{f}(\omega).
\end{align*}
Wir sehen also insbesondere, dass $\hat{f}$ differenzierbar ist. Rest folgt mit
Induktion über $k$.\qedhere
\end{enumerate}
\end{proof}

\begin{prop}
\label{prop:2.9}
$\FF: f\mapsto \hat{f}$ ist eine lineare Abbildung von $\SS(\R)$ nach
$\SS(\R)$.\fishhere
\end{prop}
\begin{proof}
Die Linearität folgt direkt aus der Linearität des Integrals. Sei nun $f\in
\SS(\R)$,
\begin{align*}
\sup\limits_{\omega\in\R} \abs{\hat{f}(\omega)}
&=  \sup\limits_{\omega\in\R} \frac{1}{\sqrt{2\pi}}
\abs{\int\limits_{-\infty}^\infty f(x)e^{-i\omega x}\dx}\\
&\le \sup\limits_{\omega\in\R} \frac{1}{\sqrt{2\pi}}
\int\limits_{-\infty}^\infty \abs{f(x)}\dx <\infty
\end{align*}
Setze $g(x) = x^k f^{(j)}(x)$, so ist $g\in\SS(\R)$ und
$\hat{g}(\omega)=i^j\omega^j \hat{f}^{(k)}(\omega)$, wobei
\begin{align*}
\sup\limits_{\omega\in\R} \abs{\omega^j\hat{f}^{(k)}(\omega)} 
&= \sup\limits_{\omega\in\R} \abs{\hat{g}(\omega)} <\infty,
\end{align*}
also ist $f\in\SS(\R)$.\qedhere
\end{proof}

\begin{bem}
\label{bem:2.10}
$\FF: \SS(\R)\to\SS(\R)$ ist stetig.\maphere
\end{bem}

\begin{prop}
\label{prop:2.11}
$\FF: \SS(\R)\to\SS(\R)$ ist bijektiv mit,
\begin{align*}
\FF^{-1}g(x) = \check{g}(x) = \frac{1}{\sqrt{2\pi}}\int\limits_{-\infty}^\infty
g(\omega)e^{+i\omega x}\domega.\fishhere
\end{align*}
\end{prop}
\begin{proof}
\begin{enumerate}[label=\arabic{*}.)]
  \item\label{proof:2.10:1} Sei zunächst $f\in C_0^\infty(\R)$. Wähle $\ep > 0$
  so, dass
\begin{align*}
\supp f \subseteq \left(-\frac{1}{\ep},\frac{1}{\ep}\right).
\end{align*}
Dann kann $f$ $\frac{2}{\ep}$-periodisch fortgesetzt werden. Die Fortsetzung
ist $\C^\infty(\R\to\C)$. Entwickeln wir $f$ in
$\left[-\frac{1}{\ep},\frac{1}{\ep}\right]$ in seine Fourierreihe,
\begin{align*}
f(x) &= \sum\limits_{j=-\infty}^\infty \frac{1}{\frac{2}{\ep}}
\int\limits_{-\frac{1}{\ep}}^{\frac{1}{\ep}}
f(t)e^{ij\frac{\pi}{\frac{1}{\ep}}(x-t)}\dt\\
&= \sum\limits_{j=-\infty}^\infty \frac{\ep}{2}
\int\limits_{-\frac{1}{\ep}}^{\frac{1}{\ep}}
f(t)e^{-ij\pi\ep t}\dt e^{ij\pi\ep x}
\end{align*}
Da $f$ kompakten Träger $\subseteq \left[-\frac{1}{\ep},\frac{1}{\ep}\right]$
hat, können wir übergehen zu,
\begin{align*}
f(x) &= \sum\limits_{j=-\infty}^\infty \frac{\ep}{2}
\underbrace{\int\limits_{-\infty}^{\infty}
f(t)e^{-ij\pi\ep t}\dt}_{=\sqrt{2\pi}\hat{f}(j\pi\ep)} e^{ij\pi\ep x}\\
&=\frac{1}{\sqrt{2\pi}}
\sum\limits_{j=-\infty}^\infty \pi\ep
\hat{f}(j\pi\ep)e^{ij\pi\ep x}.
\end{align*}
Diesen Ausdruck kann man als Riemannsumme mit Schrittweite
$\pi\ep$ interpretieren. Für $\ep\to 0$ erhalten wir so,
\begin{align*}
f(x) = \frac{1}{\sqrt{2\pi}}\int\limits_{-\infty}^\infty \hat{f}(\omega)
e^{i\omega x}\domega.
\end{align*}
\item Sei $f\in\SS(\R)$ und $\psi_R$ eine \emph{Abschneidefunktion} mit
folgenden Eigenschaften,
\begin{itemize}
  \item $\psi_R\in C_0^\infty(\R)$.
  \item $\psi_R(x) = \begin{cases}1, & \abs{x}\le R,\\ 0, & \abs{x}\ge R+2,\\
  \in[0,1], & \text{sonst}\end{cases}$
\item $\abs{\psi_R'}, \abs{\psi_R''} \le c$ unabhängig von $R$.
\end{itemize}
%TODO: Bild
Wir werden im Anschluss an diesen Beweis zeigen, dass so eine Funktion überhaupt
existiert.

Für $f\in\SS(\R)$ sei
\begin{align*}
g_R := \psi_R \cdot f.
\end{align*}
Offensichtlich ist $g_R  \in C_0^\infty(\R)$. Wir können nun die Ergebnisse aus
\ref{proof:2.10:1} verwenden. Für $\abs{x}\le R$ gilt,
\begin{align*}
&f(x) = g_R(x) = \frac{1}{\sqrt{2\pi}}\int\limits_{-\infty}^\infty
\hat{g}_R(\omega)e^{i\omega x}\domega,\\
&\hat{g}_R(\omega) = \frac{1}{\sqrt{2\pi}}\int\limits_{-\infty}^\infty
\psi_R(x)f(x)e^{-i\omega x}\dx.
\end{align*}
Sei nun $\omega$ fest, dann gilt
\begin{align*}
e^{-i\omega x}\psi_R(x)f(x) \to e^{-i\omega x}f(x).
\end{align*}
Da außerdem $f\in\SS(\R)$ gilt weitherin,
\begin{align*}
\abs{\psi_R(x)f(x)} \le \abs{f(x)} \le \frac{c}{1+x^2}.
\end{align*}
Wir können somit den Satz über majorisierte Konvergenz anwenden und erhalten,
\begin{align*}
\int\limits_{-\infty}^\infty e^{-i\omega x}\psi_R(x)f(x)\dx 
\to \int\limits_{-\infty}^\infty e^{-i\omega x}f(x)\dx.
\end{align*}
Für $\omega$ fest konvergiert also $\hat{g}_R(\omega)\to \hat{f}(\omega)$ für
$R\to\infty$. Wir wollen nun noch einmal die Fouriertransformation ausführen
und die majorisierte Konvergenz verwenden. Dazu benötigen wir eine
weitere Majorante,
\begin{align*}
\abs{\hat{g}_R(\omega)} &=
\frac{1}{\sqrt{2\pi}}\abs{\int\limits_{-\infty}^\infty \psi_R(x)f(x)e^{-i\omega
x}\dx}\\ & \overset{\small\text{2x par.int.}}{=}
\frac{1}{\sqrt{2\pi}}\abs{\frac{1}{\omega^2}\int\limits_{-\infty}^\infty
(\psi_R(x)f(x))''e^{-i\omega x}\dx}.\tag{*}
\end{align*}
Nach Voraussetzung gilt,
\begin{align*}
&\abs{\psi_R}, \abs{\psi_R'}, \abs{\psi_R''} \le c,\\
&\abs{f},\abs{f'},\abs{f''} \le \frac{c}{1+x^2},
\end{align*}
und damit ist %TODO: Der Schritt ist mir überhaupt nicht klar...
\begin{align*}
(\text{*}) \le \abs{\frac{d}{1+\omega^2}}.
\end{align*}
Mit dem Satz über majorisierte Konvergenz folgt,
\begin{align*}
f(x) = \lim\limits_{R\to\infty} \frac{1}{\sqrt{2\pi}}
\int\limits_{-\infty}^\infty \hat{g}_R(\omega)e^{i\omega x}\dx
= 
\int\limits_{-\infty}^\infty \hat{f}(\omega)e^{i\omega x}\dx.
\end{align*}
Auf $\im\FF$ gilt also,
\begin{align*}
\FF^{-1}\hat{f} = \frac{1}{\sqrt{2\pi}} \int\limits_{-\infty}^\infty
\hat{f}(\omega) e^{i\omega x}\domega.
\end{align*}
\item Wir erhalten das gleiche Ergebnis, wenn wir $e^{-i\omega x}$ und
$e^{i\omega x}$ vertauschen,
\begin{align*}
&\check{f}(\omega) = \frac{1}{\sqrt{2\pi}}\int\limits_{-\infty}^\infty
f(x)e^{+i\omega x}\dx\\
\Rightarrow & f(x) = \frac{1}{\sqrt{2\pi}}\int\limits_{-\infty}^\infty
\check{f}(x)e^{-i\omega x}\dx = \FF\check{f}.
\end{align*}
Ist also $f\in\im\FF$, so ist auch $f =
\widehat{\left(\check{f}\right)}\in\im\FF$ und $\check{f}\in\SS(\R)$.\qedhere
\end{enumerate}
\end{proof}

\begin{bsp}
\label{bsp:2.12}
Durchbiegung einer unendlich langen Schiene.
\begin{figure}[!htbp]
\centering
\begin{pspicture}(0,-1.12)(3.94,1.1)
\psline{->}(0.0,-0.82)(3.92,-0.84)
\psline{->}(1.8,-0.94)(1.8,1.08)
\psline{->}(1.5,0.16)(1.5,0.68)
\psline{->}(2.4,-0.84)(2.4,-0.22)
\psbezier[linecolor=darkblue](0.02,-0.8)(0.24,-0.3)(0.62,-0.6)(0.78,-0.2)(0.94,0.2)(1.3,0.16)(1.8,0.16)(2.3,0.16)(2.32,-0.28)(2.82,-0.42)(3.32,-0.56)(3.32,-0.12)(3.6,-0.82)

\rput(1.2,0.81){\color{gdarkgray}$f(x)$}
\rput(2.8,-0.63){\color{gdarkgray}$u(x)$}
\rput(3.7254686,-0.97){\color{gdarkgray}$x$}
\end{pspicture}
\caption{Schiene mit Kraft $f(x)$ und Durchbiegung $u(x)$.}
\end{figure}

Beschreibe $u(x)$ die Durchbiegung einer Schiene hervorgerufen durch die 
Kraft $f(x)$. $u(x)$ erfüllt folgende Differentialgleichung, 
\begin{align*}
 u^{(4)}(x) + \alpha^4 u(x) = f(x).
\end{align*}

Nehmen wir an, $f,u\in\SS(\R)$ und wenden die Fouriertransformation auf die
Differentialgleichung an,
\begin{align*}
i^4\omega^4\hat{u}(\omega) + \alpha^4 \hat{u}(\omega) = \hat{f}(\omega).
\end{align*}
Für die Fouriertransformierte der Lösung gilt somit,
\begin{align*}
\hat{u}(\omega) = \frac{1}{\omega^4+\alpha^4}\hat{f}(\omega).
\end{align*}
Die Lösung ist gegeben durch,
\begin{align*}
u(x) = \frac{1}{\sqrt{2\pi}}\int\limits_{-\infty}^\infty
\frac{\hat{f}(\omega)}{\omega^4+\alpha^4} e^{i\omega x}\domega.
\end{align*}

Wir wollen nun unsere Annhame überprüfen.
\begin{align*}
f\in\SS(\R)\Rightarrow \hat{f}\in\SS(\R) \Rightarrow
\frac{\hat{f}}{\cdot^4+\alpha^4}\in\SS(\R) \Rightarrow u\in\SS(\R).
\end{align*}
Es muss also lediglich gefordert werden, dass $f\in\SS(\R)$.

Dies ist eine lineare Differentialgleichung 4. Ordnung, der Lösungsraum ist
also 4 Dimensional, unsere Lösung ist somit nicht eindeutig.  Man kann jedoch
zeigen, dass man so die einzige Lösung erhält, die ``schnell abklingt''
($\in\SS(\R)$). So lange wir also Schienen mit endlicher Ausdehnung betrachten
ist diese Lösung die einzige.\bsphere
\end{bsp}

\begin{prop}[Abschneidefunktion]
\label{prop:2.13}
Es existiert eine Abschneidefunktion $\psi_R$ für $R>0$ mit den in
\ref{prop:2.11} geforderten Eigenschaften,
\begin{enumerate}[label=(\roman{*})]
  \item $\psi_R\in C_0^\infty(\R)$.
  \item $\psi_R(x) = \begin{cases}1, & \abs{x}\le R,\\ 0, & \abs{x}\ge R+2,\\
  \in[0,1], & \text{sonst}\end{cases}$
\item $\abs{\psi_R'}, \abs{\psi_R''} \le c$ unabhängig von $R$.\fishhere
\end{enumerate}
\end{prop}
\begin{proof}
Sei dazu,
\begin{align*}
&j(x) = \begin{cases}
       c e^{-\frac{1}{1-x^2}}, & \abs{x}<1,\\
       0, & \text{sonst}.
       \end{cases},\\
&c = \frac{1}{\int\limits_{-1}^1 e^{-\frac{1}{1-x^2}}}.
\end{align*}
$j$ hat offensichtlich die Eigenschaften,
\begin{align*}
&\int\limits_{-\infty}^\infty j(x)\dx = \int\limits_{-1}^1 j(x)\dx = 1,\\
&\supp j = [-1,1],\\
& j\in C_0^\infty(\R\to\R),\\
& j(x) \ge 0.
\end{align*}
Setze nun,
\begin{align*}
\psi_R(x) := \int\limits_{-\infty}^\infty j(x-y)\chi_{[-R-1,R+1]}(y)\dy
= \int\limits_{-R-1}^{R+1} j(x-y)\dy,
\end{align*}
so folgt unmittelbar
\begin{align*}
\psi_R\in C^\infty(\R\to\R).
\end{align*}
Für $\abs{x}\le R$ gilt,
\begin{align*}
\psi_R(x) = \int\limits_{x-1}^{x+1} j(x-y)\dy = 1.
\end{align*}
Für $\abs{x}\ge R+2$ gilt,
\begin{align*}
\psi_R(x) = \int\limits_{-R-1}^{R+1} 0\dy = 0.
\end{align*}
$j$ ist positiv, also ist auch $\psi_R$ positiv und
\begin{align*}
\psi_R(x) \le \int\limits_{-\infty}^\infty j(x-y)\dy = 1.
\end{align*}
Außerdem gilt,
\begin{align*}
\abs{\psi_R^{(k)}} &\le \abs{\int\limits_{-\infty}^\infty j^{(k)}(x-y)\dy}
\overset{z=x-y}{=} \abs{\int\limits_{-\infty}^\infty j^{(k)}(z)\dz}
\\&= \int\limits_{-1}^1 \abs{j^{(k)}(z)}\dz \le 2\max\limits_{\abs{z}\le 1}
\abs{j^{(k)}(z)}
\end{align*}
und die rechte Seite ist unabhängig von $R$.\qedhere
\end{proof}
\begin{figure}[!htbp]
\centering
\begin{pspicture}(0,-1.2421875)(3.94,1.2221875)
\psline{->}(0.0,-0.6978125)(3.92,-0.6978125)
\psline{->}(1.82,-0.8178125)(1.82,1.2021875)

\psbezier[linecolor=darkblue]%
(0.2,-0.6978125)(0.48,-0.6978125)%
(0.64,-0.6978125)(0.82,-0.6978125)%
(1.0,-0.6978125)(1.3,0.3021875)%
(1.82,0.3021875)(2.34,0.3021875)%
(2.66,-0.6978125)(2.82,-0.6978125)%
(2.98,-0.6978125)(3.14,-0.6978125)(3.42,-0.6978125)
\psline(0.8,-0.6978125)(0.8,-0.8978125)
\psline(2.82,-0.6978125)(2.82,-0.8978125)

\rput(3.7254686,-0.8478125){\color{gdarkgray}$x$}
\rput(0.780625,-1.0878125){\color{gdarkgray}$-1$}
\rput(1.7770313,-1.0878125){\color{gdarkgray}$0$}
\rput(2.846875,-1.0878125){\color{gdarkgray}$1$}
\rput(1.5896875,-0.4278125){\color{gdarkgray}$F$}
\rput(3,0.2121875){\color{gdarkgray}$y=j(x)$}
\rput(0.985,0.4121875){\color{gdarkgray}$F=1$}
\end{pspicture} 
\begin{pspicture}(-0.3,-1.21)(5.9,1.19)
\psline{->}(0.03375,-0.67)(5.03375,-0.67)
\psline{->}(2.03375,-0.85)(2.03375,1.17)
\psline(2.29375,-0.67)(2.29375,-0.87)
\psline(4.25375,-0.67)(4.25375,-0.87)
\psline(0.11375,-0.67)(0.11375,-0.87)
\psline(3.73375,0.07)(3.73375,-0.87)
\psbezier[linecolor=darkblue](1.65375,-0.67)(1.93375,-0.67)(2.09375,-0.67)(2.27375,-0.67)(2.45375,-0.67)(2.75375,0.33)(3.27375,0.33)(3.79375,0.33)(4.11375,-0.67)(4.27375,-0.67)(4.43375,-0.67)(4.59375,-0.67)(4.87375,-0.67)

\rput(5.059219,-0.9){\color{gdarkgray}$x$}
\rput(2.319375,-1.06){\color{gdarkgray}$x-1$}
\rput(1.8707813,-0.9){\color{gdarkgray}$0$}
\rput(4.35,-1.06){\color{gdarkgray}$x+1$}
\rput(4.8,0.28){\color{gdarkgray}$y=j(x-y)$}
\rput(0.23859376,-1.06){\color{gdarkgray}$-R-1$}
\rput(3.55,-1.06){\color{gdarkgray}$R+1$}
\rput(2.6225,0.42){\color{gdarkgray}$\psi_R$}

\end{pspicture} 
\caption{Verschiebung.}
\end{figure}

\begin{prop}[Plancherel Gleichung]
\label{prop:2.14}
Für $f\in\SS(\R)$ gilt,
\begin{align*}
\norm{\hat{f}}_{\LL^2(\R)} = \norm{f}_{\LL^2(\R)}.\fishhere
\end{align*}
\end{prop}
\begin{proof}
\begin{enumerate}[label=\arabic{*}.)]
  \item Sei zunächst $f\in C_0^\infty(\R)$. Wie in \ref{prop:2.11} sei 
\begin{align*}
\supp f \subseteq \left(-\frac{1}{\ep},\frac{1}{\ep}\right).
\end{align*}
Mithilfe der Parsevall-Gleichung in
$\LL^2\left(-\frac{1}{\ep},\frac{1}{\ep}\right)$ und dem  VONS
\begin{align*}
e_j(x) = \sqrt{\frac{\ep}{2}}e^{ij\pi\ep x},
\end{align*}
erhalten wir
\begin{align*}
\norm{f}_{\LL^2\left(-\frac{1}{\ep},\frac{1}{\ep}\right)} &= 
\sum\limits_{j=-\infty}^\infty \abs{\lin{f,e_j}}^2 = 
\sum\limits_{j=-\infty}^\infty \abs{\int\limits_{-\frac{1}{\ep}}^{\frac{1}{\ep}}
f(t)\sqrt{\frac{\ep}{2}}e^{ij\pi\ep t}\dt}^2\\
&= \sum\limits_{j=-\infty}^\infty
\frac{\ep}{2}2\pi\abs{\hat{f}(j\pi\ep)}^2 
\end{align*}
Dies können wir wieder als Riemann-Summe interpretieren. Für $\ep\to0$ erhalten
wir so,
\begin{align*}
\ldots = \int\limits_{-\infty}^\infty \abs{\hat{f}(\omega)}^2\domega.
\end{align*}
\item Verwende nun die Abschneidefunktion $\psi_R$ und zeige für $f\in\SS(\R)$, 
\begin{align*}
&\norm{\psi_R f} \to\norm{f},\\
&\norm{\widehat{\psi_R f}} \to\norm{\hat{f}}.\qedhere
\end{align*}
\end{enumerate}
\end{proof}

\begin{cor}
\label{prop:2.15}
Für $f,g\in\SS(\R)$ gilt,
\begin{align*}
\lin{f,g}_{\LL^2(\R)} = \lin{\hat{f},\hat{g}}_{\LL^2(\R)}.\fishhere
\end{align*}
\end{cor}
\begin{proof}
Wir führen das Skalarprodukt mittels folgender Identität auf seine erzeugte
Norm zurück,
\begin{align*}
\lin{f,g} = \frac{1}{4}\left(\norm{f+g}^2 - \norm{f-g}^2 - i\norm{f+ig}^2 -
\norm{f-ig}^2 \right).\qedhere
\end{align*}
\end{proof}

\addtocounter{prop}{1}
Die Fouriertransformation ist also eine bijektive in beide Richtungen stetige
Abbildung, die Norm und Skalarprodukt erhält.

Wir wollen die Fouriertransformation von Produkten von Funktionen betrachten.
Dazu benötigen wir noch den Begriff der Faltung.
\begin{defn}
\label{defn:2.17}
Für $f,g\in\SS(\R)$ ist
\begin{align*}
(f*g)(x) = \int\limits_{-\infty}^\infty f(x-y)g(y)\dy,
\end{align*}
die \emph{Faltung} von $f$ mit $g$.\fishhere
\end{defn}

\begin{bspn}
$\psi_R  = j*\chi_{[-R-1,R+1]}$.\bsphere
\end{bspn}

\begin{prop}
\label{prop:2.18}
Seien $f,g\in\SS(\R)$. Dann gilt,
\begin{enumerate}[label=\arabic{*}.)]
  \item\label{prop:2.18:1} $f*g\in\SS(\R)$.
  \item\label{prop:2.18:2} $\widehat{f\cdot g} = \frac{1}{\sqrt{2\pi}}
  \hat{f}*\hat{g}$,\\
  $\widehat{f*g} = \sqrt{2\pi} \hat{f}\cdot\hat{g}$.
  \item\label{prop:2.18:3} Die Faltung ist kommutativ und assoziativ,
\begin{align*}
f*g = g*f,\qquad f*(g*h) = (f*g)*h.\fishhere
\end{align*}
\end{enumerate}
\end{prop}
\begin{proof}
``\ref{prop:2.18:2}'':
\begin{align*}
\sqrt{2\pi} \widehat{f\cdot g}(\omega) &=
\int\limits_{-\infty}^\infty f(x)g(x)e^{-i\omega x}\dx
= \lin{g,\overline{e^{-i\cdot\omega}f}}_{\LL^2(\R)}
\\ &= \lin{\hat{g},\widehat{\overline{e^{-i\cdot\omega}f}}}_{\LL^2(\R)}
=  \int\limits_{-\infty}^\infty \hat{g}(\omega')
\overline{\int\limits_{-\infty}^\infty \overline{e^{-ix\omega}f(x)}
e^{-ix\omega'}\dx}\domega'
\end{align*}
Nun ist
\begin{align*}
\overline{\int\limits_{-\infty}^\infty \overline{e^{-ix\omega}f(x)}
e^{-ix\omega'}\dx}
= 
\int\limits_{-\infty}^\infty f(x)e^{-ix(\omega-\omega')}
\dx = \hat{f}(\omega-\omega').
\end{align*}
Wir erhalten so,
\begin{align*}
\sqrt{2\pi} \widehat{f\cdot g}(\omega) = 
\int\limits_{-\infty}^\infty \hat{g}(\omega')
\hat{f}(\omega-\omega')\domega' = \hat{f}*\hat{g}.
\end{align*}
Analog folgt, $\FF^{-1}\left(\hat{f}\cdot\hat{g}\right) =
\frac{1}{\sqrt{2\pi}} f* g \Rightarrow \hat{f}\cdot\hat{g} =
\frac{1}{\sqrt{2\pi}} \widehat{f*g}$.

``\ref{prop:2.18:1}'':
\begin{align*}
f,g\in\SS(\R) &\overset{\ref{prop:2.9}}{\Rightarrow} \hat{f},\hat{g}\in\SS(\R)
\Rightarrow \hat{f}\cdot\hat{g}\in\SS(\R)\\
&\overset{\ref{prop:2.9}}{\Rightarrow} f*g =
\frac{1}{\sqrt{2\pi}}\FF^{-1}\left(\hat{f}\cdot\hat{g}\right) \in\SS(\R)
\end{align*}

``\ref{prop:2.18:3}'':
\begin{align*}
f*g &= \frac{1}{\sqrt{2\pi}}\FF^{-1}\left(\hat{f}\hat{g}\right) =
\frac{1}{\sqrt{2\pi}}\FF^{-1}\left(\hat{g}\hat{f}\right) = g*f\\
\left(f*g\right)*h &= 
\frac{1}{\sqrt{2\pi}}\FF^{-1}\left(\widehat{f*g}\cdot \hat{h}\right)=
\frac{1}{\sqrt{2\pi}}\FF^{-1}\left(\frac{1}{\sqrt{2\pi}}\hat{f}\cdot\hat{g}\cdot
\hat{h}\right) \\ &= 
\frac{1}{\sqrt{2\pi}}\FF^{-1}\left(\hat{f}\cdot \widehat{g*h}\cdot
\right)= f*\left(g*h\right).\qedhere
\end{align*}
\end{proof}

\subsection{Dichte Mengen}

\begin{prop}
\label{prop:2.19}
Sei $1\le p <\infty$. Dann ist $C_0^\infty(\R)$ dicht in $\LL^p(\R)$.\fishhere
\end{prop}

Wir benötigen für diesen Beweis noch etwas Vorbereitung.

\begin{defn}
\label{defn:2.20}

Sei $j\in C_0^\infty(\R)$ mit
\begin{enumerate}[label=\arabic{*}.)]
  \item $j(x) \ge 0$,
  \item $j(x) = 0$ für $\abs{x}\ge 1$,
  \item $\int_\R j(x)\dx = 1$.
\end{enumerate}
(vgl \ref{prop:2.13} $j(x) = ce^{-\frac{1}{1-x^2}}$). Setze für $\ep>0$,
\begin{align*}
j_\ep(x) = \frac{1}{\ep}j\left(\frac{x}{\ep} \right).
\end{align*}
Offensichtlich gilt
\begin{align*}
\begin{cases}
j_\ep(x)\ge 0,\\
j_\ep(x) = 0,& \text{für }\abs{x}\ge\ep,\\
j_\ep\in C_0^\infty(\R),\\
\int_\R j_\ep(x)\dx \overset{y=\frac{x}{\ep}}{=} \int_\R j(y)\dy = 1.
\end{cases}
\end{align*}
Für $f\in\LL^p(\R)$ setze
\begin{align*}
J_\ep f := j_\ep * f = \int_\R j_\ep(\cdot -y)f(y)\dy.
\end{align*}
$J_\ep f$ heißt \emph{Glättungsoperator} oder \emph{Mollifier}.\fishhere
\end{defn}

\begin{bsp}
\label{bsp:2.21}
$f(x) = \chi_{[-1,1]}$.
%TODO: Schaubild
\begin{align*}
J_\ep f(x) &= 
\int\limits_{-\infty}^\infty j_\ep(x-y)\chi_{[-1,1]}(y)\dy
= \int\limits_{-1}^1 j_\ep(x-y)\dy\\
&= \begin{cases}
  0, & x\le 1-\ep\lor x\ge 1+\ep,\\
  1 & -1+\ep\le x \le 1-\ep\\
  \in[0,1], & \text{sonst}.
  \end{cases}
\end{align*}
%TODO: Skizzen
\end{bsp}

\begin{prop}
\label{prop:2.22}
Sei $1\le p<\infty$, $u\in\LL^p(\R)$. Dann gilt
\begin{enumerate}[label=\arabic{*}.)]
  \item $j_\ep * u\in C^\infty(\R\to\C)$.
  \item $\supp u$ beschränkt $\Rightarrow j_\ep * u \in C_0^\infty(\R)$ 
  \item $j_\ep * u\in\LL^p(\R)$, $\norm{j_\ep*u}_{\LL^p(\R)}\le
  \norm{u}_{\LL^p(\R)}$.
  \item $\norm{j_\ep*u-u}_{\LL^p(\R)} \to 0$ für $\ep\downarrow 0$.\fishhere
\end{enumerate}
\end{prop}
\begin{proof}
\begin{enumerate}[label=\arabic{*}.)]
  \item Übung.
  \item $\supp u\subseteq [-A,A]\Rightarrow \supp j_\ep * u \subseteq [-A-\ep,
  A+\ep]$.
  \item Zeige die Hilfsungleichung,
\begin{align*}
\abs{j_\ep* u(x)} \le \left(\int_\R j_\ep(x-y) \abs{u(y)}^p \dy\right)^{1/p}.
\end{align*}
Für $p=1$ folgt die Behauptung sofort. Sei nun $p> 1$ und
\begin{align*}
\frac{1}{p}+\frac{1}{q} = 1.
\end{align*}
Dann gilt,
\begin{align*}
\abs{j_\ep*u(x)}
&\le \int\limits_{-\infty}^\infty j_\ep(x-y)^{1/q}\left(j_\ep(x-y)^{1/p}u(y)
\right)\dy\\
&\le \underbrace{\left(\int\limits_{-\infty}^\infty j_\ep(x-y)\dy\right)^{1/q}}_{=1}
\left(\int\limits_{-\infty}^\infty j_\ep(x-y) \abs{u(y)}^p\dy \right)^{1/p}.
\end{align*}
Nun ist
\begin{align*}
&\int_\R \abs{j_\ep*u(x)}^p \dx
\le \int_\R \left(\int_\R \underbrace{j_\ep(x-y)}_{\ge 0}
\underbrace{\abs{u(y)}^p}_{\ge 0} \dy\right)\dx\\
&\overset{\small\text{Fubini I}}{=}
\int_\R\underbrace{\left(\int_\R  j_\ep(x-y)\dx\right)}_{=1} \abs{u(y)}^p\dy
= \norm{u}_p^p.
\end{align*}
\item
\begin{enumerate}[label=\alph{*})]
\item Für $u=\chi_{(a,b)}$ ist
\begin{align*}
&j_e*u(x) = \begin{cases}
1, & a+\ep \le x \le b-\ep,\\
0, & x\le a-\ep \land x\ge b+\ep,\\
\in[0,1], & \text{sonst}. 
\end{cases}
\end{align*}
Setzen wir $I=[a-\ep,a+\ep]\cup[b-\ep,b+\ep]$, so erhalten wir,
\begin{align*}
\Rightarrow \norm{u-j_\ep*u}_p^p &=
\int_I
\underbrace{\abs{u(x)-j_\ep*u(x)}^p}_{\le 2^p}\dx \\
&\le 2^p 4\ep \to 0
\end{align*}
\item Für $u=\chi_O$ mit $O\subseteq\R$ offen, gilt
\begin{align*}
O = \dot{\bigcup}_{j\in\N} (a_j,b_j).
\end{align*}
Setze $u_j = \sum\limits_{k=1}^j \chi_{(a_j,b_j)} \Rightarrow 0\le u_j(x) \le
u(x)$, $u_j(x)\uparrow u(x)$.
\begin{align*}
&\Rightarrow 0 \le (u(x)-u_j(x))^p \le u(x)^p,\qquad (u(x)-u_j(x))^p \to 0.
\end{align*}
Mit majorisierter Konvergenz folgt,
\begin{align*}
\norm{u-u_j}_p^p = \int_\R \abs{u(x)-u_j(x)}^p\dx \to 0.
\end{align*}
\begin{align*}
\norm{u-j_\ep*u}_p
&\le \norm{u-u_j}_p + \norm{u_j-j_\ep*u_j}_p + \norm{j_\ep*(u_j-u)}_p\\
&\le \underbrace{2\norm{u-u_j}_p}_{<\delta \text{ für }j>J_\delta} + 
\underbrace{\norm{u_j - j_\ep * u_j}_p}_{\small\text{endl. Summe } <\delta,\;
\ep<\ep_{\delta,j}}  \\
&\to 0,\quad\text{für }\ep\downarrow0.
\end{align*}
\item Sei $u=\chi_B$ mit $B\subseteq\R$ Borel-Menge. Zu $\delta > 0$ existiert
$O$ offen in $\R$ mit $B\subseteq O$ und $\mu(O\setminus B)<\delta$.
\begin{align*}
\Rightarrow \norm{\chi_B - j_\ep*\chi_B}_p \le
2\underbrace{\norm{\chi_B-\chi_O}_p}_{2\mu(O\setminus B)\;<\;2\delta^{1/p}} +
\underbrace{\norm{\chi_O-j_\ep*\chi_O}}_{\to 0,\;\ep\downarrow 0}.
\end{align*}
\item Sei $u\in\LL^p(\R)$. Ohne Einschränkung ist $u(x)\ge 0$, denn $u$ ist
messbar und daher $u(x) = u_+(x) + u_-(x)$. Es existiert nun eine Folge $(s_n)$
von einfachen Funktionen mit,
\begin{align*}
0\le s_n(x)\le u(x),\quad s_n(x)\uparrow u(x).
\end{align*}
Mit monotoner Konvergenz folgt,
\begin{align*}
&\norm{u-s_n}_p \to 0.\\
\Rightarrow&
\norm{u-j_\ep*u}_p \le 2\norm{u-s_n}_p + \norm{s_n-j_\ep*s_n}_p \to 0.\qedhere
\end{align*}
\end{enumerate}
\end{enumerate}
\end{proof}

\begin{proof}[Beweis von Satz \ref{prop:2.19}.]
Sei $u\in\LL^p(\R)$. Zu zeigen ist,
\begin{align*}
\forall\delta > 0 \exists \ph_\delta \in C_0^\infty(\R) : \norm{\ph_\delta-u}_p
<\delta.
\end{align*}
Sei zunächst,
\begin{align*}
u_j := \chi_{[-j,j]}u = \begin{cases}
                        u(x), & \abs{x}\le j,\\
                        0, &\abs{x} >j.
                        \end{cases}
\end{align*}
Offensichtlich gilt dann,
\begin{align*}
&\abs{u(x)-u_j(x)}^p \to 0,\quad \text{für }j\to\infty,\\
&\abs{u(x)-u_j(x)}^p \le \abs{u(x)}^p,\quad \int_\R \abs{u(x)}^p \dx <\infty.
\end{align*}
Wir können also majorisierte Konvergenz auf $\abs{u-u_j}^p$ anwenden und
erhalten,
\begin{align*}
\norm{u_j-u}_p \to 0.
\end{align*}
Sei nun $\delta > 0$. Wähle $j\in\N$ fest mit
$\norm{u-u_j}<\frac{\delta}{2}$. Nun ist $\supp u_j = [-j,j]$ also beschränkt.
Mit \ref{prop:2.22} folgt daher,
\begin{align*}
&j_\ep*u\in C^\infty,\\
&\norm{j_\ep*u_j - u_j}_p \to 0,\quad \text{für }\ep\downarrow 0. 
\end{align*}
Wähle $\ep >0$ fest mit $\norm{j_\ep*u_j - u_j}_p<\frac{\delta}{2}$, dann gilt
\begin{align*}
\ph_\delta := j_\ep*u_j \in C_0^\infty(\R).
\end{align*}
Wir erhalten,
\begin{align*}
\norm{u-\ph_\delta}_p \le \norm{u-u_j}_p + \norm{u_j-\ph_\delta}<
\delta.\qedhere
\end{align*}
\end{proof}

\subsection{Fortsetzung der Fouriertransformation}

Die Fouriertransformation ist eine lineare invertierbare Abbildung. Die 
Plancharell Gleichung besagt außerdem, dass sie die $\LL^2$-Norm erhält,
\begin{align*}
\norm{\hat{f}}_2 = \norm{f}_2.
\end{align*}
$D(F)=\SS(\R)$ liegt dicht in $\LL^2(\R)$. Dies wollen wir nun ausnutzen, um
die Fouriertransformation auf ganz $\LL^2(\R)$ fortzusetzten. Dazu benötigen
wir zunächst ein Hilfsmittel aus der Funktionalanalysis.

\begin{prop}[Fortsetzungssatz]
\label{prop:2.23}
Sei $(L,\norm{\cdot}_L)$ ein normierter Raum, $(B,\norm{\cdot}_B)$ Banachraum,
\begin{align*}
T:D(T)\to B,\qquad D(T)\subseteq L,
\end{align*}
linear und beschränkt und $D(T)$ dicht in $L$. Dann besitzt $T$
genau eine Fortsetzung $\tilde{T} : L\to B$. Diese ist beschränkt und linear.
Außerdem gilt
\begin{align*}
\norm{T} = \norm{\tilde{T}}.\fishhere
\end{align*}
\end{prop}
\begin{proof}
\begin{enumerate}[label=\arabic{*}.)]
  \item \textit{Konstruktion von $\tilde{T}$}.

Sei $x_0\in L$, $(x_n)$ Folge in $D(T)$ mit $x_n\to x_0$. Setze
\begin{align*}
\tilde{T}x_0 := \lim\limits_{n\to\infty} Tx_n.
\end{align*}
\begin{enumerate}[label=\alph{*})]
  \item\label{proof:2.23:1:1} $D(T)$ liegt dicht in $L$, also existiert $(x_n)$.
  \item\label{proof:2.23:1:2} $(Tx_n)$ konvergiert.
\begin{align*}
\norm{Tx_n - Tx_m}_B \le \norm{T}\norm{x_n-x_m}_L <\norm{T}\ep,
\end{align*}
für $n,m>N_\ep$. $(x_n)$ ist konvergent und daher Cauchyfolge, also ist $(Tx_n)$
konvergent.
\item\label{proof:2.23:1:3} Die Definition von $\tilde{T}x_0$ ist unabhängig
von der Folge $(x_n)$, denn seien $x_n\to x_0$ und $x_n'\to x_0$, dann gilt
\begin{align*}
&\norm{x_n-x_n'}_L\le \norm{x_n-x_0}_L+\norm{x_0-x_n'}_L\to 0\\
\Rightarrow & \norm{Tx_n-Tx_n'}_B\le \norm{T}\norm{x_n-x_n'}_L \to 0.
\end{align*}
\end{enumerate}
\ref{proof:2.23:1:1}-\ref{proof:2.23:1:3} $\Rightarrow$ $\tilde{T}x_0$ ist
definiert für beliebige $x_0\in L$.
\item \textit{$\tilde{T}$ besitzt die geforderten Eigenschaften.}
\begin{enumerate}[label=\alph{*})]
\item $\tilde{T}$ ist linear.
\begin{align*}
\tilde{T}(ax+by) &= \lim\limits_{n\to\infty} T(ax_n+by_n) =
\lim\limits_{n\to\infty} aTx_n + \lim\limits_{n\to\infty} bTy_n 
\\ &= a\tilde{T}x + b\tilde{T}y.  
\end{align*}
\item $\tilde{T}$ ist Fortsetzung von $T$. Für $x\in D(T)$ gilt aufgrund der
Stetigkeit von $T$,
\begin{align*}
\tilde{T}x = Tx,\qquad \text{wähle z.B. }x_n = x.
\end{align*}
\item $\tilde{T}$ ist beschränkt.
\begin{align*}
\norm{\tilde{T}x}_B &= \norm{\lim\limits_{n\to\infty}Tx_n}_B =
\lim\limits_{n\to\infty}\norm{Tx_n}_B \le
\lim\limits_{n\to\infty}\norm{T}\norm{x_n}_L \\ &= \norm{T}\norm{x}_L.
\end{align*}
Also gilt $\norm{\tilde{T}}\le\norm{T}$. Insbesondere ist $\tilde{T}$
beschränkt.
\item $\norm{\tilde{T}}=\norm{T}$.
\begin{align*}
\norm{\tilde{T}} = \sup\limits_{\atop{x\neq 0}{x\in L}}
\frac{\norm{\tilde{T}x}_B}{\norm{x}_L} \ge
\sup\limits_{\atop{x\neq 0}{x\in D(T)}}
\frac{\norm{\tilde{T}x}_B}{\norm{x}_L} =\norm{T}.
\end{align*}
\end{enumerate}
\item \textit{$\tilde{T}$ ist eindeutig}. $\tilde{T}$ ist beschränkt also
stetig, sei $x_0\in L$, $(x_n)$ in $D(T)$ mit $x_n\to x_0$, dann gilt
\begin{align*}
\tilde{T}x_n = \lim\limits_{n\to\infty} \tilde{T}x_n \lim\limits_{n\to\infty}
Tx_n. 
\end{align*}
Es gibt also nur eine Möglichkeit, $T$ fortzusetzen.
Also ist $\tilde{T}$ eindeutig.\qedhere
\end{enumerate}
\end{proof}

\begin{prop}
\label{prop:2.24}
Die Fouriertransformation
\begin{align*}
\FF: \SS(\R)\to \SS(\R),\; f\mapsto \hat{f}
\end{align*}
besitzt eine eindeutige lineare beschränkte Fortsetzung,
\begin{align*}
F: \LL^2(\R)\to \LL^2(\R).
\end{align*}
Die Fortsetzung ist unitär, d.h. bijektiv und Skalarprodukt-erhaltend, und es
gilt $\norm{\FF}=\norm{F}$. Die dazu inverse Abbildung
$F^{-1}$ ist die Fortsetzung der ursprünglichen inversen Abbildung $f\mapsto
\check{f}$.\fishhere
\end{prop}
\begin{proof}
Wir führen den Beweis in drei Schritten.
\begin{enumerate}[label=\arabic{*}. Schritt]
  \item $L=B=\LL^2(\R)$, $T=\FF$, $D(T)=\SS(\R)$. Mit \ref{prop:2.23} folgt, es
  existiert eine eindeutige lineare beschränkte Fortsetzung $F$ der
  Fouriertransformation auf $\LL^2(\R)$ mit $\norm{F}=\norm{\FF}=1$.
  
  Analog folgt, es existiert eine eindeutige lineare beschränkte Fortsetzung
  $G$ der Umkehrtransformation mit $\norm{G}=\norm{\FF^{-1}}=1$.
  \item \textit{Zeige $G=F^{-1}$.} Sei $f\in\LL^2(\R)$, $(f_n)\in\SS(\R)$ mit
  $\norm{f_n-f}\to 0$.
\begin{align*}
(G\circ F)(f) = G(Ff) = G\left(\lim\limits_{n\to\infty} \hat{f}_n\right)
= \lim\limits_{n\to\infty} \left(\hat{f}_n\right)^{\lor} =
\lim\limits_{n\to\infty} f_n = f.
\end{align*}
Also ist $G\circ F = \id$. Analog folgt $F\circ G=\id$.
\item $\norm{Ff}_2 = \norm{f}_2$.
\begin{align*}
\norm{f}_2 = \norm{G(Ff)}_2 \le \norm{Ff}_2 \le \norm{f}_2.
\end{align*}
Mit Polarisation folgt,
\begin{align*}
\lin{Ff,Fg} = \lin{f,g}.\qedhere
\end{align*}
\end{enumerate}
\end{proof}

\begin{bem}[Achtung.]
\label{bem:2.25}
Für $f\in\LL^2(\R)$ konvergiert das Integral
\begin{align*}
\int\limits_{-\infty}^\infty f(x)e^{i\omega x}\dx = 
\lim\limits_{a\to-\infty} \int\limits_a^0 f(x)e^{i\omega x}\dx
+ \lim\limits_{b\to\infty} \int\limits_0^b f(x)e^{i\omega x}\dx
\end{align*}
im Allgemeinen nicht. Abhilfe leistet
\begin{align*}
Ff = \frac{1}{\sqrt{2\pi}}\lim\limits_{R\to\infty}\int\limits_{-R}^R
f(x)e^{i\omega x}\dx,
\end{align*}
wobei es sich hier lediglich um $\LL^2$-Konvergenz handelt.\maphere
\end{bem}

\begin{prop}
\label{prop:2.26}
Für $f\in\LL^2(\R)$ gilt,
\begin{align*}
\norm{Ff - \frac{1}{2\pi}\int\limits_{-R}^R f(x)e^{-i\omega x}\dx}_2 \to
0.\fishhere
\end{align*}
\end{prop}
\begin{proof}
\begin{enumerate}[label=\arabic{*}.)]
  \item Setze $f_R := \chi_{[-R,R]}f$, dann gilt
\begin{align*}
&\norm{f-f_R}_2 \to 0,\quad\text{für }R\to\infty,\\
\Rightarrow & \norm{Ff-Ff_R}_2 \to 0.
\end{align*}
\item Wähle nun $R$ fest. Zeige 
\begin{align*}
Ff_R = \frac{1}{\sqrt{2\pi}}\int\limits_{-R}^R f_R(x)e^{-i\omega x}\dx =:
\hat{f}_R(\omega).
\end{align*}
Setze $f_n = j_{1/n}*f_R\in C_0^\infty(\R)\subseteq\SS(\R)$.
\begin{enumerate}[label=\alph{*})]
  \item $\norm{\hat{f}_n-Ff_R}_2 = \norm{f_n - f_R}_2 \to 0$.
  \item 
\begin{align*}
\abs{\hat{f}_n(\omega) - \hat{f}_R(\omega)}
&\le \frac{1}{\sqrt{2\pi}}\int\limits_{-R-1/n}^{R+1/n}
\abs{f_n(x)-f_R(x)}\abs{e^{-i\omega x}}\dx\\
&\overset{\small\text{Hölder}}{\le}
\frac{1}{\sqrt{2\pi}}\norm{f_n-f_R}_2\sqrt{2R+\frac{2}{n}} \to 0,
\end{align*}
gleichmäßig bezüglich $\omega$.
\item $\forall \ph\in C_0^\infty(\R) : \lin{Ff_R-\hat{f}_R,\ph}_2 = 0$, denn
\begin{align*}
&\lin{Ff_R,\ph}_2 = \lim\limits_{n\to\infty} \lin{\hat{f}_n,\ph},\\
&\lin{\hat{f}_R,\ph}_2 = 
\int\limits_{\supp \ph} \hat{f}_R(\omega)\overline{\ph(\omega)}\domega =
\lim\limits_{n\to\infty} \int\limits_{\supp\ph}
\hat{f}_n(\omega)\overline{\ph(\omega)}\domega,\\
\Rightarrow & \lin{Ff_r,\ph} = \lin{f_R,\ph}.
\end{align*}
\item Da $C_0^\infty(\R)$ dicht in $\LL^2(\R)$ existiert eine Folge $(\ph_j)$
in $C_0^\infty$ mit
\begin{align*}
\norm{\ph_j - Ff_R}_2 \to 0.
\end{align*}
Daher gilt
\begin{align*}
&0 = \lin{Ff_R-\hat{f}_R,\ph_j} \to 
\lin{Ff_R-\hat{f}_R,Ff_R-\hat{f}_R}\\
\Rightarrow &\norm{Ff_r-\hat{f}_R}_2 \to 0\\
 \Rightarrow& Ff_R = \hat{f}_R.\qedhere
\end{align*}
\end{enumerate}
\end{enumerate}
\end{proof}

\begin{bsp}
\label{bsp:2.27}
Betrachte die Funktion
\begin{align*}
f(x) = \frac{x}{x^2+a^2}.
\end{align*}
$f$ ist stetig also messbar und $\abs{f}^2$ fällt wie $x^2$, also ist $f\in
\LL^2(\R)$. Wir können die Fouriertransformation
\begin{align*}
\lim\limits_{R\to\infty}\int\limits_{-R}^R \frac{x}{x^2+a^2}e^{-i\omega x}\dx
\end{align*}
berechnen, indem wir den Residuensatz verwenden. Setze
\begin{align*}
g(z) = \frac{ze^{i\omega z}}{z^2+a^2} = 
\frac{ze^{i\omega z}}{(z+ia)(z-ia)}. 
\end{align*}
Wir müssen eine Fallunterscheidung für $\omega$ machen.
Für $\omega > 0$

\begin{figure}[!htbp]
\begin{pspicture}(-0.1,-1.3092188)(2.8496876,1.3292187)
\psline{->}(0.0,-0.08921875)(2.54,-0.08921875)
\psline{->}(1.22,-1.2892188)(1.22,1.0707812)
\psline(1.12,-0.76921874)(1.34,-0.76921874)
\psline(2.22,0.03078125)(2.22,-0.30921876)
\psline(0.22,0.05078125)(0.22,-0.28921875)
\psbezier[linecolor=darkblue](0.22,-0.08921875)(0.22,-0.08921875)(2.22,-0.08921875)(2.22,-0.08921875)(2.22,-0.08921875)(2.22,-1.0492188)(1.22,-1.0692188)(0.22,-1.0892187)(0.22,-0.08921875)(0.22,-0.08921875)

\rput(2.6117187,-0.31921875){\color{gdarkgray}$\Re$}
\rput(1.0770313,-0.29921874){\color{gdarkgray}$0$}
\rput(0.85,-0.7392188){\color{gdarkgray}$-a$}
\rput(2.2785938,0.18078125){\color{gdarkgray}$R$}
\rput(0.1,0.18078125){\color{gdarkgray}$-R$}
\rput(2.4,-0.69921875){\color{gdarkgray}$\gamma_R$}
\rput(0.9546875,1.1607813){\color{gdarkgray}$\Im$}
\end{pspicture} 
\end{figure}


\begin{align*}
\int_{\gamma_R} g(z)\dz &= 
2\pi\nu(\gamma, -ia)\Res(g,-ia)
= -2\pi i\frac{-ia e^{-i\omega(-ia)}}{-ia-ia}\\
&= -\pi ie^{-\omega a}.
\end{align*}
Zeige nun, dass das Integral über den Kreisbogen verschwindet,
\begin{align*}
\Rightarrow
\lim\limits_{R\to\infty}\int\limits_{-R}^R \frac{x}{x^2+a^2}e^{-i\omega x}\dx
= -\pi ie^{-\omega a}.
\end{align*}
Verfahre analog für $\omega < 0$,
\begin{align*}
Ff(\omega) =-\pi ie^{-\abs{\omega} a}.\bsphere
\end{align*}
\end{bsp}

\begin{cor}
\label{prop:2.28}
Für $f\in\SS(\R)$ gilt
\begin{align*}
\norm{\hat{f}}_\infty \le \frac{1}{\sqrt{2\pi}}\norm{f}_1.\fishhere
\end{align*}
\end{cor}
\begin{proof} Für $\omega\in\R$ gilt,
\begin{align*}
\abs{\hat{f}(\omega)} \le \frac{1}{\sqrt{2\pi}}\int_\R
\abs{f(x)}\abs{e^{-i\omega x}}\dx = \frac{1}{\sqrt{2\pi}}\norm{f}_1.
\end{align*}
Die rechte Seite hängt nicht mehr von $\omega$ ab, also gilt auch,
\begin{align*}
\norm{\hat{f}}_\infty \le\frac{1}{\sqrt{2\pi}}\norm{f}_1.\qedhere
\end{align*}
\end{proof}

Nun stellt sich die Frage, ob $\frac{1}{\sqrt{2\pi}}$ die optimale Konstante
ist.

\begin{prop}
\label{prop:2.29}
Für $F: (\SS(\R),\norm{\cdot}_1)\to (\SS(\R),\norm{\cdot}_\infty),\; f\mapsto
\hat{f}$, gilt $\norm{F}=\frac{1}{\sqrt{2\pi}}$.\fishhere
\end{prop}
\begin{proof}
Sei $f_n = j_{1/n}$, so gilt
\begin{align*}
\hat{f}_n(\omega) &= \frac{1}{\sqrt{2\pi}}
\int\limits_{-\infty}^\infty j_{1/n}(x)e^{-i\omega x}\dx
\overset{y=-x}{=} \frac{1}{\sqrt{2\pi}}\int\limits_{-\infty}^\infty
j_{1/n}(-y)e^{i\omega y}\dy\\
&= \frac{1}{\sqrt{2\pi}}j_{1/n}*e^{i\omega \cdot}(0)
\overset{n\to\infty}{\longrightarrow} \frac{e^{i\omega 0}}{\sqrt{2\pi}} =
\frac{1}{\sqrt{2\pi}}.
\end{align*}
Nun ist $\norm{f_n}_1 = 1$, somit folgt
\begin{align*}
\frac{\norm{f_n}_\infty}{\norm{f_n}_1} =
\frac{1}{\sqrt{2\pi}}j_{1/n}*e^{i\omega \cdot}(0) \to \frac{1}{\sqrt{2\pi}}
\end{align*}
Daher ist auch
\begin{align*}
\sup\limits_{f\neq 0} \frac{\norm{f}_\infty}{\norm{f}_1}
\ge \frac{1}{\sqrt{2\pi}}.\qedhere
\end{align*}
\end{proof}

\begin{prop}
\label{prop:2.30}
Die Fouriertransformation besitzt genau eine lineare beschränkte Fortsetzung
\begin{align*}
F: \LL^1(\R)\to C_B(\R) :=\setdef{f\in C(\R\to\C)}{\sup_\R \abs{f} <\infty},
\end{align*}
und es gilt $\norm{F}=\frac{1}{\sqrt{2\pi}}$.\fishhere
\end{prop}
\begin{proof}
Fortsetzungssatz $L=\LL^1(\R)$, $B=(C_B(\R),\norm{\cdot}_\infty)$, $T=\FF$,
$D(T)=\SS(\R)$.\qedhere
\end{proof}

\begin{prop}
\label{prop:2.31}
\begin{enumerate}
  \item Für $f\in\LL^1(\R)$ gilt,
\begin{align*}
Ff(\omega) = \frac{1}{\sqrt{2\pi}}\int\limits_{-\infty}^\infty f(x)e^{-i\omega
x}\dx.
\end{align*}
\item $\im F\subseteq \setdef{f\in C(\R\to\C)}{f(x)\to0,x\to\pm\infty}$.
\end{enumerate}
\end{prop}
\begin{proof}
\begin{enumerate}[label=\arabic{*}.)]
  \item Sei $(f_n)$ Folge in $\SS(\R)$ mit $\norm{f_n-f}_1 \to 0$.
\begin{enumerate}[label=\alph{*})]
  \item $\forall \omega\in\R: \hat{f}_n(\omega) \to Ff(\omega)$, denn
\begin{align*}
\norm{\hat{f}_n - Ff}_\infty \le \frac{1}{\sqrt{2\pi}}\norm{f_n-f}_1 \to 0.
\end{align*}
  \item $\forall \omega\in\R: \hat{f}_n(\omega) \to \frac{1}{\sqrt{2\pi}}\int\limits_{-\infty}^\infty
f(x)e^{-i\omega x}\dx$, denn
\begin{align*}
&\abs{\hat{f}_n(\omega)-\frac{1}{\sqrt{2\pi}}\int\limits_{-\infty}^\infty
f(x)e^{-i\omega x}\dx }\\
&\le
\frac{1}{\sqrt{2\pi}}\int\limits_{-\infty}^\infty
\abs{f_n(x)-f(x)}\abs{e^{-i\omega x}}\dx = \frac{1}{\sqrt{2\pi}}\norm{f_n-f}_1.
\end{align*}
\end{enumerate}
  \item Sei $f\in\LL^1(\R)$, $\ep >0$, dann
\begin{align*}
&\exists \ph_\ep\in \SS(\R) : \norm{f-\ph_\ep}_1<\ep,\\
\Rightarrow &
\abs{Ff(\omega)-\hat{\ph}_\ep(\omega)} = 
\frac{1}{\sqrt{2\pi}}\norm{f-\ph_\ep}_1
< \frac{\ep}{\sqrt{2\pi}}.
\end{align*}
Wegen $\hat{\ph}_\ep\in\SS(\R)$ existiert ein  $M_\ep > 0$, sodass für
$\abs{\omega}>M_\ep$ folgt,
\begin{align*}
&\abs{\ph_\ep(\omega)} < \frac{\ep}{2}.
\end{align*}
Für $\omega$ mit $\abs{\omega}>M_\ep$ gilt somit,
\begin{align*}
\Rightarrow &
\abs{Ff(\omega)} \le \abs{Ff(\omega)-\hat{\ph}(\omega)} +
\abs{\hat{\ph}(\omega)} < \frac{\ep}{\sqrt{2\pi}} + \frac{\ep}{2} < \ep,\\
\Rightarrow&
Ff(\omega) \to 0,\quad \text{für }\omega\to\pm\infty.\qedhere
\end{align*}
\end{enumerate}
\end{proof}

\begin{prop}
\label{prop:2.32}
Sei $f\in\LL^1(\R)$, $x\in\R$ und
\begin{align*}
\exists \delta > 0 \int\limits_{-\delta}^\delta 
\abs{\frac{f(t+x)-f(x)}{t}}\dt < \infty.\quad\text{Dini Bedingung}
\end{align*}
Dann konvergiert das uneigentliche Integral
\begin{align*}
f(x) = \frac{1}{\sqrt{2\pi}}\int\limits_{-\infty}^\infty Ff(\omega)e^{i\omega
x}\domega.\fishhere
\end{align*}
\end{prop}
\begin{proof}
Wir zeigen nur $f(x) = \lim\limits_{R\to\infty} \frac{1}{\sqrt{2\pi}}
\int\limits_{-R}^R Ff(\omega)e^{i\omega
x}\domega$.
\begin{align*}
&\frac{1}{\sqrt{2\pi}}\int\limits_{-R}^R Ff(\omega)e^{i\omega x}\domega =
\frac{1}{2\pi} \int\limits_{-R}^R \int_\R f(y)e^{-i\omega y}\dy e^{i\omega
x}\domega
\\ &
\overset{\text{Fubini II}}{=} \frac{1}{2\pi} 
\int\limits_{-\infty}^\infty f(y) 
\int\limits_{-R}^R e^{i\omega(x-y)}\domega \dy
= \frac{1}{\pi}\int\limits_{-\infty}^\infty f(y)\frac{\sin(R(y-x))}{y-x}\dy\\
&\overset{t=y-x}{=} \frac{1}{\pi} \int\limits_{-\infty}^\infty f(t+x)\frac{\sin
Rt}{t}\dt.
\end{align*}
Nun ist
\begin{align*}
&\int\limits_{-\infty}^\infty \frac{\sin Rt}{t}\dt = \pi,\quad \text{für }R>0.\\
\end{align*}
\begin{align*}
\Rightarrow & 
\abs{\frac{1}{\pi}\int\limits_{-\infty}^\infty f(t+x)\frac{\sin Rt}{t}} =
\abs{\frac{1}{\pi} \int\limits_{-\infty}^\infty \left(f(t+x)-f(x)\right)
\frac{\sin Rt}{t}\dt} < \ep,
\end{align*}
für $R > R_\ep$. Wegen
\begin{align*}
&\abs{\int\limits_{\abs{t}\ge M} f(t+x)\frac{\sin Rt}{t}\dt }
\le \frac{1}{M} \norm{f}_1 < \frac{\ep}{3},\qquad\text{für }M>M_\ep.\\
&\abs{\int\limits_{\abs{t}\ge M} f(x)\frac{\sin Rt}{t}\dt}  = 
\abs{f(x)}\abs{\int\limits_{\abs{t}\ge M} \frac{\sin Rt}{t} \dt} <
\frac{\ep}{3},\quad\text{für } M>M_\ep,
\end{align*}
da $\int\limits_0^\infty \frac{\sin Rt}{t}\dt $ konvergiert.
\begin{align*}
\int\limits_{\abs{t}\le M}\frac{f(t+x)-f(x)}{t}\sin Rt\dt \to 0,\quad \text{für
}R\to\infty.
\end{align*}
Siehe \ref{prop:1.51}\qedhere
\end{proof}

\begin{prop}
\label{prop:2.33}
\begin{enumerate}[label=\arabic{*}.)]
  \item Falls $f\in\LL^1(\R)$ und $x\mapsto x^j f(x)\in\LL^1(\R)$, dann gilt
\begin{align*}
&Ff \in C^j(\R\to\C),\\
&Ff^{(k)}(\omega) \to 0,&& \text{für }\abs{\omega}\to\infty,\quad
k=0,1,\ldots,j.
\end{align*}
\item Falls $f\in C^{j-1}(\R\to\C)$ und $f^{(j)}\in\LL^1(\R)$ und
\begin{align*}
f^{(k)}(x)\to 0,&& \text{für } \abs{x}\to\infty,\quad k=0,1,\ldots,j,
\end{align*}
dann gilt
\begin{align*}
Ff^{(k)}(\omega) = i\omega^k Ff(\omega),&& k=0,1,\ldots,j,
\end{align*}
insbesondere gilt $\omega^j Ff(\omega)\to 0$ für $\omega\to\pm\infty$.\fishhere
\end{enumerate}
\end{prop}
\begin{proof}
\begin{enumerate}[label=\arabic{*}.)]
  \item Sei $j=1$. Der Differenzenquotient ist
\begin{align*}
\frac{Ff(\omega+h)-Ff(\omega)}{h} =
\frac{1}{\sqrt{2\pi}}\int\limits_{-\infty}^\infty
f(x)\frac{e^{-ix(\omega+h)}-e^{i\omega x}}{h}\dx.
\end{align*}
Der Integrand konvergiert punktweise gegen $-ixe^{-ix\omega}$. Es gilt
\begin{align*}
\abs{f(x)\frac{e^{-ix(\omega+h)}-e^{-i\omega x}}{h}}
\overset{\text{MWS}}{=}
\abs{f(x)(-ix)e^{-ix\th}} = \abs{x f(x)}.
\end{align*}
$xf(x)\in\LL^1$, wir können also majorisierte Konvergenz anwenden und
erhalten,
\begin{align*}
\ldots \to \frac{1}{\sqrt{2\pi}}\int\limits_{-\infty}^\infty
(-ix)f(x)e^{-i\omega x}\dx = F((-ix)f(x)).
\end{align*}
Mit \ref{prop:2.30} und \ref{prop:2.31} folgt
\begin{align*}
(Ff)'(\omega) = F(\LL^1\text{-Funktion}) \in C(\R\to\C). 
\end{align*}
\item Sei $j=1$.
\begin{align*}
\sqrt{2\pi} (Ff')(\omega) &= \int\limits_{-\infty}^\infty f'(x)e^{-i\omega x}\dx
\\ &\overset{\small\text{part.int.}}{=} \underbrace{f(x)e^{i\omega
x}\bigg|_{-\infty}^\infty}_{=0} + \int\limits_{-\infty}^\infty
f(x)(-i\omega)e^{-i\omega x}\dx \\ &= i\omega Ff(\omega)\sqrt{2\pi}.\qedhere
\end{align*}
\end{enumerate}
\end{proof}

\subsection{Fouriertransformation im $\R^n$}
Wir haben bisher alle Sätze und Beweise in $\R$ geführt, man kann sie jedoch 
problemlos auf den $\R^n$ verallgemeinern.

\begin{bemn}[Erinnerung.]
Sei $f:\R^n\to\C$, dann
\begin{align*}
&\partial_{x_j}f(x) = \frac{\partial f}{\partial x_j}(x) = \lim\limits_{h\to 0}
\frac{f(x+he_j)-f(x)}{h}.\\
&\frac{\partial^2f}{\partial x_j \partial x_j} = \frac{\partial }{\partial
x_j}\frac{\partial f}{\partial x_k}.
\end{align*}
Für $\alpha_1,\ldots,\alpha_n\in\N_0$ ist dann
\begin{align*}
\frac{\partial^{\alpha_1+\alpha_2+\ldots+\alpha_n}}{\partial
x_1^{\alpha_1}\partial x_2^{\alpha_2}\cdots \partial x_n^{\alpha_n}}.
\end{align*}
\end{bemn}
Da diese Schreibweise sehr aufwändig ist, wollen wir eine Abkürzung einführen.
\begin{defn}
\label{defn:2.34}
$\alpha\in\N_0^n$ heißt \emph{Multiindex}. Setze
\begin{align*}
\abs{\alpha} = \alpha_1 + \alpha_2 + \ldots + \alpha_n.
\end{align*}
Für $\alpha,\beta\in\N_0^n$ und $\alpha,\beta$ vergleichbar gilt,
\begin{align*}
\alpha\le \beta \Leftrightarrow \alpha_1\le \beta_1, \ldots, \alpha_n\le.
\beta_n.
\end{align*}
Die Multiindize-Potenz ist definiert als,
\begin{align*}
x^\alpha := x_1^{\alpha_1} \cdot x_2^{\alpha_2} \cdots x_n^{\alpha_n},\qquad
x\in\R^n.
\end{align*}
Die Ableitung hat nun die Form,
\begin{align*}
D^\alpha f = \frac{\partial^{\abs{\alpha}}f}{\partial x_1^{\alpha_1}\cdots
\partial x_2^{\alpha_2}\cdots x_n^{\alpha_n}}.
\end{align*}
Der Multiindize-Binominalkoeffizient ist
\begin{align*}
\binom{\alpha}{\beta} =
\binom{\alpha_1}{\beta_1}\binom{\alpha_2}{\beta_2}\cdots
\binom{\alpha_n}{\beta_n}.
\end{align*}
\end{defn}

\begin{bspn}
Im $\R^3$ sei $\alpha=(0,1,4)$, dann ist
\begin{align*}
D^\alpha f = \frac{\partial^5}{\partial^1 x_1\partial^4 x_3}.\bsphere
\end{align*}
\end{bspn}

\begin{prop}[Anwendung]
\label{prop:2.35}
\begin{align*}
&\abs{\alpha}+\abs{\beta}\\
&(\lambda x)^\alpha = \lambda^{\abs{\alpha}}x^\alpha\\
&D^\alpha D^\beta f = D^{\alpha+\beta}f,\qquad f\in C^{\abs{\alpha+\beta}}.
\end{align*}
\emph{Leibniz Regel}
\begin{align*}
D^\alpha (f\cdot g) = \sum\limits_{\beta\le \alpha} \binom{\alpha}{\beta}
\left(D^{\alpha-\beta}f\right)\left(D^\beta g\right).\fishhere
\end{align*}
\end{prop}

\begin{defn}
\label{defn:2.36}
Der Schwarzraum im $\R^n$ ist gegeben durch,
\begin{align*}
\SS(\R^n) := \setdef{f\in C^\infty(\R^n\to\C)}{\forall x\in\N_0^n \forall
f\in\N : \sup\limits_{x\in\R^n} \abs{x}^j \abs{D^\alpha f(x)}<\infty}.
\end{align*}
Für $f\in\SS(\R)$ gilt
\begin{align*}
\hat{f}(\omega) = \frac{1}{(2\pi)^{n/2}} \int_{\R^n} f(x)e^{-i\omega\cdot
x}\dx.\fishhere
\end{align*}
\end{defn}

\begin{prop}
\label{prop:2.37}
\begin{enumerate}[label=\arabic{*}.)]
  \item $\FF: \SS(\R^n)\to\SS(\R^n),\; f\mapsto \hat{f}$ ist bijektiv mit
  Umkehrabbildung,
\begin{align*}
&\FF^{-1}: \SS(\R^n)\to\SS(\R^n),\; f\mapsto \check{f},\quad\\
&\check{f}(x) = \frac{1}{(2\pi)^{n/2}} \int_{\R^n} f(x)e^{i\omega\cdot
x}\domega.
\end{align*}
\item
\begin{align*}
&\widehat{D^\alpha f}(\omega) = (i\omega)^\alpha \hat{f}(\omega) =
i^{\abs{\alpha}} \omega^\alpha \hat{f}(\omega),\\
&\widehat{x^\alpha f} = i^{\abs{\alpha}}D^\alpha \hat{f}.
\end{align*}
\item Mit $f*g(x) \int_\R^n f(x-y)g(y)\dy$ folgt,
\begin{align*}
&\widehat{f\cdot g} = \frac{1}{(2\pi)^{n/2}}\hat{f}*\hat{g},\\
&\widehat{f*g} = (2\pi)^{n/2} \hat{f}\cdot\hat{g}.\fishhere
\end{align*}
\end{enumerate}
\end{prop}

\begin{bsp}
\label{bsp:2.38}
Sei $f\in\SS(\R^n)$. Löse die inhomogene Laplace Gleichung,
\begin{align*}
-\Delta u + u = f,\qquad u\in\SS(\R^n).
\end{align*}
Wenden wir die Fouriertransforamtion auf die Diffrentialgleichung an,
\begin{align*}
&-\sum\limits_{j=1}^n i^2 \omega_j^2 \hat{u} + \hat{u} = \hat{f}\\
\Leftrightarrow & \left(\abs{\omega}^2 +1\right)\hat{u} =\hat{f},\\
\Rightarrow &\hat{u} = \frac{1}{\abs{\omega}^2 + 1}\\
\Rightarrow &u(x) = \frac{1}{(2\pi)^{n/2}}\int\limits_{\R^n}
\frac{1}{\abs{\omega}^2+1}\hat{f}(\omega) e^{i\omega x}\dx.
\end{align*}
Man kann zeigen, dass höchstens eine Lösung $u\in\SS(\R^n)$ existiert.\bsphere
\end{bsp}

\begin{prop}[Fortsetzung]
\label{prop:2.39}
\begin{enumerate}[label=\arabic{*}.)]
  \item $F: \LL^2(\R^n)\to \LL^2(\R^n)$ ist unitär, d.h.
\begin{align*}
\lin{Ff,Fg} = \lin{f,g}.
\end{align*}
Es gilt
\begin{align*}
Ff = \LL^2(\R^n)-\lim\limits_{R\to\infty}
\frac{1}{(2\pi)^{n/2}}\int\limits_{\abs{x}\le R}f(x)e^{-ix\cdot}\dx.
\end{align*}
\item $F: \LL^1(\R^n)\to C_\infty(\R^n) := \setdef{f\in
C^\infty(\R^n\to\C)}{f(x)\to 0\text{ für }\abs{x}\to\infty}$ und es gilt,
\begin{align*}
\norm{F}_\infty := \max\limits_{\omega\in\R^n}\abs{Ff(\omega)} \le
\frac{1}{(2\pi)^{n/2}}\norm{f}_1.\fishhere
\end{align*}
\end{enumerate}
\end{prop}

\subsection{Interpolation}

Wir wollen jetzt die Frage stellen, ob man die
Fouriertransformation noch weiter fortsetzen kann (z.B. auf $\LL^p$)? Um den
Fortsetzungssatz anwenden zu können, benötigen wir, dass $F$ beschränkt ist. Für
$\LL^p$ und $1<p<2$ können wir dies aus Satz \ref{prop:2.39} gewinnen. Dazu
kann man den $\LL^p$ aus $\LL^1$ und $\LL^2$ \emph{interpolieren}.

Die Interpolation von Banachräumen ist sehr wichtiges Konzept für das Lösen von
partiellen Differentialgleichungen. Man definiert dazu banachwertige
Funktionen, die von komplexen Parametern abhängen. Sie ist stark mit der
Funktionentheorie verknüpft.

\begin{prop}[Interpolation von Banachräumen]
\label{prop:2.40}
\begin{align*}
&B_0 = L^q(\R^n), && 1\le 1\le \infty,\\
&B_1 = L^p(\R^n), && 1\le p\le \infty,\\
\Rightarrow & B_t = L^{r_t}(\R^n), &&
\frac{1}{r_t} = \frac{t}{p} + \frac{1-t}{q},\; \SS(\R^n)\text{ dicht in
}L^{r_t}(\R^n).\fishhere
\end{align*}
\end{prop}
%TODO: Bild Interpolation

\begin{bspn}
Setze $q=2$, $p=1$, dann ist
\begin{align*}
\frac{1}{r_t} = \frac{t}{1}+\frac{1-t}{2} = \frac{1+t}{2} \Leftrightarrow r_t =
\frac{2}{1+t}.
\end{align*}
Im Fall $q=2$, $p=\infty$, ist
\begin{align*}
\frac{1}{r_t} = \frac{t}{\infty} + \frac{1-t}{2} = \frac{1-t}{2}
\Leftrightarrow r_t = \frac{2}{1-t}.\bsphere
\end{align*}
\end{bspn}

\begin{prop}[Satz von Riesz-Thorin]
\label{prop:2.41}
Sei $T: \LL^q(\R^n) \to \LL^{\tilde{q}}(\R^n)$ linear und beschränkt,
\begin{align*}
\norm{T}_{q,\tilde{q}} = M_0,
\end{align*}
und $T: \LL^p(\R^n) \to \LL^{\tilde{p}}(\R^n)$ linear und beschränkt,
\begin{align*}
\norm{T}_{p,\tilde{p}} = M_1,
\end{align*}
dann ist $T$ von $\LL^p\cap \LL^q\supseteq \SS(\R)$ auf $B_t$ eindeutig
fortsetzbar,
\begin{align*}
T: B_t \to \tilde{B}_t
\end{align*}
ist linear und beschränkt und
\begin{align*}
\norm{T}_{B_t\to \tilde{B}_t} \le M_0^{1-t}M_1^t.\fishhere
\end{align*}
\end{prop}

\begin{bemn}[Anwendung.]
\begin{align*}
&F : \LL^2(\R^n)\to \LL^2(\R^n),\quad \norm{T}_{2,2} = 1\\
&F : \LL^1(\R^n)\to \LL^\infty(\R^n),\quad \norm{T}_{1,\infty} =
\frac{1}{(2\pi)^{n/2}}\\
\Rightarrow & F: B_t\to \tilde{B}_t,
\end{align*}
mit
\begin{align*}
&B_t = \LL^{r_t}(\R^n), r_t = \frac{2}{1+t}\\
&\tilde{B}_t = \LL^{\tilde{r}_t}(\R^n), \tilde{r}_t = \frac{2}{1-t},\\
&\norm{B_t\to \tilde{B}_t} \le 1^{1-t}\left(\frac{1}{(2\pi)^{n/2}}\right)^t.
\end{align*}
\textit{Beachte.}
\begin{align*}
&\frac{1}{r_t} + \frac{1}{\tilde{r}_t} = \frac{1+t}{2} + \frac{1-t}{2} = 1,\quad
1\le r_t\le 2,\\
&p = r_t \Leftrightarrow \frac{1}{p} = \frac{1+t}{2} \Leftrightarrow t = 2p-1
\end{align*}
Es gilt somit
\begin{align*}
\Rightarrow &F: \LL^p(\R^n) \to \LL^q(\R^n),\quad 1\le p\le
2,\quad \frac{1}{p}+\frac{1}{q} = 1\\
&\norm{F}_{\LL^p\to\LL^{q}} \le \left(\frac{1}{(2\pi)^{n/2}}\right)^{2p-1} =
\frac{1}{(2\pi){n(p-1/2)}}.\maphere
\end{align*}
\end{bemn}

\begin{prop}[Satz von Hausdorff-Young]
\label{prop:2.42}
Sei $1\le p\le 2$, $\frac{1}{p}+\frac{1}{q}=1$. Die Fouriertransformation
besitzt eine eindeutige lineare beschränkte Fortsetzung
\begin{align*}
F: \LL^p(\R^n)\to \LL^q(\R^n)
\end{align*}
und es gilt $\norm{F}\le\dfrac{1}{(2\pi)^{n(1/p-1/2)}}$.\fishhere
\end{prop}